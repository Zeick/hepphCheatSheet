%#!platexmake CheatSheet
%%% Time-Stamp: <2014-09-29 23:10:21 misho>
%%% 一部で日本語が使用されています。

\section{Cherry on the Cake}
\paragraph{Conversion of Units}
\begin{align}
 1\GeV &= \frac1{6.5821\EE{-25}\un{s}} = \frac1{2.086\EE{-32}\un{yr}}
        = \frac1{0.19733\un{fm}}  = 1.1605\EE{13}\un{K} = 1.7827\EE{-24}\un{g}\notag\\
       &= \frac{1.519268\EE{24}}{1\un{s}} = \frac{4.79\EE{31}}{1\un{yr}}
        = \frac{5.0677}{1\un{fm}}\\
1\un{K}&= 8.6173\EE-5\un{eV} = \frac1{8.0591\EE{-21}\un{s}} = \frac{1.2408\EE{20}}{1\un{s}}\\
1\GeV^{-2}&= 3.8938\EE-4\un{barn} = \frac{1\un{barn}}{2568.2}=1.1683\EE{-17}\un{cm^3/s}\qquad(1\un{barn}=10^{-28}\un{m^2} = 100\un{fm^2})
\end{align}
\vspace{-2zw}
\begin{equation}\begin{split}
 &1\un{\text{tropical yr}} = 3.1557\EE7\un{s},\quad 1\un{\text{sidereal yr}} = 3.1558\EE7\un{s};\\
 &1\un{s} = 3.1689\EE-8\un{\text{tr-yr}} = 3.1688\EE-8\un{\text{sr-yr}}.
\end{split}\end{equation}
\paragraph{Physical Constants}
\begin{align}
 G\s F &= \frac{1}{\sqrt2v^2} = 1.16637(1)\EE-5\un{GeV^{-2}},\qquad
 G\s N = 6.70881(67)\EE{-39}\un{GeV^{-2}}\\
 \sqrt{G\s N} &= 1.61624(8)\EE{-35}\un{m} = \frac1{1.22089(6)\EE{19}\un{GeV}}
               = \frac1{2.17644(11)\EE-8\un{kg}}\\
 \sqrt{8\pi G\s N}&= 8.1026(4)\EE{-35}\un{m} = \frac1{2.4353(1)\EE{18}\un{GeV}}
                   = \frac1{4.3413(2)\EE-9\un{kg}}
\end{align}
\paragraph{Component of Spinor in Weyl Representation}\mbox{}\par
$(\chi_\alpha=\epsilon_{\alpha\beta}\chi^\beta,\
  \chi^\alpha=\epsilon^{\alpha\beta}\chi_\beta,\
  \chi_\dalpha=\epsilon_{\dalpha\dbeta}\chi^\dbeta,\
  \chi^\dalpha=\epsilon^{\dalpha\dbeta}\chi_\dbeta;\quad
  \epsilon^{12}=\epsilon^{\dot1\dot2}=1, \
  \epsilon_{12}=\epsilon_{\dot1\dot2}=-1.)$

\begin{align}
\psi&=\pmat{\psi\s L\\\psi\s R}=\pmat{\xi_\alpha\\\bar\chi^\dalpha}
     =\spmat{\xi_1\\\xi_2\\\bar\chi^{\dot1}\\\bar\chi^{\dot2}}&
&\buildrel{C}\over{\longrightarrow}&
\psi\cc &=
-\ii\G2\psi^*=\spmat{-\!\!&(\bar\chi^{\dot2})^*\\&(\bar\chi^{\dot1})^*\\&(\xi_2)^*\\-\!\!&(\xi_1)^*}
             =\spmat{-\!\!&\chi^2 \\ &\chi^1\\&\bar\xi_{\dot2}\\ -\!\!&\bar\xi_{\dot1}}
             =\pmat{\chi_\alpha \\ \bar\xi^\dalpha}\\
\overline\psi&=\pmat{\chi^\alpha&\bar\xi_\dalpha}
        =\spmat{\chi^1&\chi^2&\bar\xi_{\dot 1}&\bar\xi_{\dot 2}}&
&\buildrel{C}\over{\longrightarrow}&
\overline{\psi\cc} &=\ii\trans\psi\G0\G2
              =\spmat{\xi_2&-\xi_1&-\bar\chi^{\dot2}&\bar\chi^{\dot1}}
              =\pmat{\xi^\alpha&\bar\chi_{\dalpha}}
\end{align}
\begin{align}
 A^\alpha B_\alpha &= \overline\psi_{A\cc}\PL\psi_{B} = \overline\psi_{B\cc}\PL\psi_{A}&
 \bar A_\dalpha\bar B^\dalpha &= \overline\psi_A\PR\psi_{B\cc} = \overline\psi_B\PR\psi_{A\cc}
\end{align}
\begin{align}
 \overline\psi \slashed A \chi &= \overline\psi\s L\slashed A\PL\chi\s L + \overline\psi\s R\slashed A\PR\chi\s R\notag\\
                         &= \overline\psi\s L\slashed A\PL\chi\s L - \overline\chi\s R\cc\slashed A\PL\psi\s R\cc
\end{align}

%%% Local Variables:
%%% TeX-master: "CheatSheet.tex"
%%% End:
