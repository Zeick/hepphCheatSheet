%%% Time-Stamp: <2019-08-01 07:06:55 misho>
\documentclass[CheatSheet]{subfiles}

\newcommand{\OQ}{{\mathcal Q}}
\newcommand{\OD}{{\mathcal D}}
\begin{document}

\summarystyle
\section[Supersymmetry with $\eta=\diag(+,-,-,-)$]{Supersymmetry with $\bm{\eta=\mathop{\mathsf{diag}}(+,-,-,-)}$}

\input{calculator/susy/formulae.txt}

\paragraph{Superfields}


\clearpage
\detailstyle

\subsection{Lorentz symmetry as SU(2)$\times$SU(2)}


\subsection{Supersymmetry algebra}
We define the generators as
\begin{alignat}{3}
 P_\mu
 &:= \ii\partial_\mu,
\quad&
 \{\OQ_\alpha, \bar \OQ_\dalpha\}
 &= -2\ii\sigma^{\mu}_{\alpha\dalpha}\partial_\mu = -2\sigma^{\mu}_{\alpha\dalpha}P_\mu,
\quad&
  \{\OQ_\alpha, \OQ_\beta\} =   \{\bar \OQ_\dalpha, \bar \OQ_\dbeta\} = 0,
\end{alignat}
which is realized by
\begin{align*}
\OQ_\alpha &= \frac{\partial}{\partial \theta^{\alpha}}+\ii (\sigma^{\mu}\btheta){}_{\alpha}\partial_{\mu},
&
\bar \OQ_\dalpha&=-\frac{\partial}{\partial \btheta^{\dalpha}}-\ii (\theta \sigma^{\mu}){}_{\dalpha}\partial_{\mu},
&
\OQ^\alpha&=-\frac{\partial}{\partial \theta_{\alpha}}-\ii (\btheta\bsigma^{\mu})^{\alpha}\partial_{\mu},
&
\bar \OQ^{\dalpha}&=\frac{\partial}{\partial \btheta_{\dalpha}}+\ii (\bsigma^{\mu}\theta)^{\dalpha}\partial_{\mu},
\\
\OD_\alpha&=\frac{\partial}{\partial \theta^{\alpha}}-\ii (\sigma^{\mu}\btheta){}_{\alpha}\partial_{\mu},
&
\bar \OD_\dalpha&=-\frac{\partial}{\partial \btheta^{\dalpha}}+\ii (\theta \sigma^{\mu}){}_{\dalpha}\partial_{\mu},
&
\OD^\alpha&=-\frac{\partial}{\partial \theta_{\alpha}}+\ii (\btheta\bsigma^{\mu})^{\alpha}\partial_{\mu},
&
\bar \OD^\dalpha&=\frac{\partial}{\partial \btheta_{\dalpha}}-\ii (\bsigma^{\mu}\theta)^{\dalpha}\partial_{\mu};
\end{align*}
$\OD_\alpha$ etc.~works as covariant derivatives because of the commutation relations
\begin{align*}
\{\OD_\alpha, \bar \OD_\dalpha\}&=+2\ii\sigma^{\mu}_{\alpha\dalpha}\partial_\mu,&
\{\OQ_\alpha, \OD_\beta\}&=
\{\OQ_\alpha, \bar \OD_\dbeta\}=
\{\bar \OQ_\dalpha, \OD_\beta\}=
\{\bar \OQ_\dalpha, \bar \OD_\dbeta\}=
\{\OD_\alpha, \OD_\beta\}=
\{\bar \OD_\dalpha, \bar \OD_\dbeta\}=0.
\end{align*}

\paragraph{Derivative formulae}

\begin{align*}
&
\epsilon^{\alpha\beta}\frac{\partial}{\partial\theta^\beta}=-\frac{\partial}{\partial\theta_\alpha}
&&
\frac{\partial}{\partial \theta^{\alpha}}\theta \theta =2 \theta_{\alpha}
&&
\frac{\partial}{\partial \theta^{\alpha}}\frac{\partial}{\partial \theta_{\beta}}\theta \theta =-2 \delta^{\beta}_{\alpha}
&&
\frac{\partial}{\partial \btheta^{\dalpha}}\frac{\partial}{\partial \btheta_{\dbeta}}\btheta\btheta=2 \delta^{\dbeta}_{\dalpha}
\\&
\epsilon_{\alpha\beta}\frac{\partial}{\partial\theta_\beta}=-\frac{\partial}{\partial\theta^\alpha}
&&
\frac{\partial}{\partial \theta_{\alpha}}\theta \theta =-2 \theta^{\alpha}
&&
\frac{\partial}{\partial \theta_{\alpha}}\frac{\partial}{\partial \theta_{\beta}}\theta \theta =2 \epsilon^{\alpha \beta}
&&
\frac{\partial}{\partial \btheta_{\dalpha}}\frac{\partial}{\partial \btheta_{\dbeta}}\btheta\btheta=-2 \epsilon^{\dalpha\dbeta}
\\&
\epsilon^{\dalpha\dbeta}\frac{\partial}{\partial\btheta^\dbeta}=-\frac{\partial}{\partial\btheta_\dalpha}
&&
\frac{\partial}{\partial \btheta_{\dalpha}}\btheta\btheta=2 \btheta^{\dalpha}
&&
\frac{\partial}{\partial \theta_{\alpha}}\frac{\partial}{\partial \theta^{\beta}}\theta \theta =2 \delta^{\alpha}_{\beta}
&&
\frac{\partial}{\partial \btheta_{\dalpha}}\frac{\partial}{\partial \btheta^{\dbeta}}\btheta\btheta=-2 \delta^{\dalpha}_{\dbeta}
\\&
\epsilon_{\dalpha\dbeta}\frac{\partial}{\partial\btheta_\dbeta}=-\frac{\partial}{\partial\btheta^\dalpha}
&&
\frac{\partial}{\partial \btheta^{\dalpha}}\btheta\btheta=-2 \btheta_{\dalpha}
&&
\frac{\partial}{\partial \theta^{\alpha}}\frac{\partial}{\partial \theta^{\beta}}\theta \theta =-2 \epsilon_{\alpha \beta}
&&
\frac{\partial}{\partial \btheta^{\dalpha}}\frac{\partial}{\partial \btheta^{\dbeta}}\btheta\btheta=2 \epsilon_{\dalpha\dbeta}
\end{align*}

In addition, we define
\begin{align}
 & (y,\theta',\btheta'):=(x-\ii\theta\sigma^\mu\btheta,\theta,\btheta):
 \\&
\bar\OD_\dalpha = -\pd{}{\btheta^{\prime\dalpha}};\qquad
 \left(\begin{matrix}\pd{}{x^\mu}\\\pd{}{\theta^\alpha}\\\pd{}{\btheta^\dalpha}\end{matrix}\right)
=\left(\begin{smallmatrix}\delta^\nu_\mu&0&0\\-\ii(\sigma^\nu\btheta)_\alpha&\delta^\beta_\alpha&0\\\ii(\theta\sigma^\nu)_\dalpha&0&\delta^\dbeta_\dalpha
       \end{smallmatrix}\right)
 \left(\begin{matrix}\pd{}{y^\nu}\\\pd{}{\theta^{\prime\beta}}\\\pd{}{\btheta^{\prime\dbeta}}\end{matrix}\right),\quad
 \left(\begin{matrix}\pd{}{y^\nu}\\\pd{}{\theta^{\prime\beta}}\\\pd{}{\btheta^{\prime\dbeta}}\end{matrix}\right)
=\left(\begin{smallmatrix}\delta^\mu_\nu&0&0\\\ii(\sigma^\mu\btheta)_\beta&\delta_\beta^\alpha&0\\-\ii(\theta\sigma^\mu)_\dbeta&0&\delta_\dbeta^\dalpha
       \end{smallmatrix}\right)
 \left(\begin{matrix}\pd{}{x^\mu}\\\pd{}{\theta^\alpha}\\\pd{}{\btheta^\dalpha}\end{matrix}\right),
\end{align}
and a function $f:\mathbb C^4\to \mathbb C$ (independent of $\theta'$ and $\btheta'$) is expanded as
\begin{equation}
  f(y)
= f(x-\ii\theta\sigma\btheta)
=f(x)-\ii (\theta \sigma^{\mu}\btheta) \partial_{\mu}f(x)-\frac{1}{4} \theta^4 \partial^2f(x).
\end{equation}
Note that we differentiate $[f(y)]^*$ and $f^*(y)$:
\begin{equation}
 \left[f(y)\right]^* =
f(x)+\ii (\theta \sigma^{\mu}\btheta) \partial_{\mu}f^*(x)-\frac{1}{4} \theta^4 \partial^2f^*(x)
=f^*(y+\ii\theta\sigma\btheta)=f^*(y^*).
\end{equation}

\subsection{Superfields}
\paragraph{SUSY-invariant Lagrangian}
SUSY transformation is induced by $\xi \OQ + \bar\xi\bar \OQ = \xi^\alpha\partial_\alpha + \bar\xi_\dalpha\partial^{\dalpha} + \ii(\xi\sigma^\mu\btheta+\bar\xi\bar\sigma^\mu\theta)\partial_\mu$.
Therefore, for an object $\Psi$ in the superspace,
\begin{equation}
 \bigl[\Psi\bigr]_{\theta^4}\xrightarrow{\text{SUSY}}
 \left[\Psi + \xi^\alpha\partial_\alpha\Psi + \bar\xi_\dalpha\partial^{\dalpha}\Psi + \ii(\xi\sigma^\mu\btheta+\bar\xi\bar\sigma^\mu\theta)\partial_\mu\Psi\right]_{\theta^4}
=
 \left[\Psi +
\ii(\xi\sigma^\mu\btheta+\bar\xi\bar\sigma^\mu\theta)\partial_\mu\Psi\right]_{\theta^4},
\end{equation}
which means $\bigl[\Psi\bigr]_{\theta^4}$ is SUSY-invariant up to total derivative, i.e., $\int\dd^4x \bigl[\Psi\bigr]_{\theta^4}$ is SUSY-invariant action. Also,
\begin{equation}
  \bigl[\Psi\bigr]_{\theta^2}\xrightarrow{\text{SUSY}}
 \left[\Psi + \bar\xi_\dalpha\left(\partial^{\dalpha} + \ii(\bar\sigma^{\mu}\theta)^\dalpha\partial_\mu\right)\Psi\right]_{\theta^2}
=\left[\Psi + \bar\xi_\dalpha\bar \OD^\dalpha\Psi+2\ii(\bar\sigma^{\mu}\theta)^\dalpha\partial_\mu\Psi\right]_{\theta^2}
\end{equation}
will be SUSY-invariant if $\bar \OD_\dalpha\Psi=0$, i.e., $\Psi$ is a chiral superfield. Therefore, SUSY-invariant Lagrangian is given by
\begin{equation}
 \mathcal L = \bigl[\text{(any real superfield)}\Bigr]_{\theta^4} + \bigl[\text{(any chiral superfield)}\Bigr]_{\theta^2} + \bigl[\text{(any chiral superfield)}^*\Bigr]_{\btheta^2}.
\label{eq:susy-inv}
\end{equation}
\paragraph{Chiral superfield} A chiral superfield is a superfield that satisfies $\bar \OD_\dalpha \Phi = 0$, i.e.,
we find
\begin{align}
 \Phi
 &= \phi(y) + \sqrt{2} \theta' \psi(y) + \theta'^2 F(y)\\
 &= \phi(x)+\sqrt{2} \theta \psi(x)-\ii \partial_{\mu}\phi(x) (\theta \sigma^{\mu}\btheta)+F(x) \theta^2+\frac{\ii}{\sqrt{2}} (\partial_{\mu}\psi(x)\sigma^{\mu}\btheta) \theta^2-\frac{1}{4} \partial^2\phi(x) \theta^4
\\
 \Phi^*
 &= \phi^*(x)+\sqrt{2} \bar{\psi}(x)\btheta+F^*(x) \btheta^2+\ii \partial_{\mu}\phi^*(x) (\theta \sigma^{\mu}\btheta)-\frac{\ii}{\sqrt{2}}[\theta \sigma^{\mu}\partial_{\mu}\bar{\psi}(x)] \btheta^2-\frac{1}{4} \partial^2\phi^*(x) \theta^4;
\end{align}
their product is expanded as
\begin{align}
\begin{split}
 \Phi_i^* \Phi_j
 &=
 \phi_i^*\phi_j + \sqrt{2} \phi_i^*(\theta \psi_j)+\sqrt{2} (\bar{\psi_i}\btheta)\phi_j
 + \phi_i^* F_j \theta^2+2 (\bar{\psi_i}\btheta)(\theta \psi_j)
 -\ii \left(\phi_i^*\partial_{\mu}\phi_j-\partial_{\mu}\phi_i^*\phi_j\right) (\theta \sigma^{\mu}\btheta)
 +F_i^*\phi_j \btheta^2
 \\&\quad
+\left[
\sqrt{2} \bar{\psi_i}\btheta F_j-\frac{\ii\left( \partial_{\mu}\phi_i^*\cdot\psi_j\sigma^{\mu}\btheta-\phi_i^*\partial_{\mu}\psi_j\sigma^{\mu}\btheta\right)}{\sqrt2}\right]\theta^2
+\left[\sqrt2 F_i^*\theta \psi_j+\frac{\ii\left(\theta \sigma^{\mu}\bar{\psi_i}\partial_{\mu}\phi_j-\theta \sigma^{\mu}\partial_{\mu}\bar{\psi_i}\phi_j\right)}{\sqrt2}\right] \btheta^2
\\&\quad
+\frac{1}{4} \left(4 F_i^*F_j-\phi_i^*\partial^2\phi_j-(\partial^2\phi_i^*)\phi_j+2 (\partial_{\mu}\phi_i^*)(\partial^{\mu}\phi_j)+2 \ii (\psi_j\sigma^{\mu}\partial_{\mu}\bar{\psi_i})-2 \ii (\partial_{\mu}\psi_j\sigma^{\mu}\bar{\psi_i})\right) \theta^4
\end{split}
\\
\begin{split}
 &\equiv
 \phi_i^*\phi_j + \sqrt{2} \phi_i^*(\theta \psi_j)+\sqrt{2} (\bar{\psi_i}\btheta)\phi_j
 + \phi_i^* F_j \theta^2+2 (\bar{\psi_i}\btheta)(\theta \psi_j)
 -2\ii \left(\phi_i^*\partial_{\mu}\phi_j\right) (\theta \sigma^{\mu}\btheta)
 +F_i^*\phi_j \btheta^2
 \\&\quad
+\sqrt2\left(
\bar{\psi_i}\btheta F_j+\ii\phi_i^*\partial_{\mu}\psi_j\sigma^{\mu}\btheta\right)\theta^2
+\sqrt2\left(F_i^*\theta \psi_j-\ii\theta \sigma^{\mu}\partial_{\mu}\bar{\psi_i}\phi_j\right) \btheta^2
\\&\quad
+\left(F_i^*F_j+(\partial_{\mu}\phi_i^*)(\partial^{\mu}\phi_j)+\ii\bar{\psi_i}\sigma^\mu\partial_{\mu}\psi_j\right) \theta^4
\end{split}
\end{align}
\begin{align}
 \Phi_i\Phi_j\Big|_{\theta^2}&=-\psi_i\psi_j+F_i\phi_j+\phi_iF_j
\\
\Phi_i\Phi_j\Phi_k\Big|_{\theta^2}&=
-(\psi_i\psi_j)\phi_k-(\psi_k\psi_i)\phi_j-(\psi_j\psi_k)\phi_i+\phi_i\phi_jF_k+\phi_k\phi_iF_j+\phi_j\phi_kF_i
\end{align}
\begin{equation}
 \begin{split}
\ee^{k\Phi} &=
\ee^{k\phi}\left[1+\sqrt{2} k \theta \psi+\Bigl(kF-\frac{k^2}{2}\psi\psi\Bigr) \theta^2-\ii k \partial_{\mu}\phi (\theta \sigma^{\mu}\btheta)
%\right.\\&\qquad\qquad\left.
+\frac{\ii k \left(\partial_{\mu}\psi+k\psi\partial_{\mu}\phi\right)\sigma^{\mu}\btheta \theta^2}{\sqrt{2}}-\frac{k}{4} \left(\partial^2\phi+k \partial_{\mu}\phi\partial^{\mu}\phi\right) \theta^4\right];
\end{split}
\end{equation}
note that $\Phi_i\Phi_j$, $\Phi_i\Phi_j\Phi_k$, and $\ee^{k\Phi}$ are all chiral superfields.

\paragraph{Vector superfield}
A vector superfield is a superfield $V$ that satisfies $V=V^*$.
It is given by real fields $\{C, M, N, D, A_\mu\}$ and Grassmann fields $\{\chi, \lambda\}$ as\footnote{Different coordination of ``$\ii$''s are found in literature. Take care, especially, $\lambda(\text{ours})=\ii{\col{\lambda}(\text{Wess-Bagger})}=\ii{\col{\lambda}(\text{SLHA})}$.}
\begin{equation}
\begin{split}
  V(x,\theta,\btheta)
&=
C(x)+\ii\theta \chi(x)-\ii\btheta\bar{\chi}(x)+\frac{1}{2} \left(M(x)+\ii N(x)\right) \theta^2
+\frac{1}{2} \left(M(x)-\ii N(x)\right) \btheta^2
+(\btheta\bsigma^{\mu}\theta)A_{\mu}(x)
\\&\qquad
\left(\lambda(x)+\frac12\partial_{\mu}\bar{\chi}(x)\bsigma^{\mu}\right)\theta\btheta^2
+\theta^2\btheta\left(\bar{\lambda}(x)+\frac12\bsigma^{\mu}\partial_{\mu}\chi(x)\right)
+\frac{1}{2} \left(D(x)-\frac12\partial^2C(x)\right) \theta^4.
\end{split}
\end{equation}
With this convention,
\begin{equation}
V\to V-\ii\Phi+\ii\Phi^*\Longleftrightarrow\left\{
\begin{split}
   &C\to C-\ii\phi+\ii\phi^*,&
 &\chi\to \chi-\sqrt2\psi,&
 &\lambda\to\lambda,&
\\
 &M+\ii N\to M+\ii N - 2\ii F,&
 &A_\mu\to A_\mu+\partial_\mu(\phi+\phi^*),&
 &D\to D.
\end{split}\right.
\end{equation}

The exponential of a vector superfield is also a vector superfield:
\begin{equation}
\begin{split}
 \ee^{kV}&=\ee^{kC}\Bigg\{
 1+\ii k (\theta \chi-\btheta\bar{\chi})
 +\left(\frac{M+\ii N}{2}k+\frac{\chi\chi}{4}k^2\right) \theta^2
 +\left(\frac{M-\ii N}{2}k+\frac{\bchi\bchi}{4}k^2\right) \btheta^2
 +(k^2\theta \chi\btheta\bar{\chi}-k\theta \sigma^{\mu}\btheta A_{\mu})
 \\&\qquad
 +\left[k\btheta \blambda-\ii k \btheta \bchi\left(\frac{M+\ii N}{2}k+\frac{\chi\chi}{4}k^2\right)
 +\frac12 k\btheta\bsigma^{\mu}\left(\partial_{\mu}\chi -\ii k \chi A_\mu\right)
 \right] \theta^2
 \\&\qquad
 +\left[k\theta \lambda+\ii k \theta \chi\left(\frac{M-\ii N}{2}k+\frac{\bar{\chi}\bar{\chi}}{4}k^2\right)
 -\frac12 k\theta \sigma^{\mu}\left(\partial_{\mu}\bar{\chi}+\ii k \bchi A_\mu\right)
 \right] \btheta^2
 \\&\qquad
 +\left[
 \frac{k}{2} \left(D-\frac12\partial^2 C\right)
 -\frac{1}{2} \ii k^2 (\lambda\chi-\blambda\bchi)
 + \left(\frac{M+\ii N}{2}k+\frac{\chi\chi}{4}k^2\right)\left(\frac{M-\ii N}{2}k+\frac{\bchi\bchi}{4}k^2\right)
 \right.\\&\qquad\qquad\left.
 +\frac{k^3}{4} \bchi\bsigma^{\mu}\chi A_{\mu}
 +\frac{k^2}{4}\left(\ii \bchi\bsigma^{\mu}\partial_{\mu}\chi-\ii \partial_{\mu}\bchi\bsigma^{\mu}\chi+A^{\mu}A_{\mu}\right)
 \right]\theta^4\Bigg\}.
\end{split}
\end{equation}
\paragraph{Supergauge symmetry}
The gauge transformation $\phi(x)\to\ee^{\ii g \theta^a(x)t^a}\phi(x)$ is not closed in the chiral superfield; i.e., $\ee^{\ii g \theta^a(x)t^a}\Phi(x)$ is not a chiral superfield if the parameter $\theta(x)$ has $x^\mu$-dependence.
Hence, in supersymmetric theories, it is extended to \emph{supergauge symmetry} parameterized by a chiral superfield $\Omega(x)$, which is given by
\begin{alignat}{3}
 \Phi   &\to\ee^{2\ii g \Omega^a(x) t^a}\Phi,
&\qquad
 \Phi^* &\to\Phi^*\ee^{-2\ii g \Omega^{*a}(x) t^a}
\end{alignat}
for a chiral superfield $\Phi$ and an anti-chiral superfield $\Phi^*$.
The supergauge-invariant Lagrangian should be
\begin{equation}
 \mathcal L \sim \Phi^*\cdot\text{(real superfield)}\cdot\Phi;
\end{equation}
we parameterize the ``real superfield'' as $\ee^{2g V^a(x)t^a}$:
\begin{equation}
 \mathcal L = \Bigl[\Phi^*\ee^{2g V^a(x)t^a}\Phi\Bigr]_{\theta^4};
\qquad
 \ee^{2g V^a(x)t^a}\to
\ee^{2\ii g \Omega^{*a}(x) t^a}
 \ee^{2g V^a(x)t^a}
\ee^{-2\ii g \Omega^a(x) t^a}.
\end{equation}

In Abelian case, $t^a$ is replaced by the charge $Q$ of $\Phi$ and
\begin{align}
  \mathcal L = \Bigl[\Phi^*\ee^{2g Q V(x)}\Phi\Bigr]_{\theta^4};
\qquad
  &\Phi\to \ee^{2\ii g Q \Omega(x)}\Phi,
\quad
  \Phi^*\to \Phi^*\ee^{-2\ii g Q \Omega^*(x)},
\\
&\ee^{2gQ V(x)}\to
\ee^{2\ii gQ \Omega^*(x)}
 \ee^{2gQ V(x)}
\ee^{-2\ii gQ \Omega(x)}
=\ee^{2gQ\left(V-\ii\Omega+\ii\Omega^*\right)}.
\end{align}
The usual gauge transformation corresponds to the real part of the lowest component of $\Omega$, i.e., $\theta\equiv2\Re\phi=\phi+\phi^*$,
and we use the other components to fix the supergauge so that $C$, $M$, $N$ and $\chi$ are eliminated:
\begin{align}
 \text{supergauge fixing:}\quad &V(x)\longrightarrow(\btheta\bsigma^{\mu}\theta)A_{\mu}(x)+\btheta^2\theta\lambda(x)+\theta^2\btheta\bar{\lambda}(x)+\frac{1}{2}D(x)\qquad\text{(Wess-Zumino gauge)};\\
& \ee^{2gQV}\longrightarrow
 1 +gQ\left(
  - 2\theta \sigma^{\mu}\btheta A_{\mu} +2\theta^2\btheta \blambda+2\btheta^2\theta\lambda
  + D\theta^4
\right)
 +g^2Q^2A^\mu A_\mu\theta^4.
\end{align}
The gauge transformation is the remnant freedom:
 $\Theta= \phi(y)=\phi-\ii \partial_{\mu}\phi (\theta \sigma^{\mu}\btheta)-\partial^2\phi \theta^4/4$ with $\phi$ being real;
\begin{equation}
 \Phi_i\to \ee^{2\ii gQ \Theta}\Phi_i,
\qquad
 \ee^{2gQ V}\to \ee^{2gQ (V-\ii\Theta+\ii\Theta^*)}.
\end{equation}
Rules for each component is obvious in $(y,\theta,\bar\theta)$-basis and given by
\begin{equation}
  \{\phi,\psi,F\}\to \ee^{\ii g Q \theta}\{\phi,\psi,F\},\qquad
  A_\mu\to A_\mu+\partial_\mu \theta,\qquad
  \lambda\to\lambda,\qquad
  D\to D.
\end{equation}

For non-Abelian gauges, the supergauge transformation for the real field is evaluated as
\begin{align}
  \ee^{2g V}
&\to
\ee^{2\ii g \Omega^{*}}
 \ee^{2g V}
\ee^{-2\ii g \Omega}
real super\\&=
\left(\ee^{2\ii g\Omega^*}\ee^{2g V}\ee^{-2\ii g\Omega^*}\right)
\left(\ee^{2\ii g\Omega^*}\ee^{-2\ii g\Omega}\right)
\\&=\exp\left(\ee^{[2\ii g\Omega^*,}2gV\right)\ee^{2\ii g(\Omega^*-\Omega)}+\Order(\Omega^2)
\\&=\exp\left(2gV+[2\ii g\Omega^*,2gV]\right)\ee^{2\ii g(\Omega^*-\Omega)}+\Order(\Omega^2);
\\&=\exp\left[
2gV+[2\ii g\Omega^*,2gV] + \int_0^1\dd t\,g(\ee^{[2gV,})2\ii g(\Omega^*-\Omega)]+\Order(\Omega^2)
\right]
\\&=\exp\left[
2gV+[2\ii g\Omega^*,2gV] + \sum_{n=0}^{\infty}\frac{B_n\left([2gV,\right)^n}{n!}2\ii g(\Omega^*-\Omega)]
\right]+\Order(\Omega^2)
\\&=\exp\left[
2g\left(
V + \ii (\Omega^*-\Omega) - [V,\ii g(\Omega^*+\Omega)]+
 \sum_{n=2}^{\infty}\frac{\ii B_n\left([2gV,\right)^n}{n!} (\Omega^*-\Omega)]
\right)+\Order(\Omega^2)
\right].
\end{align}
Here, again we can use the ``non-gauge'' component of $\Omega$ to eliminate the $C$-term etc., i.e., we fix $\ii(\Omega^*-\Omega)$, the second term of the expansion, to remove those terms:
\begin{equation}
 V - [V,\ii g(\Omega^*+\Omega)]+
 \left(\ii+
\sum_{n=2}^{\infty}\frac{\ii B_n\left([2gV,\right)^n}{n!}\right) (\Omega^*-\Omega)]+\Order(\Omega^2)
=(\btheta\bsigma^{\mu}\theta)A_{\mu}+\btheta^2\theta\lambda+\theta^2\btheta\bar{\lambda}+\frac{1}{2}D;
\end{equation}
this defines the Wess-Zumino gauge:
\begin{align}
 \text{supergauge fixing:}\quad &V^a(x)\longrightarrow(\btheta\bsigma^{\mu}\theta)A^a_{\mu}(x)+\btheta^2\theta\lambda^a(x)+\theta^2\btheta\bar{\lambda}^a(x)+\frac{1}{2}D^a(x),
\\
& \ee^{2gV^at^a}\longrightarrow
 1 +g\left(
  - 2\theta \sigma^{\mu}\btheta A^a_{\mu} +2\theta^2\btheta \blambda^a+2\btheta^2\theta\lambda^a
  + D^a\theta^4
\right)t^a
 +g^2Q^2A^{a\mu} A^b_\mu\theta^4t^at^b.
\end{align}
The gauge transformation is given by
\begin{equation}
  \Phi\to \ee^{2\ii g \Theta^at^a}\Phi,
\quad
  \ee^{2g V^at^a}\to\ee^{2\ii g \Theta^bt^b} \ee^{2g V^at^a}\ee^{-2\ii g \Theta^ct^c}.
\end{equation}
For components in chiral superfields,
\begin{equation}
   \{\phi,\psi,F\}\to \ee^{\ii g \theta^at^a}\{\phi,\psi,F\},
\end{equation}
while for vector superfield we can express as infinitesimal transformation:
\begin{align}
 V\to V'
&\simeq
V + \ii (\Theta^*-\Theta)
 - [V,\ii g(\Theta^*+\Theta)]+
 \sum_{n=2}^{\infty}\frac{\ii B_n\left([2gV,\right)^n}{n!} (\Theta^*-\Theta)]
\\&=
V +2(\btheta\bsigma^{\mu}\theta)\partial_{\mu}\phi
 - \left[V,\ii g\left(2\phi-\frac{\theta^4}2\partial^2\phi\right)\right]
 +2\sum_{n=2}^{\infty}\frac{B_n\left([2gV,\right)^n}{n!}(\btheta\bsigma^{\mu}\theta)\partial_{\mu}\phi]
\\&=
V +2(\btheta\bsigma^{\mu}\theta)\partial_{\mu}\phi
 +2 g f^{abc}V^b\phi^ct^a
\qquad\text{(Wess-Zumino gauge)}
\end{align}
\begin{equation}
\begin{split}
   \therefore A_\mu^a&\to A^a_\mu+\partial_\mu \theta^a
   + g f^{abc}A_\mu^b\theta^c  + \Order(\theta^2),
 &\lambda^a&\to\lambda^a+gf^{abc}\lambda^b\theta^c+\Order(\theta^2),
 \\
 D^a&\to D^a+gf^{abc}D^b\theta^c+\Order(\theta^2),
& \blambda^a&\to\blambda^a+gf^{abc}\blambda^b\theta^c + \Order(\theta^2).
\end{split}
\end{equation}
\paragraph{Gauge-field strength}
The real superfield $\ee^{V}$ is gauge-invariant in Abelian case and a candidate in Lagrangian term, but this is not case in non-Abelian case.
We thus define a chiral superfield from $\ee^{V}$:
\begin{equation}
 \mathcal W_\alpha = \frac14\bar\OD_\dalpha\bar\OD^\dalpha\left(\ee^{-2gV}\OD_\alpha\ee^{2gV}\right);
\qquad
\mathcal W_\alpha\xrightarrow{\text{gauge}}
\ee^{2\ii g \Omega}\mathcal W_{\alpha}\ee^{-2\ii g \Omega};
\end{equation}
it is not supergauge- or Lorentz-invariatn, but $\Tr(\mathcal W^\alpha \mathcal W_\alpha) = \Tr(\epsilon^{\alpha\beta}\mathcal W_\beta \mathcal W_\alpha)$ is supergauge- and Lorentz-invariant, and its $\theta^2$-term is SUSY-invariant, which becomes a candidate in SUSY Lagrangian with its Hermitian conjugate.

In Wess-Zumino gauge, it is given by
\begin{align}
\mathcal W_\alpha &=\left\{
\lambda_{\alpha}^{a}(y)+\theta_\alpha D^{a}(y)+\frac{[\ii(\sigma^{\mu}\bsigma^{\nu}-\sigma^{\nu}\bsigma^{\mu})\theta]_\alpha}{4} F^a_{\mu\nu}(y)
+\theta^2 \left[\ii\sigma^{\mu}\DD_\mu\bar{\lambda}^a(y^*)\right]_\alpha
\right\}t^a
\\\begin{split}&=\left[
\lambda_{\alpha}^{a}+\theta_\alpha D^{a}+\frac{\ii}{2} (\sigma^{\mu}\bsigma^{\nu}\theta)_\alpha F^a_{\mu\nu}
+\ii \theta^2(\sigma^{\mu}\DD_{\mu}\bar{\lambda}^{a})_\alpha
+\ii(\btheta\bsigma^{\mu}\theta)\partial_{\mu}\lambda _{\alpha}^{a}
-\frac{\theta^4}{4} \partial^2\lambda _{\alpha}^{a}
\right.\\&\qquad\qquad\qquad\left.
+\frac{\ii\theta^2(\sigma^{\mu}\btheta)_{\alpha}}{2} \left(\partial_{\mu}D^{a}
+\ii\partial^\nu F^a_{\mu\nu}
- gf^{abc}\epsilon_{\mu\nu \rho\sigma}  A^{\nu b}\partial^{\rho}A^{\sigma c}
\right)\right]T^{a},
\end{split}
\end{align}
where, as usual,
\begin{alignat}{2}
F^a_{\mu\nu}&=\partial_{\mu}A_\nu^a-\partial_{\nu}A_\mu^a+g A_\mu^b A_\nu^c f^{abc},
&\qquad
\DD_\mu\lambda^a_\alpha
&=\partial_{\mu}\lambda^{a}_\alpha+g f^{abc}A_\mu^{b}\lambda^{c}_\alpha.
\end{alignat}
Also,
\begin{align}
 \left[\Tr(\mathcal W^\alpha\mathcal W_\alpha)\right]_{\theta^2}
&=
\left[
\ii\lambda^a\sigma^{\mu}\DD_{\mu}\bar{\lambda}^{b}
+\ii\lambda^b\sigma^{\mu}\DD_{\mu}\bar{\lambda}^{a}
+D^a D^{b}
-\frac{1}{4}\left(\ii \epsilon^{\sigma\mu \nu\rho}+2\eta^{\mu\rho}\eta^{\nu\sigma}\right)
F^a_{\mu\nu}F^b_{\rho\sigma}
\right]
\Tr(t^at^b)
\\
&=
\ii\lambda^a\sigma^{\mu}\DD_{\mu}\bar{\lambda}^a
+\frac12D^a D^a
-\frac14F^a_{\mu\nu}F^{a\mu\nu}
+\frac{\ii}{8}\epsilon^{\mu \nu\rho\sigma}
F^a_{\mu\nu}F^a_{\rho\sigma},
\\
 \left[\Tr(\mathcal W^\alpha\mathcal W_\alpha)\right]_{\theta^4}
&=
\frac{\theta^4}{4} \left(2 (\partial^{\mu}\lambda^{a})(\partial_{\mu}\lambda^{b})-\lambda^{a}\partial^2\lambda^{b}-(\partial^2\lambda^{a})\lambda^b\right)\Tr(t^at^b)
=
\frac{\theta^4}{4} \left((\partial^{\mu}\lambda^{a})(\partial_{\mu}\lambda^{a})-\lambda^{a}\partial^2\lambda^{a}\right).
\end{align}

For Abelian theory,
\begin{align}
& \mathcal W_\alpha = \frac14\bar\OD_\dalpha\bar\OD^\dalpha\left(\ee^{-2gV}\OD_\alpha\ee^{2gV}\right)
=\frac14\bar\OD_\dalpha\bar\OD^\dalpha\OD_\alpha(2gV),
\\
& \frac12\left[\mathcal W^\alpha\mathcal W_\alpha\right]_{\theta^2}
=\ii\lambda\sigma^{\mu}\DD_{\mu}\bar{\lambda}
+\frac12DD
-\frac14F_{\mu\nu}F^{\mu\nu}
+\frac{\ii}{8}\epsilon^{\mu \nu\rho\sigma}
F_{\mu\nu}F_{\rho\sigma}.
\end{align}

\subsection{Lagrangian blocks}
\paragraph{Lagrangian construction}
The supergauge transformation is summarized as
\begin{equation}
 \Phi_i   \to U_{ij}\Phi_j,
\qquad
 \tilde\Phi_j \to \tilde\Phi_i [U^{-1}]_{ij},
\qquad
\mathcal W_\alpha\to U'\mathcal W_\alpha U^{\prime-1},
\end{equation}
where
\begin{equation}
 \tilde\Phi^*_j :=  \Phi^*_i [\ee^{2gV t^a_\Phi}]_{ij},
\qquad
 U:=\exp(2\ii g \Omega^at^{a}_\Phi),
\qquad
 U':=\exp(2\ii g \Omega^at^{a}_{\mathcal W}),
\end{equation}
$t^a_\Phi$ is the representation matrix or U(1) charge for the field $\Phi$,
and $t^a_{\mathcal W}$ is the representation matrix that is used to define $\mathcal W_\alpha$.
To construct a Lagrangian, we should composite these ingredients in real and invariant under SUSY, supergauge, and Lorentz transformation.
A sufficient condition for SUSY invariance is given by \eqref{eq:susy-inv}, so
\begin{equation}
 \mathcal L =
\left[K(\Phi_i, \tilde\Phi^*_j)\right]_{\theta^4}
+
\left\{
\left[f_{ab}(\Phi_i)\mathcal W^a\mathcal W^b\right]_{\theta^2} + \text{H.c.}
\right\}
+
\left\{
\left[W(\Phi_i)\right]_{\theta^2} + \text{H.c.}
\right\} + D
\end{equation}
is one possible construction.
The last term $D$ (Fayet-Illiopoulos term) comes from $V$ of an U(1) gauge boson; note that its supergauge invariance is due to the intentional definition of $V$.

One can construct more general Lagrangian; for example, one can introduce a vector superfield that is not associated to a gauge symmetry, but then the supergauge fixing is not available and one has to include $C$ or $M$ fields.

\paragraph{Renormalizable Lagrangian}
Since $[\Phi]_{\theta^4}$ is a total derivative, renormalizable Lagrangian is limited to
\begin{equation}
  \mathcal L =
\left[\Phi^*_i [\ee^{2gV t^a_\Phi}]_{ij}\Phi_j\right]_{\theta^4}
+
\left\{
\left[\mathcal W^a\mathcal W^a\right]_{\theta^2} 
+\left[W(\Phi_i)\right]_{\theta^2} + \text{H.c.}
\right\}
+
D
\end{equation}
up to numeric coefficients.
\end{document}
