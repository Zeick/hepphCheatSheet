%#!platexmake CheatSheet
%%% Time-Stamp: <>
%%% 一部で日本語が使用されています。

\newcommand{\Hom}{\mathop{\mathrm{Hom}}}
\section{Mathematics}
\subsection{Algebra}
\subsubsection{Algebraic Structure}
\begin{tabular}{l@{ :\ \ \ }l}
Semigroup     & For $a,b\in A\stx{set}$, $ab\in A$; Associative.\\
Monoid        & For $a,b\in A\stx{set}$, $ab\in A$; Associative, Unit.\\
Group         & For $a,b\in A\stx{set}$, $ab\in A$; Associative, Unit, Inverse.\\
Module {\tiny (加群/可換群)}
              & For $a,b\in A\stx{set}$, $a+b\in A$; Commutative, Associative, Unit, Inverse.\\
Semimodule    & For $a,b\in A\stx{set}$, $a+b\in A$; Commutative, Associative, Unit.\\
Ring   {\tiny (環)}
              & $+$: Module, $\times$: Semigroup{\footnotesize (Monoid)}, Distributive.\\
Semiring      & $+$: Semimodule, $\times$: Monoid, $0\neq1$, Distributive, $0\times a = a\times 0 = 0$.\\
Field         & $+$: Module, $\times$: Commutative Monoid, $a^{-1}$ but $0$, $1\neq0$, Distributive.
\end{tabular}

\vspace{.5zw}

\Paragraph{Vector Space}
\begin{tabular}{l@{ :\ \ \ }l}
Vector space on $K$  & For $v\in(V,+)\stx{module}$ and $k\in K\stx{field}$,\\
\hfill ($K$-module)  & \qquad $k v\in (V,+)$; Compatible, Distributive, $1v=v$.\\
Norm &
    $\norm x\ge0$, $\norm x=0\Leftrightarrow x=0$, $\norm{kx}=k\norm x$,
    $\norm{x+y}\le \norm x+\norm y$\\
Inner product & $\braket xx\ge0$, $\braket xx=0\Leftrightarrow x=0$,$\braket{x}{y}=\braket{y}{x}$,\\
              & \qquad $\braket{x+y}z=\braket xz + \braket yz$, $\braket{kx}y =k\braket xy$

\end{tabular}

\vspace{.5zw}

\Paragraph{$K$-algebra}
$K$-algebra $C(V)$とは,vector空間$V$に,distributiveな乗法を入れたもの:\\
\qquad$xy\in C(V);$\quad $(xy)z=x(yz)$, $(x+y)z=xz+yz$, $x(y+z)=xy+xz$, $k(xy)=(kx)y=x(ky)$.

\subsubsection{Lie Algebra}
\begin{tabular}{l@{ :\ \ \ }l}
Lie Algebra & For a Finite-dimensional $K$-module $(A,+)$ and $x,y,z\in(A,+)$, $a,b\in K$,\\
            & \quad $[u,v]\in (A,+)$ (Lie product), and\\
            & \quad Bilinear: $[ax+by, z] = a[x,z]+b[y,z]$, $[x, ay+bz] = a[x,y]+b[x,z]$,\\
            & \quad Alternating: $[x,x]=0$ \quad ($\then [x,y]=-[y,x]$),\\
            & \quad Jacobi id.: $[x,[y,z]]+[y,[z,x]]+[z,[x,y]]=0$.\\
\end{tabular}\par
則ち$[A,B]:=AB-BA$として閉じていればLie algebraとなる。
\subsubsection{Clifford Algebra}
Here $V$ is a vector space on $K$ with inner product which need not be positive definite.

For $C(V)$, {\bf Clifford algebra} $(C(V),\theta)$ is defined as
\begin{itemize}
 \item $C(V)$: $K$-algebra with $1$ \quad ($1x=x1=x$),
 \item $\theta: V\to C(V)$, homomorphism, $\theta(x)^2 = \braket xx1$,
 \item Any $C'$: $K$-algebra with 1 and any homomorphism $\phi: V\to C'$ with $\phi(x)^2=\braket xx1$, there's unique $\bar\phi: C\to C'$, homomorphism, $\bar\phi(1)=1$.
\end{itemize}
\begin{rightnote}
The Gamma matrices form Clifford Algebra:
\[
  V: \Real^4 \text{ with } \{G_0, G_1, G_2, G_3\},\quad
  \braket{G_0}{G_0}=1, \braket{G_i}{G_i}=-1; \qquad
  C(V): M(n,\Complex).
\]\vspace{-2zw}
\end{rightnote}

\subsubsection{Multilinear Algebra}
可換環$K$上のvector空間$V$とその双対空間$V^*$について:\vspace{.6zw}\par
\begin{tabular}{l@{ :\ \ \ }l}
{\bf Tensor Algebra} $T(V)$
    & \begin{minipage}[t]{280pt}
       線形写像$f: V\to T(V)$ を持つ$K$-algebra であり,別の$K$-algebra $A$への線型写像 $g: V\to A$が与えられたときに可換な準同型$h:T(V)\to A$ s.t. $h\circ f=g$が一意に存在するもの。
      \end{minipage}\vspace{.5zw}\\
{\bf Symmetric Algebra} $S(V)$
    & 上の定義で,$C(V)$ (および $A$)を可換$K$-algebraとしたもの。\\
{\bf Exterior Algebra} $\wedge(V)$
    & 上の定義で,$g(\cdots vv\cdots)=0$を要請したもの。
\end{tabular}

\vspace{1zw}

\Paragraph{Tensor and Tensor Space}
\begin{tabular}{l@{ :\ \ \ }l}
$p$次反変Tensor積 & $T^p(V):=V^{\otimes p} = V\otimes V\otimes \cdots \otimes V$\\
$q$次共変Tensor積 & $T_q(V):=(V^*)^{\otimes q} = V^*\otimes V^*\otimes \cdots \otimes V^*$\\
混合Tensor積      & $T^p_q(V):=T^p(V)\otimes T_q(V)$ (ただし$V$と$V^*$の順序を変えたものは同型)\\
{\bf Tensor Space}& $T(V):=\oplus T^p_q(V)$
\end{tabular}\vspace{.5zw}

Tensor spaceの代数$(+,\otimes)$は環を為しており,また更に:\vspace{.2zw}\par
\begin{tabular}{l@{ :\ \ \ }l}
Contraction    & $V^*$は$V\to K$なので,$T^p_q(V)\to T^{p-1}_{q-1}(V)$が定義される。\\
内積           & $T_2(V)\ni g_{ij}: V\times V\to \Real$ (通常は対称にする。正定値としてもよい。)\\
添字の上げ下げ & $T_2(V)\cong\Hom(V,V^*)$(同型)なので,内積から誘導される。\\
               & 同型なので,$g^{ij}: \Hom(V^*,V)$は逆写像になる。
\end{tabular}


\Paragraph{Grassmann Operator}
台集合$V$をHilbert空間であるとする。$V\ni v$について
\begin{equation}
 \|\psi v\|^2=(\psi_{ab}v_b)^*(\psi_{ac}v_c) = v_b^*v_c(\psi_{ab})^*\psi_{ac}\ge 0
\qquad\therefore
(\psi_{ab})^*\psi_{ac}
 = -\left(\psi_{ab}(\psi_{ac})^*\right)^*
\end{equation}
即ち反可換な作用素について
$(ab)^\dagger = a^\dagger b^\dagger$,
$\trans{(ab)} = -\trans a \trans b$,
$(ab)^* = b^*a^*$ である。

\TODO{$\psi$の正体がわからない……。。。}


\subsubsection{Lie Group and Lie Algebra}
\begin{itemize}
 \item 群$G$が{\bf Lie group}である … $G$が同時に$C^\infty$多様
       体であり,積演算と逆元写像が共に$C^\infty$級である。
 \item Lie群$G$が{\bf COMPLEX Lie group}である … 積演算と逆元写像が共に正則写像である。
\end{itemize}
\begin{itemize}
 \item Lie群$G$の単位元における接空間を,{\bf $G$のLie algebra}
       $\mathfrak g$という。
\begin{itemize}
 \item $\mathfrak g$は$G$の左不変なvector場全体である。
 \item $\mathfrak g$はvector場の括弧積の下でLie algebraとなる。
\end{itemize}
 \item $G$として有限次元Lie群を考えると,
\begin{itemize}
 \item そのLie代数の基底$B_i$に対して{\bf structure
       constant} $c$が$[B_i,B_j]=c^{k}_{ij}B_k$として定義できる。
\end{itemize}
\end{itemize}
\starline
\begin{itemize}
 \item Compact Lie群は線型Lie群である。
 \item $G$として{\bf Linear group} $\gGL(n;\Real)$を考えると,
 \begin{itemize}
  \item そのLie代数は$n$次実正方行列全体となる。
  \item Vector場の括弧積は{\bf commutation relation} $[X,Y]=XY-YX$となる。
 \end{itemize}
 \item Lie群は,$\gGL(n;\Complex)$の部分Lie群と局所同型になるような位相群
       でかつ連結成分が高々可算個であるものである。
\end{itemize}

以下では,Lie群として$\gGL(n;\Real)$の部分群を考えることにし,Lie代数の
元を行列により表現する。

\subsubsection{Matrix Representation}
\begin{itemize}
 \item Lie群$G$のLie代数の基底の組を,$G$の{\bf generators}と言う。
 \item $\gGL(n;\Real)$の元は$n$次元行列で表せる。
 \item Lie群$G$の生成子$\{T_i\}$に対し,
       以下の2つは共に$G$の単位元近傍の局所座標系を与える。
\begin{align}
 (x_1,\cdots,x_m)&\mapsto\ee^{x_1T_1+\cdots+x_mT_m}&
(x_1,\cdots,x_m)&\mapsto\ee^{x_1T_1}\cdots\ee^{x_mT_m}
\end{align}
\end{itemize}

\begin{itemize}
 \item Lie群$G$がcompactである …
\begin{enumerate}
 \item 多様体$G$がcompactである。       \TODO{これは何故同値なのか?}
 \item $G$の生成子$\{T_i\}$を,
       $\Tr(T_iT_j)=k\,\delta_{ij}$かつ$k>0$となるように取り替える
       ことができる。\\【この基底の下では構造定数が完全反対称になる。】
\end{enumerate}
 \item Compact群$G$は,{\bf unitary representation}を持つ。\\
       故に,単位元の近傍では有限個の{\bf Hermitian matrix} $T^i$と
       parameters $x^i\in\Real$により,$G$の元を
       \begin{equation}
        \ee^{\ii x^iT^i}
       \end{equation}
       と表すことが出来る。
\end{itemize}
\subsubsection{結論}
Compact Lie群の元のうち,単位元近傍にあるもの$V$は,
\subparagraph{Hermitian Representation}
\begin{align*}
 V=\exp(\ii x^iT^i)
 \where
& T^i:\text{Hermitian Matrix},\quad x^i\in\Real,\\
&[T^i,T^j]=\ii f^{ijk}T^k,\quad \Tr(T^iT^j)=\lambda\,\delta^{ij}>0;\qquad f\in\Real
\end{align*}
\subparagraph{Real Representation}
\begin{align*}
 V=\exp(x^iR^i)
 \where
& R^i:\text{Real Matrix},\quad x^i\in\Real,\\
&[R^i,R^j]=-f^{ijk}R^k,\quad \Tr(R^iR^j)=-\lambda\,\delta^{ij}<0;\qquad f\in\Real
\end{align*}
と表すことが出来る。


%%% Local Variables:
%%% TeX-master: "CheatSheet.tex"
%%% End:
