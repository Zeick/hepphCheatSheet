\documentclass[CheatSheet]{subfiles}
\begin{document}



\summarystyle
\section{Minimal Supersymmetric Standard Model}

Gauge symmetry: $\gSU(3)\w{color}\times\gSU(2)\w{weak}\times\gU(1)_{Y}$

\vspace{1em}

\noindent
 Particle content:
\begin{table}[h!]
\newcommand\h[1]{\multicolumn{1}{@{\,\,}c@{\,\,}}{#1}}
\begin{subtable}[t]{0.635\textwidth}\centering
\caption{Chiral superfields}
\catcode`#=\active \def#{\phantom{-}}
\begin{tabular}[t]{c|ccc|cc|l@{,\,}l@{~}l}\toprule
 \h{}      & \h{$\gSU(3)$} & \h{$\gSU(2)$} & \h{$\gU(1)$}
                                              & \h{$B$} & \h{$L$} & \multicolumn{3}{c}{~~~~scalar/spinor~~~~}\\\midrule
 $Q_i$     &$\THREE$   & $\TWO$    & $#1/6$   & $#1/3$  &         & $\tqL$     & $\qL$ & [$\to (\uL,\dL)$] \\
 $L_i$     &           & $\TWO$    & $-1/2$   &         & $#1$    & $\tlL$     & $\lL$ & [$\to (\nuL, \lL)$]\\
 $U_i^\cc$ &$\THREEbar$&           & $-2/3$   & $-1/3$  &         & $\tuR^\cc$ & $\uR^\cc$ \\
 $D_i^\cc$ &$\THREEbar$&           & $#1/3$   & $-1/3$  &         & $\tdR^\cc$ & $\dR^\cc$ \\
 $E_i^\cc$ &           &           & $ #1 $   &         & $-1$    & $\teR^\cc$ & $\eR^\cc$ \\
 $\Hu$     &           & $\TWO$    & $#1/2$   &         &         & $\hu$      & $\thu$ & [$\to (\hup, \huz)$] \\
 $\Hd$     &           & $\TWO$    & $-1/2$   &         &         & $\hd$      & $\thd$ & [$\to (\hdz, \hdm)$]\\\bottomrule
\end{tabular}
\end{subtable}
\begin{subtable}[t]{0.36\textwidth}
\caption{Vector superfields}\centering
\begin{tabular}[t]{c|ccc|c}\toprule
 \h{}      & \h{$\gSU(3)$} & \h{$\gSU(2)$} & \h{$\gU(1)$} & \h{ino/boson} \\\midrule
 $g$ & adj.      &           &          & $\tig, g_\mu$ \\
 $W$ &           &  adj.     &          & $\tiw, W_\mu$ \\
 $B$ &           &           &          & $\tib, B_\mu$ \\\bottomrule
\end{tabular}
\end{subtable}
\end{table}

 \noindent
Here, each of the column groups shows (from left to right) superfield name, charges for the gauge symmetries, other quantum numbers if relevant, and notation for corresponding fields (and $\gSU(2)$ decomposition).


\paragraph{``c''-notation}
For scalars, $\tilde\phi\w R^\cc:=\phi\w R^*=\opC \phi\w R\opC$ (because the intrinsic phase for $\mathsf C$ is $+1$ for quarks and leptons.)

For matter spinors, $\psi\w R^\cc:=\bar\psi\w R$ (and $\psi\w R=\bar\psi\w R^\cc$); Dirac spinors are thus
\begin{equation}
\displaystyle \psi\w L =\pmat{\psi\w L\\0},\quad
\displaystyle \overline{\psi\w L} = \pmat{0&\bar\psi\w L},\quad
 \psi\w R^\cc :=\pmat{\psi\w R^\cc \\ 0}=C\pmat{0\\\psi\w R} = C\psi\w R,
\quad
 \overline{\psi\w R^\cc} = \pmat{0 & \psi\w R} = \pmat{\bar\psi\w R & 0}C  = \overline{\psi\w R} C.\notag
\end{equation}

\paragraph{Superpotential and \cancel{SUSY}-terms}
\begin{align}
 W\w{RPC} &= \mu\Hu\Hd
           - \yu[ij] U^\cc_i\Hu Q_j
           + \yd[ij] D^\cc_i\Hd Q_j
           + \ye[ij] E^\cc_i\Hd L_j,\\
 W\w{RPV} &= -\kappa_i L_i\Hu
           + \frac12 \lambda  _{ijk} L_i L_j E^\cc_k
           +         \lambda' _{ijk} L_i Q_j D^\cc_k
           + \frac12 \lambda''_{ijk} U^\cc_i D^\cc_j D^\cc_k,
\\
\mathcal L_{\text{\cancel{SUSY}}}&=
- \frac12\left(M_3\tig\tig+M_2\tiw\tiw+M_1\tib\tib+\text{H.c.}\right)
-V_{\text{\cancel{SUSY}}};\\
V^{\mathrm{RPC}}_{\text{\cancel{SUSY}}}&=
\left(\tqL^*m_Q^2\tqL+\tlL^*m_L^2\tlL+\tuR^* m_{U^\cc}^2\tuR + \tdR^* m_{D^\cc}^2\tdR + \teR^* m_{E^\cc}^2\teR
       +m^2_{\Hu}|\hu|^2 + m^2_{\Hd}|\hd|^2\right)
\notag\\&\quad
       +\left(-\tuR^*\hu {a\w u}\tqL + \tdR^* \hd {a\w d} \tqL+ \teR^* \hd {a\w e} \tlL + b\Hu\Hd+\text{H.c.}\right)
\notag\\&\quad
       +\left(+\tuR^*\hd^*{c\w u}\tqL+\tdR^* \hu^*{c\w d}\tqL+\teR^*\hu^*{c\w e}\tlL+\text{H.c.}\right),
\\
V^{\mathrm{RPV}}_{\text{\cancel{SUSY}}}&=
    \left(-b_i\tlL[i]\Hu +\frac12T_{ijk}\tlL[i]\tlL[j]\teR[k]^*+T'_{ijk}\tlL[i]\tqL[j]\tdR[k]^*
             +\frac12T''_{ijk}\tuR[i]^*\tdR[j]^*\tdR[k]^*
    +\tlL[i]^*M^2_{Li}\Hd+\text{H.c.}\right)
\notag\\&\qquad
   + \left(C^1_{ijk}\tlL[i]^*\tqL[j]\tuR[k]^* + C^2_i\hu^*\hd\teR[i]^* + C^3_{ijk}\tdR[i]\tuR[j]^*\teR[k]^*+\frac12C^4_{ijk}\tdR[i]\tqL[j]\tqL[k]+\text{H.c.}\right),
\end{align}
($\lambda_{ijk}=-\lambda_{jik}$, $\lambda''_{ijk}=-\lambda''_{ikj}$, and
$C^4_{ijk}=C^4_{ikj}$.)



\clearpage

\detailstyle



\subsection{Notation}
Our notation in this section (and the previous section)  follows DHM~\cite[PhysRept]{0812.1594} and Martin~\cite[v7]{hep-ph9709356} (but note that Martin uses $(-,+,+,+)$-metric) for RPC part and SLHA2 convention for RPV part.
In particular, the sign of gauge bosons are fixed by $\DD_\mu\phi = \partial_\mu\phi - \ii g A_\mu^a t^a_{ij}\phi_j$, and the phase of gauginos are by $\mathcal L\ni \sqrt2 g (\phi^*t^a\psi\lambda^a)$.
Phases of $\phi$ and $\psi$ in chiral superfields are not yet specified; they are later used to remove $F\tilde F$ terms and diagonalize Yukawa matrices.

\subsection{Lagrangian construction}
The most generic form of the Lagrangian is given by
\begin{align}
\mathcal L &= \mathcal L\w{matter}+\mathcal L\w{gauge} + \mathcal L\w{super} + \mathcal L\w{FI} + \mathcal L_{\text{\cancel{SUSY}}};
\\%
&\mathcal L\w{matter}=
\Phi_Q^*
\exp\left(2g_Y(\tfrac16)V_B+2g_2V_W^aT^a+2g_3V_g^a\tau^a\right)\Phi_Q\Big|_{\theta^4} + \cdots;
\\&
\mathcal L\w{gauge}
=
\left[
 \frac14\left(1-\frac{\ii g_Y^2\Theta_B}{8\pi^2}\right)\mathcal W_B\mathcal W_B
+ \frac14\left(1-\frac{\ii g_2^2\Theta_W}{8\pi^2}\right)\mathcal W_W^a\mathcal W_W^a
+ \frac14\left(1-\frac{\ii g_3^2\Theta_g}{8\pi^2}\right)\mathcal W_g^a\mathcal W_g^a
\right]_{\theta^2} +\text{H.c.};
\\&
\mathcal L\w{super} = W(\Phi)\Big|_{\theta^2}+\text{H.c.},
\\&\qquad W(\Phi)=W\w{RPC}+W\w{RPV},
\\&\qquad
 W\w{RPC} = \mu\Hu\Hd
           - \yu[ij] U^\cc_i\Hu Q_j
           + \yd[ij] D^\cc_i\Hd Q_j
           + \ye[ij] E^\cc_i\Hd L_j,
\\&\qquad
 W\w{RPV} = -\kappa_i L_i\Hu
           + \frac12 \lambda  _{ijk} L_i L_j E^\cc_k
           +         \lambda' _{ijk} L_i Q_j D^\cc_k
           + \frac12 \lambda''_{ijk} U^\cc_i D^\cc_j D^\cc_k;
\\&
\mathcal L\w{FI}=\Lambda\w{FI} D_B;
\\&
\mathcal L_{\text{\cancel{SUSY}}}=
- \frac12\left(M_3\tig\tig+M_2\tiw\tiw+M_1\tib\tib+\text{H.c.}\right)
-
\left(
V^{\mathrm{RPC}}_{\text{\cancel{SUSY}}}
+V^{\mathrm{RPV}}_{\text{\cancel{SUSY}}}\right),
\\&\qquad
V^{\mathrm{RPC}}_{\text{\cancel{SUSY}}}=
\left(\tqL^*m_Q^2\tqL+\tlL^*m_L^2\tlL+\tuR^* m_{U^\cc}^2\tuR + \tdR^* m_{D^\cc}^2\tdR + \teR^* m_{E^\cc}^2\teR
       +m^2_{\Hu}|\hu|^2 + m^2_{\Hd}|\hd|^2\right)
\notag\\&\qquad\qquad\qquad
       +\left(-\tuR^*\hu {a\w u}\tqL + \tdR^* \hd {a\w d} \tqL+ \teR^* \hd {a\w e} \tlL + b\Hu\Hd+\text{H.c.}\right)
\notag\\&\qquad\qquad\qquad
       +\left(\tuR^*\hd^*{c\w u}\tqL+\tdR^* \hu^*{c\w d}\tqL+\teR^*\hu^*{c\w e}\tlL+\text{H.c.}\right),
\\&\qquad
V^{\mathrm{RPV}}_{\text{\cancel{SUSY}}}
=
    \left(-b_i\tlL[i]\Hu +\frac12T_{ijk}\tlL[i]\tlL[j]\teR[k]^*+T'_{ijk}\tlL[i]\tqL[j]\tdR[k]^*
             +\frac12T''_{ijk}\tuR[i]^*\tdR[j]^*\tdR[k]^*
    +\tlL[i]^*M^2_{Li}\Hd+\text{H.c.}\right)
\notag\\&\qquad\qquad\qquad
   + \left(C^1_{ijk}\tlL[i]^*\tqL[j]\tuR[k]^* + C^2_i\hu^*\hd\teR[i]^* + C^3_{ijk}\tdR[i]\tuR[j]^*\teR[k]^*+\frac12C^4_{ijk}\tdR[i]\tqL[j]\tqL[k]+\text{H.c.}\right).
\end{align}
As usual, we remove $\Theta_W$ and $\Theta_B$ by rotating fermions\footnote{%
Fail-safe memo:
The $\Theta$-terms are total derivatives and relevant in non-perturbative discussion.
Redefinition of chiral fermions generates those terms (Fujikawa method) as a non-perturbative effect, so we can remove $\Theta$-terms as long as we have such freedoms.
Note also that the absence of gauge anomaly means the corresponding gauge transformations do not induce additional $\Theta$-terms.
}
, which is compatible with mass diagonalization (discussed later), and assume the absence of Fayet-Illiopoulos term: $\Lambda\w{FI}=0$.
The SU(3) angle $\Theta_g$ forms QCD phase $\Theta\w{QCD}$ together with the phases from Yukawa matrices.
Then,
\begin{align}
 \mathcal L\w{matter} &= \sum\w{matters}\Bigl[\DD^\mu\phi^*\DD_\mu\phi+\ii\bpsi\bsigma^\mu\DD_\mu\psi
- \sqrt2\sum\w{gauge} g\left(\lambda^a(\phi^*t^a\psi)+\blambda^a(\bpsi t^a\phi)\right)
\Bigr] + \text{($F$-terms)},\\
 \mathcal L\w{gauge} &= \sum\w{gauges}\Bigl(-\frac14F^a_{\mu\nu}F^{a\mu\nu}+\ii\blambda^a\bsigma^\mu\DD_\mu\lambda^a\Bigr) + \frac{g_3^2\Theta_g}{64\pi^2}\epsilon^{\mu\nu\rho\sigma}G^a_{\mu\nu}G^a_{\rho\sigma}+ \text{($D$-terms)},
\\
\begin{split}
  \mathcal L\w{super} &=
   \input{calculator/mssm/superfermion.txt}
+ \text{H.c.} +\text{($F$-terms)},
\end{split}\\
\mathcal L_{\text{\cancel{SUSY}}}
&=
- \frac12\left(M_3\tig\tig+M_2\tiw\tiw+M_1\tib\tib+\text{H.c.}\right)
-
\left(
V^{\mathrm{RPC}}_{\text{\cancel{SUSY}}}
+V^{\mathrm{RPV}}_{\text{\cancel{SUSY}}}\right),
\end{align}
and the $F$- and $D$-terms form the supersymmetric scalar potential
\begin{align}
& V\w{SUSY} = F^{*}_i F_i+\frac12 D^aD^a;
\qquad
 F_i = -W_i^* = -\frac{\delta W^*}{\delta \phi_i^*},
\qquad
 D^a = -g(\phi^* t^a \phi),
\\
&V=V\w{SUSY} +
V^{\mathrm{RPC}}_{\text{\cancel{SUSY}}}
+V^{\mathrm{RPV}}_{\text{\cancel{SUSY}}},
\end{align}
where $t_a$ corresponds to the gauge-symmetry generator relevant for each $\phi$.

Each auxiliary term is given by
\begin{align}
\input{calculator/mssm/Fterms.txt}
\\
D\w{SU(3)}^\alpha &= -g_3\sum_{i=1}^3\left(
 \sum_{a=1,2}
  \tqL[i]^{a*} \tau^\alpha \tqL[i]^{a}
- \tuR[i]^{*} \tau^\alpha \tuR[i]
- \tdR[i]^{*} \tau^\alpha \tdR[i]
\right),\\
%
D\w{SU(2)}^\alpha &= -g_2\left[
   \sum_{i=1}^3\left(
  \sum_{x=1}^3 \tqL[i]^{x*}T^\alpha\tqL[i]^{x}
+              \tlL[i]^{*}T^\alpha \tlL[i]
  \right)
+ \hu^*T^\alpha\hu
+ \hd^*T^\alpha\hd
\right],\\
%
 D\w{U(1)} &= -g_1\left(
  \frac16|\tqL|^2
- \frac12|\tlL|^2
- \frac23|\tuR|^2
+ \frac13|\tdR|^2
+        |\teR|^2
+ \frac12|\hu|^2
- \frac12|\hd|^2
\right).
\end{align}

\subsection{Full Lagrangian}
Here the Lagrangian
 $\mathcal L = \mathcal L\w{vector} + \mathcal L\w{fermions} + \mathcal L\w{SFG} + \mathcal L\w{scalar}$ is explicitly given:
\begin{align}
 \mathcal L\w{vector}&=
- \frac14F^a_{\mu\nu}F^{a\mu\nu}
+ \frac{g_3^2\Theta_g}{64\pi^2}\epsilon^{\mu\nu\rho\sigma}G^a_{\mu\nu}G^a_{\rho\sigma},
\\
 \mathcal L\w{fermions}&=
  \ii\bpsi\bsigma^\mu\DD_\mu\psi
 + \ii\blambda^a\bsigma^\mu\DD_\mu\lambda^a
 - \frac12\left(M_3\tig\tig+M_2\tiw\tiw+M_1\tib\tib+\text{H.c.}\right)
 + \mathcal L\w{super}\big|_{\text{no $F$-terms}},
\\\mathcal L\w{SFG}&=
 -\sqrt2g\lambda^a(\phi^*t^a\psi) -\sqrt2g\blambda^a(\bpsi t^a\phi),
\\\mathcal L\w{scalar}&=
\DD^\mu\phi^*\DD_\mu\phi - V.
\end{align}
\subsubsection{Vector part}
\begin{align}
\begin{split}
   \mathcal L\w{vector}
 &=
 \input{calculator/mssm/lag_vector.txt},
\end{split}
\end{align}
where
\begin{align*}
 &W^1_\mu = \frac{W^+_\mu+W^-_\mu}{\sqrt2},\quad
 W^2_\mu = \frac{\ii(W^+_\mu-W^-_\mu)}{\sqrt2};\qquad
 W^\pm_\mu = \frac{W^1_\mu\mp\ii W^2_\mu}{\sqrt2};
\\
 &\pmat{W^3_\mu\\B_\mu} = \pmat{\co{\mathrm w} & \si{\mathrm w} \\ -\si{\mathrm w} & \co{\mathrm w}} \pmat{Z_\mu \\ A_\mu};\qquad
 \pmat{Z_\mu\\A_\mu} = \pmat{\co{\mathrm w} & -\si{\mathrm w} \\ \si{\mathrm w} & \co{\mathrm w}} \pmat{W^3_\mu \\ B_\mu};\\
 &|e| = g_2\sw = g_Y\cw = g_Z\sw\cw, \quad
 g_Z  = g_2/\cw = g_Y/\sw; \qquad
 g_Y = |e|/\cw = g_Z\sw = g_2{\mathrm t}_{\mathrm w},\quad
 g_2 = |e|/\sw = g_Z\cw.
\end{align*}



\subsubsection{Fermion part}
\begin{align}
 \begin{split}
  &\mathcal L\w{fermions}\\
&=
    \ii\bqL\bsigma^\mu \left(\partial_\mu - \ii g_3 g_\mu^a\tau^a - \ii g_2 W_\mu^a T^a - \tfrac16 \ii g_Y B_\mu\right)\qL
\\&\quad
 +  \ii\buR^\cc\bsigma^\mu \left(\partial_\mu + \ii g_3 g_\mu^a\tau^{a*} + \tfrac23 \ii g_Y B_\mu\right)\uR^\cc
 +  \ii\bdR^\cc\bsigma^\mu\left(\partial_\mu + \ii g_3 g_\mu^a\tau^{a*} - \tfrac13 \ii g_Y B_\mu\right)\dR^\cc
\\&\quad
 +  \ii\blL\bsigma^\mu \left(\partial_\mu - \ii g_2 W_\mu^aT^a + \tfrac12 \ii g_Y B_\mu\right)\lL
 +  \ii\beR^\cc\bsigma^\mu\left(\partial_\mu - \ii g_Y B_\mu\right)\eR^\cc
\\&\quad
 +  \ii\bthu\bsigma^\mu\left(\partial_\mu - \ii g_2 W_\mu^aT^a - \tfrac12 \ii g_Y B_\mu\right)\thu
 +  \ii\bthd\bsigma^\mu\left(\partial_\mu - \ii g_2 W_\mu^aT^a + \tfrac12 \ii g_Y B_\mu\right)\thd
\\&\quad
 + \ii\bar\tig^a\bsigma^\mu\left(\partial_\mu\tig^a + g_3f^{abc}g_\mu^b\tig^c\right)
 + \ii\bar\tiw^a\bsigma^\mu\left(\partial_\mu\tiw^a + g_2\epsilon^{abc}W^b_\mu\tiw^c\right)
 + \ii\bar\tib\bsigma^\mu\partial_\mu\tib
\\&\quad
 - \frac12\left(M_3\tig^a\tig^a+M_2\tiw^a\tiw^a+M_1\tib\tib+\text{H.c.}\right)
 + \mathcal L\w{super}\big|_{\text{no $F$-terms}}
 \end{split}
\\
\begin{split}
&=
 \ii\bar\tib\bsigma^\mu\partial_\mu\tib - \frac12\left(M_1\tib\tib+M_1^*\bar\tib^a\bar\tib^a\right)
 + \ii\bar\tig^a\bsigma^\mu\partial_\mu\tig^a - \frac12\left(M_3\tig^a\tig^a+M_3^*\bar\tig^a\bar\tig^a\right)
 - \ii g_3 f^{abc} (\bar\tig^a\bsigma^\mu\tig^b) g_\mu^c
\\&\quad
 + \ii\bar\tiw^+\bsigma^\mu \partial_\mu\tiw^+
 + \ii\bar\tiw^-\bsigma^\mu \partial_\mu\tiw^-
 + \ii\bar\tiw^3\bsigma^\mu \partial_\mu\tiw^3
 -\left(M_2\tiw^+\tiw^- + M_2^*\bar\tiw^+\bar\tiw^-\right)
 -\frac12\left(M_2\tiw^3\tiw^3 + M_2^*\bar\tiw^3\bar\tiw^3\right)
\\&\quad
 + g_2(\bar{\tiw}^{3}\bsigma^\mu\tiw^{-} - \bar{\tiw}^{+}\bsigma^\mu\tiw^{3})W_{\mu}^{+}
 - g_2(\bar{\tiw}^{3}\bsigma^\mu\tiw^{+} - \bar{\tiw}^{-}\bsigma^\mu\tiw^{3})W_{\mu}^{-}
 + g_2(\bar{\tiw}^{+}\bsigma^\mu\tiw^{+} - \bar{\tiw}^{-}\bsigma^\mu\tiw^{-})(\co{\mathrm w}Z_{\mu} + \si{\mathrm w}A_{\mu})
\\&\quad
 + \buL\bsigma^\mu(\ii\partial_{\mu} + g_3\tau^ag_{\mu}^{a})\uL
 + \buR^{\cc}\bsigma^\mu(\ii\partial_{\mu} - g_3\tau^{a*} g_{\mu}^{a})\uR^{\cc}
 + {\ii} \bnuL\bsigma^\mu\partial_{\mu}\nuL
\\&\quad
 + \bdL\bsigma^\mu(\ii\partial_{\mu} + g_3\tau^ag_{\mu}^{a})\dL
 + \bdR^{\cc}\bsigma^\mu(\ii\partial_{\mu} - g_3\tau^{a*} g_{\mu}^{a})\dR^{\cc}
 + {\ii} \beL\bsigma^\mu\partial_{\mu}\eL
 + {\ii} \beR^{\cc }\bsigma^\mu\partial_{\mu}\eR^{\cc }
\\&\quad
 + {\ii} \bthdm\bsigma^\mu\partial_{\mu}\thdm
 + {\ii} \bthdz\bsigma^\mu\partial_{\mu}\thdz
 + {\ii} \bthup\bsigma^\mu\partial_{\mu}\thup
 + {\ii} \bthuz\bsigma^\mu\partial_{\mu}\thuz
\\&\quad
 + \frac{g_2}{\sqrt2} \left(\buL\bsigma^\mu\dL
 + \bnuL\bsigma^\mu\eL
 + \bthup\bsigma^\mu\thuz
 + \bthdz\bsigma^\mu\thdm\right) W_{\mu}^{+}
 + \frac{g_2}{\sqrt2} \left(\bdL\bsigma^\mu\uL
 + \beL\bsigma^\mu\nuL
 + \bthuz\bsigma^\mu\thup
 + \bthdm\bsigma^\mu\thdz\right) W_{\mu}^{-}
\\&\quad
+ \frac{g_Z(3-4\sisi{\mathrm w})}{6}\buL\bsigma^\mu\uL Z_{\mu}
 + \frac{g_Z\sisi{\mathrm w}}{3}\buR^{\cc}\bsigma^\mu\uR^{\cc}Z_{\mu}
+ \frac{g_Z(2\sisi{\mathrm w}-3)}{6}\bdL\bsigma^\mu\dL Z_{\mu}
 - \frac{g_Z\sisi{\mathrm w}}{3}\bdR^{\cc}\bsigma^\mu\dR^{\cc}Z_{\mu}
\\&\quad
+ \frac{g_Z}{2} \bnuL\bsigma^\mu\nuL Z_{\mu}
+ \frac{g_Z(2\sisi{\mathrm w}-1)}{2} \beL\bsigma^\mu\eL Z_{\mu}
 - g_Z\sisi{\mathrm w}\beR^{\cc }\bsigma^\mu\eR^{\cc }Z_{\mu}
\\&\quad
+ \frac{g_Z(1-2\sisi{\mathrm w})}{2}\bthup\bsigma^\mu\thup Z_{\mu}
+ \frac{g_Z}{2} \bthuz\bsigma^\mu\thuz Z_{\mu}
+ \frac{g_Z(2\sisi{\mathrm w} - 1)}{2} \bthdm\bsigma^\mu\thdm Z_{\mu}
+ \frac{g_Z}{2}\bthdz\bsigma^\mu\thdz Z_{\mu}
\\&\quad
 + \frac{2|e|}{3}(\buL\bsigma^\mu\uL - \buR^{\cc}\bsigma^\mu\uR^{\cc})A_{\mu}
 - \frac{|e|}{3}(\bdL\bsigma^\mu\dL - \bdR^{\cc}\bsigma^\mu\dR^{\cc})A_{\mu}
 - |e|(\beL\bsigma^\mu\eL-\beR^{\cc }\bsigma^\mu\eR^{\cc }) A_{\mu}
\\&\quad
 + |e|\bthup\bsigma^\mu\thup A_{\mu}
 - |e|\bthdm\bsigma^\mu\thdm A_{\mu}
 + \mathcal L\w{super}\big|_{\text{no $F$-terms}};
 \end{split}
\end{align}
here,
\begin{equation}
\begin{split}
  \mathcal L\w{super}\big|_{\text{no F-terms}}&=
 \input{calculator/mssm/lag_fermion_super.txt} + \text{H.c.}
\end{split}
\end{equation}



\subsubsection{Scalar-fermion-gaugino interaction}
\begin{equation}
\begin{split}
  \mathcal L\w{SFG} &=
 \input{calculator/mssm/lag_sfg.txt}
\end{split}
\end{equation}

\subsubsection{Scalar part}
\begin{equation}
\begin{split}
  \mathcal L\w{scalar} &=
 \input{calculator/mssm/lag_scalar_kin.txt}
\\&\quad -\left(V\w{SUSY} + V_{\cancel{\mathrm{SUSY}}}\right),
\end{split}
\end{equation}
where the scalar potential is given by
\begin{equation}
  \begin{split}
   V\w{SUSY} &= \input{calculator/mssm/SUSYPotential.txt}.
  \end{split}
\end{equation}

\subsection{Higgs mechanism and fermion composition}
The scalar potential includes
\begin{align}
\begin{split}
 V\w{SUSY}&\supset
 |\hu|^2\Bigl( |\mu|^2 + \sum|\kappa_i|^2 \Bigr)
 + |\mu|^2|\hd|^2
 + \frac{g_Z^2}{8}\left(|\hu|^2 - |\hd|^2\right)^2
 + \frac{g_2^2}{2}|\hd^{*}\hu|^2
\\&\qquad
 + \Bigl( \kappa^*_{i}\mu\tlL[i]^{*}\hd + \text{H.c.} \Bigr)
 + \kappa^{*}_{i}\kappa_{j}\tlL[i]^{*}\tlL[j]^{}
\end{split}
\\
V_{\text{\cancel{SUSY}}}&\supset
       m^2_{\Hu}|\hu|^2 + m^2_{\Hd}|\hd|^2 +\epsilon^{ab}\left(b\hu^a\hd^b + b^*\hu^{a*}\hd^{b*}
       -b_i\tlL[i]^a\hu^b-b_i^*\tlL[i]^{a*}\hu^{b*}\right)+\tlL[i]^*M^{2}_{Li}\hd + \tlL[i]M^{2*}_{Li}\hd^*;
\end{align}
the Higgs mass term is given by
\begin{equation}
 V\supset\pmat{\hu &\hd^* &\tlL[i]^*}
\pmat{
  |\mu|^2+m^2_{\Hu}+\sum|\kappa_i|^2 & b & -b_j\\
  b^* & |\mu|^2+m^2_{\Hd} & \kappa_j\mu^*+M^{2*}_{Lj}\\
  -b_i^* & \kappa_i^*\mu+M^{2}_{Li} & (m_L^2)_{ij} + \kappa_i^*\kappa_j
}
\pmat{\hu^* \\\hd\\\tlL[j]}
\end{equation}
while corresponding fermion terms are
\begin{equation}
 \mathcal L\supset \epsilon^{ab}\left(
  -\mu\thu^a\thd^b-\kappa_i\thu^a\lL[i]^b
\right).
\end{equation}
If the $R$-parity is not conserved, we redefine $(\Hd, L)$ superfields so that the mass matrix is block-diagonal, which corresponds to $\gU(4)_{\Hd,L}\to\gU(3)_L\times\gU(1)_{\Hd}$ (DOF counting: $16\to9+1$ to remove $b'_i$).
Then lepton and $\thd$ are mixed.\footnote{If we separated leptons and $\thd$ first, sleptons would acquire VEVs and lepton-gaugino mixings would be induced.}
With $R$-parity conservation, we do not suffer from these mixings.

\subsubsection{Higgs potential and induced mass in  R-parity conserved case}
We perform ``SU(2)-notation fixing'', i.e., use the freedom associated to $T_1$ and $T_2$ of SU(2), so that $\vev{\hup}=0$. Then $\vev{\hdm}=0$ and effectively
\begin{equation}
 V\w{pot}=
   (|\mu|^2 + m^2_{\Hu})  |\huz|^2
 + (|\mu|^2 + m^2_{\Hd}) |\hdz|^2
 + \frac{g_Z^2}{8}\left(|\huz|^2 - |\hdz|^2\right)^2
 - \left(b\huz\hdz +\text{H.c.}\right).
\end{equation}
We redefine $\Hd$ superfield so that $b>0$.\footnote{%
Note that $T_3$-rotation induces $\huz\to\ee^{\ii\theta/2}\huz$ and $\hdz\to\ee^{-\ii\theta/2}\hdz$; it cannot the remove phase of $b$.}
Then $\arg\vev{\huz}=-\arg\vev{\hdz}$ and, with $T_3$-rotation, $\vev{\huz}>0$ and $\vev{\hdz}>0$:
\begin{align}
 &\vev{\huz} =: \vu =: \frac{v\w{SM}}{\sqrt2}\sin\beta,\qquad
 \vev{\hdz} =: \vd =: \frac{v\w{SM}}{\sqrt2}\cos\beta;
\\& V\w{pot}=
   \frac12(|\mu|^2 + m^2_{\Hu})  v\w{SM}^2\sin^2\beta
 + \frac12(|\mu|^2 + m^2_{\Hd}) v\w{SM}^2\cos^2\beta
 + \frac{g_Z^2}{32}v\w{SM}^4\cos^2 2\beta
 - \frac12 v\w{SM}^2 b\sin2\beta.
\end{align}
This potential can have two minima; one with $0<\beta\le\pi/4$ and the other with $\pi/4\le\beta<\pi/2$:
\begin{equation}
 \tan\beta = \frac{B\mp\sqrt{B^2-4b^2}}{2b}
\quad
\left(\cos2\beta = \pm\frac{\sqrt{B^2-4b^2}}{B}\right),
 \qquad
 m_Z^2:=\frac{g_Z^2}{4}v\w{SM}^2 = \left(\pm\frac{m_{\Hd}^2 - m_{\Hu}^2}{\sqrt{B^2-4b^2}}-1\right)B,
\end{equation}
where $B:=2|\mu|^2+m_{\Hu}^2+m_{\Hd}^2>2b>0$ and $m_Z$ is the $Z$-boson tree-level mass.
Also,
\begin{equation}
 \sin2\beta = \frac{2b}{2|\mu|^2 + m_{\Hu}^2 + m_{\Hd}^2},
\qquad
 m_Z^2 = \frac{-(m_{\Hd}^2 - m_{\Hu}^2)}{\cos2\beta} - \left(2|\mu|^2 + m_{\Hu}^2 + m_{\Hd}^2\right)
\end{equation}
are satisfied in both solutions.

\paragraph{Higgs sector}
The \NAMBU-Goldstone--Higgs mixings and the mass terms for the charged Higgs bosons are given by
\begin{equation}
\begin{split}
  \mathcal L&\supset
 \partial_\mu\hd^{-*} \partial^\mu\hd^{-}
+ \partial_\mu\hu^{+*} \partial^\mu\hu^+
+ \left(-b-\frac12g_2^2\vu\vd\right)(\hup\hdm+\hu^{+*}\hd^{-*})
\\&\quad
+ \left[\frac{g_Y^2(\vu^2-\vd^2)-g_2^2(\vu^2+\vd^2)}{4}-|\mu|^2-m_{\Hd}^2\right]|\hdm|^2
+ \left[\frac{g_Y^2(\vd^2-\vu^2)-g_2^2(\vu^2+\vd^2)}{4}-|\mu|^2-m_{\Hu}^2\right] |\hup|^2
\\&\quad
+\frac{\ii g_2}{\sqrt2}W^-_\mu\partial^\mu\left(
\vu\hup - \vu \hu^{+*} - \vd \hd^{-*} + \vd \hdm
\right)
\end{split}
\end{equation}
and those for the neutral Higgs bosons are
\begin{equation}
\begin{split}
 \mathcal L&\supset
  \partial_\mu\hd^{0*} \partial^\mu\hd^0
 + \partial_\mu\hu^{0*} \partial^\mu\hu^0
 -\frac{g_Z^2\vd^2}{8}(\hdz\hdz+\hd^{0*}\hd^{0*})
 -\frac{g_Z^2\vu^2}{8}(\huz\huz+\hu^{0*}\hu^{0*})
\\&\quad
 +\left(b+\frac{g_Z^2\vu\vd}{4}\right)(\huz\hdz+\hu^{0*}\hd^{0*})
 +\frac{g_Z^2\vu\vd}{4}(\huz\hd^{0*}+\hu^{0*}\hdz)
\\&\quad
 + \left(\frac{g_Z^2(\vu^2-2\vd^2)}{4}-|\mu|^2-m_{\Hd}^2\right)|\hdz|^2
 + \left(\frac{g_Z^2(\vd^2-2\vu^2)}{4}-|\mu|^2-m_{\Hu}^2\right)|\hdz|^2
\\&\quad
 + \frac{\ii g_Z}{2}Z_\mu\partial^\mu\left(
  \vd\hdz - \vd \hd^{0*} - \vu\huz + \vu \hu^{0*}
\right).
\end{split}
\end{equation}
Therefore, with $m_W:=\cw m_Z$ and
\begin{equation}
 \pmat{\hup\\\hd^{-*}} = \pmat{\si\beta & \co\beta \\ -\co\beta & \si\beta} \pmat{-\ii G^+\\H^+},
\quad
 \pmat{\huz\\\hdz}
 = \frac{1}{\sqrt2}\pmat{\phi\w u\\\phi\w d}
+  \frac{\ii}{\sqrt2} \pmat{\si\beta & \co\beta \\ -\co\beta & \si\beta} \pmat{G^0\\A^0},
\end{equation}
we have
\begin{equation}
\begin{split}
  \mathcal L &\supset
\partial_\mu G^{+*}\partial^\mu G^+
+\partial_\mu H^{+*}\partial^\mu H^+
+m_W(W^-_\mu\partial^\mu G^+ + W^+_\mu\partial^\mu G^{+*})
+ \left(\frac{m_{\Hd}^2-m_{\Hu}^2}{\cos2\beta} + m_Z^2\sw^2\right)|H^+|^2
\\&\quad
+\frac12(\partial_\mu\phi_1)^2
+\frac12(\partial_\mu\phi_2)^2
+\frac12(\partial_\mu A^0)^2
+\frac12(\partial_\mu G^0)^2
+m_Z Z_\mu\partial^\mu G^0
-\frac{B}{2}A_0^2
\\&\quad
-\frac14\left(B+m_Z^2+(B-m_Z^2)\cos2\beta\right)\phi\w u^2.
-\frac14\left(B+m_Z^2-(B-m_Z^2)\cos2\beta\right)\phi\w d^2
+\frac12(B+m_Z^2)(\sin 2\beta)\phi\w u\phi\w d.
\end{split}
\end{equation}
In particular, the tree-level masses are
\begin{align}
 m_{A_0}^2 &= B = 2|\mu|^2+m_{\Hu}^2+m_{\Hd}^2,
\\
 m_{H^+}^2 &= m_{A_0}^2 + m_W^2,\\
 m_{h,H} &= \frac12\left(m_{A_0}^2 + m_Z^2 \mp \sqrt{\left(m_{A_0}^2-m_Z^2\right)^2+4m_{A_0}^2m_Z^2\sin^22\beta}\right)
\end{align}
with
\begin{equation}
 \pmat{\phi\w d\\\phi\w u} = \pmat{\cos\alpha&-\sin\alpha\\\sin\alpha & \cos\alpha}\pmat{H\\h},
\qquad
\frac{\tan2\alpha}{\tan2\beta} = \frac{m_{A_0}^2+m_Z^2}{m_{A_0}^2-m_Z^2}.
\end{equation}

The mixing $\alpha$ is stored in \texttt{ALPHA} block of SLHA, while \texttt{HMIX} stores
\begin{equation}
 \CONV{\mu}={\mu},\quad
 \CONV{\tan\beta}=\tan\beta,\quad
 \CONV{v}=v\w{SM}(\sim246\GeV),\quad
 \CONV{m_A^2}=\frac{2b}{\sin2\beta},
\end{equation}
at the scale specified.
The above discussion holds even with $CP$-violation, but quantum corrections mix the three Higgs bosons; such information should be stored in \texttt{(IM)VCHMIX}.
\TODO{discuss when needed}


\paragraph{Mass terms in the Lagrangian}
The other mass terms are given by
\begin{equation}
 \begin{split}
\mathcal L&\supset
   \input{calculator/mssm/lag_massterms.txt},
 \end{split}
\end{equation}
where, at the tree level, the gauge boson mass $m_W$ and $m_Z$, the gluino mass $M_3$, and matter-fermion masses $\vu\yu$, $\vd\yd$, and $\vd\ye$ are given with the ``correct'' sign (as far as $M_3>0$, etc.).

\paragraph{Neutralinos and charginos}
The mass matrices for neutralinos and charginos are given by
\begin{equation}
\begin{split}
  -\mathcal L&\supset
 \frac12\pmat{\tib\\\tiw^3\\\thdz\\\thuz}^\TT
 \pmat{
 M_1&0&-\co\beta\sw m_Z&+\si\beta\sw m_Z \\
 0&M_2&+\co\beta\cw m_Z&-\si\beta\cw m_Z \\
 -\co\beta\sw m_Z&+\co\beta\cw m_Z&0&-\mu \\
 +\si\beta\sw m_Z&-\si\beta\cw m_Z&-\mu&0
 }
 \pmat{\tib\\\tiw^3\\\thdz\\\thuz} + \text{h.c.}
\\&\quad
+\pmat{\tiw^- & \thdm}
\pmat{ M_2& \sqrt2\si\beta m_W\\ \sqrt2\co\beta m_W& \mu}
\pmat{\tiw^+ \\ \thup}
+\pmat{\bar\tiw^- & \bthdm}
\pmat{ M_2^*& \sqrt2\si\beta m_W\\ \sqrt2\co\beta m_W& \mu^*}
\pmat{\bar\tiw^+ \\ \bthup}.
\end{split}
\end{equation}
Note that the mass matrices themselves are the same as those in SLHA convention, $\mathcal M_{\tilde\psi^0}$ and $\mathcal M_{\tilde\psi^+}$, while the fields are in different convention.
Therefore, we continue our discussion based only on the mass matrices so that the discussion is free from the choice of field convention.

As $\mathcal M_{\tilde\psi^0}$ is a complex symmetric matrix, there is a unitary matrix $\tilde N$ such that
$M_{\tilde\psi^0}=\tilde N^* \mathcal M_{\tilde\psi^0}\tilde N^\dagger$, where $M_{\tilde\psi^0}$ is a \emph{positive} diagonal matrix whose elements are (non-negative) singular values of $\mathcal M_{\tilde\psi^0}$ and in increasing order (Autonne-Takagi factorization).
In SLHA2 convention with $CP$-violation, this matrix $\tilde N$ is stored as the \texttt{(IM)NMIX} blocks and the (positive) masses are stored in the \texttt{MASS} block.
Meanwhile, if $M_1$, $M_2$ and $\mu$ are real, $\mathcal M_{\tilde\psi^0}$ is a real symmetric matrix and there is a real orthogonal matrix $\hat N$ such that
$\hat M_{\tilde\psi^0}=\hat N^* \mathcal M_{\tilde\psi^0}\hat N^\dagger=\hat N \mathcal M_{\tilde\psi^0}\hat N^\TT$, where $\hat M_{\tilde\psi^0}$ is a \emph{real} diagonal matrix whose elements are the eigenvalues of $\mathcal M_{\tilde\psi^0}$ and in absolute-value-increasing order (spectral theorem).
This matrix $\hat N$ is the \texttt{NMIX} block of SLHA convention and $\hat M_{ii}$ is stored in the \texttt{MASS} block, hence \texttt{MASS} block may have negative values for neutralinos.

The chargino mass matrix $\mathcal M_{\tilde\psi^+}$ is decomposed as
$M_{\tilde\psi^+}=U^* \mathcal M_{\tilde\psi^+} V^\dagger$, where $U$ and $V$ are unitary matrices and the elements of the diagonal matrix $M_{\tilde\psi^+}$ are singular values of $\mathcal M_{\tilde\psi^+}$ (thus non-negative) and sorted in increasing order (singular value decomposition).
These $U$ and $V$ are stored in \texttt{(IM)UMIX} and \texttt{(IM)VMIX}, and the singular values are stored in \texttt{MASS} block.
Because the SVD theorem is closed in $\mathbb R$, if $M_2$ and $\mu$ are real, $U$ and $V$ can be real, and the \texttt{IM}-blocks are omitted.

In summary,
\begin{align}
      M_{\tilde\psi^0}&=\tilde N^* \mathcal M_{\tilde\psi^0}\tilde N^\dagger,
   && \tilde N=\texttt{(IM)NMIX},
   && \text{(\texttt{MASS})} = [M_{\tilde\psi^0}]_{ii} \ge 0 \quad\text{(singular values)};
\\
 \hat M_{\tilde\psi^0}&=\hat   N \mathcal M_{\tilde\psi^0}\hat N^\TT,
  && \hat N = \texttt{NMIX},
  &&  \text{(\texttt{MASS})} = [\hat M_{\tilde\psi^0}]_{ii}\in \mathbb R \quad\text{(eigenvalues)};
\\
      M_{\tilde\psi^+}&=U^* \mathcal M_{\tilde\psi^+} V^\dagger,
   && U=\texttt{(IM)UMIX},\quad V=\texttt{(IM)VMIX},
   && \text{(\texttt{MASS})} = [M_{\tilde\psi^+}]_{ii} \ge 0\quad\text{(singular values)}.
\end{align}
Note that the singular values are equal to absolute values of the eigenvalues, which guarantees consistency of the two decomposition.

We then define matrix $N$ by\footnote{The sign of $\varphi_i$ is arbitrary and (should be) unphysical.}
\begin{align}
 &N = \begin{cases}
  \tilde N\\
  \diag(\varphi_i) \cdot \hat N
  \end{cases}
   = \diag(\varphi_i) \cdot\Bigl(\text{(\texttt{NMIX})} + \ii\text{(\texttt{IMNMIX})}\Bigr);
  &\varphi_i = \begin{cases}
 1 & \text{if (\texttt{MASS})$_i\ge0$,}\\
 \ii & \text{if (\texttt{MASS})$_i<0$.}
             \end{cases}
\end{align}
It gives the proper mass diagonalization in both of the \text{NMIX} convention:
\begin{equation}
 N^*\mathcal M_{\tilde\psi^0}N^\dagger
=
\begin{cases}
 \tilde N^*\mathcal M_{\tilde\psi^0}\tilde N^\dagger = M_{\tilde\psi^0},\\
 \diag(\varphi_i^*)\hat N^* \mathcal M_{\tilde\psi^0}\hat N^{\dagger}\diag(\varphi_i^*)
 =
 \diag(\varphi_i^*)\hat M_{\tilde\psi^0}\diag(\varphi_i^*)
\end{cases}
 = M_{\tilde\psi^0}
 \quad\text{(neutralino masses $\ge 0$)}.
\end{equation}
Noting that the discussion up here is irrelevant of the convention, we have the neutralino/chargino mass eigenstates,
\begin{equation}
 \neut[i] = N_{ij} \pmat{\tib\\\tiw^3\\\thdz\\\thuz}_j,
\quad
 \charP[i] = V_{ij} \pmat{\tiw^+\\\thup}_j,
\quad
 \charM[i] = U_{ij} \pmat{\tiw^-\\\thdm}_j,
\end{equation}
in our convention and the mass terms are now
\begin{equation}
  -\mathcal L\supset
\frac12(\neut)^\TT M_{\tilde\psi^0} \neut
+ (\charM)^\TT M_{\tilde\psi^+}\charP + \text{h.c.}
\end{equation}

\paragraph{Quarks, leptons, and super-CKM basis}
We here take the super-CKM basis.
In the ``original'' Lagrangian,
\begin{align}
 -\mathcal L&\supset
   \uR^\cc (\vu \yu)\uL
  +\dR^\cc (\vd \yd)\dL
  +\eR^\cc (\vd \ye)\eL + \text{h.c.}
 \\
&=
   \uR^\cc (\vu U_u \yu^{\mathrm{diag}} V_u^\dagger)\uL
  +\dR^\cc (\vd U_d \yd^{\mathrm{diag}} V_d^\dagger)\dL
  +\eR^\cc (\vd U_e \ye^{\mathrm{diag}} V_e^\dagger)\eL + \text{h.c.},
\end{align}
so the super-CKM basis is given by
\begin{equation}
 \bigl[
 Q^1, Q^2, L, U^\cc, D^\cc, E^\cc
\bigr]_{\text{super-CKM}} =
 \bigl[
 V_u^\dagger Q^1, V_d^\dagger Q^2, V_e^\dagger L, U^\cc U_u, D^\cc U_d, E^\cc U_e
\bigr]_{\text{``original''}}.
\end{equation}
Then the CKM mixings appear as, for example,
\begin{equation}
 \left[\buL\bsigma^\mu\dL W^+_\mu + \bdL\bsigma^\mu\uL W^-_\mu\right]_{\text{''original''}}
=
 \left[\buL V_u^\dagger V_d \bsigma^\mu\dL W^+_\mu + \bdL V_d^\dagger V_u\bsigma^\mu\uL W^-_\mu\right]_{\text{super-CKM}};
\end{equation}
i.e., defining $V\w{CKM} = V_u^\dagger V_d$ as in \cref{sec:SM-CKM},
the Lagrangian is amended as, e.g., $\buL\dL\to \buL V\w{CKM}\dL$, $\tdL^*\tuL\to \tdL^* V^\dagger\w{CKM}\tuL$.

The $\Theta_B$- and $\Theta_W$-terms are now removable.
To this end, we rotate the matter superfields as
\begin{equation}
 Q^1\to\ee^{\ii\theta}Q^1,\quad
 Q^2\to\ee^{\ii\theta'}Q^2,\quad
 U^\cc\to\ee^{-\ii\theta}U^\cc,\quad
 D^\cc\to\ee^{-\ii\theta'}D^\cc,
\end{equation}
which induces
$\Delta \Theta_G=0$,
$\Delta \Theta_W\propto \theta+\theta'$,
and
$\Delta \Theta_B\propto 5\theta+\theta'$, while the Lagrangian is unchanged except for the overall phase of $V\w{CKM}$.
So we can remove $\Theta_B$ and $\Theta_W$, or in other words, pass the $CP$-violation in $\Theta$-terms to $V\w{CKM}$.\footnote{%
There are other possible transformations $(L,E^\cc)\to \ee^{(\pm)\ii\theta''}(L,E^\cc)$ and in total we may have $\Delta\Theta_W \propto \theta+\theta'+2\theta''$ and
 $\Delta\Theta_B \propto 5\theta+\theta'+6\theta''$.
 To remove both $\Theta$-terms, we have to take $\theta\neq\theta'$ and $V\w{CKM}$ is anyway modified. The angle may chosen generation-dependently, and then $\Delta\Theta$'s should be read as $\theta\to\sum\theta_i/3$, etc.}
If one (or more) quarks were massless, we could rotate the corresponding right-handed quark to remove $\Theta_G$ as well.

\paragraph{Squark masses in super-CKM basis}
Finally, the squark masses are given by
\begin{equation}
 \begin{split}
-\mathcal L&\supset
 \tuL^*\left(
   m_{Q}^2
 +  m_u^2
 + \frac{3-4\sw^2}{6} \co{2\beta} m_Z^2
\right)\tuL
 +\tuR^*\left(
   m_{U^\cc}^2
 + m_u^2
 + \frac{4\sw^2}{6} \co{2\beta} m_Z^2
\right)\tuR
\\&
 + \tuR^*(\vu a_{{\mathrm u}}- \mu^{*}m_{u}\cot\beta)\tuL
 + \tuL^*(\vu a^{\dagger}_{{\mathrm u}}- \mu m_u\cot\beta)\tuR
\\&
 +\tdL^*\left(
V_d^\dagger (V_u m_{Q}^2 V_u^\dagger)V_d
 + m_d^2
 + \frac{-3+2\sw^2}{6} \co{2\beta} m_Z^2
\right)\tdL
 +\tdR^*\left(
   m_{D^\cc}^2
 + m_d^2
 + \frac{-2\sw^2}{6} \co{2\beta} m_Z^2
\right)\tdR
\\&
 + \tdR^*(\vd a_d -\mu^{*}m_d\tan\beta)\tdL
 + \tdL^*(\vd a^{\dagger}_d-\mu m_d\tan\beta)\tdR
\\&
 +\tnuL^*\left(
 m_{L}^2
 + \frac{1}{2} \co{2\beta} m_Z^2
 \right)\tnuL
\\&
 +\teL^*\left(
 m_{L}^2
 + m_e^2
 + \frac{-1+2\sw^2}{2} \co{2\beta} m_Z^2
\right)
 +\teR^*\left(
  m_{E^\cc}^2
 + m_e^2
 + (-\sw^2)\co{2\beta} m_Z^2
 \right)\teR
\\&
 + \teR^*(\vd a_e -\mu^{*}m_e\tan\beta)\teL
 + \teL^*(\vd a^{\dagger}_e-\mu m_e\tan\beta)\teR,
 \end{split}
\end{equation}
where the sfermion soft masses, yukawas, and $a$-terms are rewritten in super-CKM basis:
\begin{align}
& [m_Q^2, m_{U^\cc}^2, m_{D^\cc}^2,m_L^2,m_{E^\cc}^2]_{\text{super-CKM}}
=
 [V_d^\dagger m_Q^2 V_d, U_u^\dagger m_{U^\cc}^2 U_u, U_d^\dagger m_{D^\cc}^2 U_d,
V_e^\dagger m_L^2 V_e, U_e^\dagger m_{E^\cc}^2 U_e
]_{\text{``original''}},
\\
&[a_u, a_d, a_e]_{\text{super-CKM}}
=[U_u^\dagger a_u V_u, U_d^\dagger a_d V_d, U_e^\dagger a_e V_e
]_{\text{``original''}}
\end{align}
(note that $m_Q^2$ is diagonalied for down-type; not for up-type). In matrix form,
\begin{align}
\begin{split}
  -\mathcal L&\supset
\pmat{\tuL[i]^*& \tuR[i]^*}
\pmat{
 [V\w{CKM}m_Q^2V\w{CKM}^\dagger]_{ij}+\left(m_u^2+\frac{3-4\sw^2}{6}\co{2\beta}m_Z^2\right)\delta_{ij} &
v_u[a_u^\dagger]_{ij}-(\mu m_u\cot\beta)\delta_{ij}\\
v_u[a_u]_{ij}-(\mu^*m_u\cot\beta)\delta_{ij} &
 [m_{U^\cc}^2]_{ij}+\left(m_u^2+\frac{2\sw^2}{3}\co{2\beta}m_Z^2\right)\delta_{ij}
}
\pmat{\tuL[j]\\\tuR[j]}
\\&+
\pmat{\tdL[i]^*& \tdR[i]^*}
\pmat{
 [m_Q^2]_{ij}+\left(m_d^2+\frac{-3+2\sw^2}{6}\co{2\beta}m_Z^2\right)\delta_{ij} &
v_d[a_d^\dagger]_{ij}-(\mu m_d\tan\beta)\delta_{ij}\\
v_d[a_d]_{ij}-(\mu^*m_d\tan\beta)\delta_{ij} &
 [m_{D^\cc}^2]_{ij}+\left(m_d^2-\frac{\sw^2}{3}\co{2\beta}m_Z^2\right)\delta_{ij}
}
\pmat{\tdL[j]\\\tdR[j]}
\\&+
\tnuL[i]^*\Bigl([m_L^2]_{ij}+\left(\tfrac12\co{2\beta}m_Z^2\right)\delta_{ij}\Bigr)\tnuL[j]
\\&+
\pmat{\teL[i]^*& \teR[i]^*}
\pmat{
 [m_L^2]_{ij}+\left(m_e^2+\frac{-1+2\sw^2}{2}\co{2\beta}m_Z^2\right)\delta_{ij} &
v_d[a_e^\dagger]_{ij}-(\mu m_e\tan\beta)\delta_{ij}\\
v_d[a_e]_{ij}-(\mu^*m_e\tan\beta)\delta_{ij} &
 [m_{E^\cc}^2]_{ij}+\left(m_e^2-\sw^2\co{2\beta}m_Z^2\right)\delta_{ij}
}
\pmat{\teL[j]\\\teR[j]}
\end{split}
\\&=
\pmat{\tuL[i]^*& \tuR[i]^*}\mathcal M_u\pmat{\tuL[j]\\\tuR[j]}
+
\pmat{\tdL[i]^*& \tdR[i]^*}\mathcal M_d\pmat{\tdL[j]\\\tdR[j]}
+
\tnuL^*\mathcal M_\nu\tnuL
+
\pmat{\teL[i]^*& \teR[i]^*}\mathcal M_e\pmat{\teL[j]\\\teR[j]}
\end{align}


The sfermion mass matrices are diagonalized by unitary matrices as
\begin{equation}
 \mathcal M=R\mathcal M\wdiag R^\dagger;
\quad
 \tilde f_i = R_{ij}\pmat{\tilde f\w L\\\tilde f\w R}_j
=\pmat{R^{\mathrm L}_{ij} & R^{\mathrm R}_{ij}}\pmat{\tilde f_{\mathrm Lj}\\\tilde f_{\mathrm Rj}};\qquad
\tilde f_{\mathrm Li}= [R^{\mathrm L\dagger}]_{ij}\tilde f_i,\quad
\tilde f_{\mathrm Ri}= [R^{\mathrm R\dagger}]_{ij}\tilde f_i.
\end{equation}
where $R_{ij}$ is $6\times6$ and $R_{ij}^{\mathrm {L,R}}$ are $3\times6$ matrices (except for sneutrinos).

These $R$-matrices are the same as \texttt{DSQMIX} etc.\ of SLHA2 format, but
note that our notation for the other parameters is slightly different from SLHA's:
\begin{align}
 \CONV{m^2_{\tilde Q,\tilde L}} &=  m^2_{Q,L}|_\text{``orig''},&
 \CONV{m^2_{\tilde u,\tilde d,\tilde e}} &=  (m^2_{U^c,D^c,E^c}|_\text{``orig''})^\TT,&
 \CONV{T_{U,D,E}} &= (a_{u,d,e}|_\text{``orig''})^\TT,
\\
 \CONV{\hat m^2_{\tilde Q,\tilde L}} &=  m^2_{Q,L}|_\text{sCKM},&
 \CONV{\hat m^2_{\tilde u,\tilde d,\tilde e}} &=
 \CONV{U^\dagger T^\TT V} = m^2_{U^c,D^c,E^c}|_\text{sCKM},&
 \CONV{\hat T_{U,D,E}} &= \CONV{U^\dagger T^\TT V} = a_{u,d,e}|\w{sCKM},
\end{align}
together with $\CONV{Y_{u,d,e}} = (y_{u,d,e})^\TT$.
Anyway, the SLHA2 blocks corresponds to the variable in our convention as
\begin{equation}
 \texttt{(IM)MSX2(IN)} = m^2_{Q, L, U^\cc, D^\cc, E^\cc}|\w{sCKM;\overline{DR}},
\quad
 \texttt{(IM)TX(IN)} =  a_{u,d,e}|\w{sCKM;\overline{DR}},
\quad
 \texttt{YX} =  y_{u,d,e}|\w{sCKM;\overline{DR}}.
\end{equation}

The sfermion mass matrices above are in super-CKM basis, so their off-diagonal entries immediately induce flavor violation or sfermion left-right mixing.
In old SLHA format, we assume that flavor- and $CP$-violation is absent and left-right mixing is ignorable except for third generation, which leads
\begin{equation}
 \mathcal M_d= \left(\begin{smallmatrix}
  \tilde m^2_{dL11} & & & & &\\
  & \tilde m^2_{dL22} & & & &\\
  & & \tilde m^2_{dL33} & & & v_d a_{d33}-\mu m_b\tan\beta\\
  & & & \tilde m^2_{dR11} & &\\
  & & & & \tilde m^2_{dR22} &\\
  & & v_da_{d33}-\mu m_b\tan\beta&& & \tilde m^2_{dR33}
\end{smallmatrix}\right);\quad
\left(\begin{smallmatrix}
\tilde d\w L\\\tilde s\w L \\\tilde b_1\\\tilde d\w R\\\tilde s\w R \\\tilde b_2
\end{smallmatrix}\right)
=
\left(\begin{smallmatrix}
1&&&&&\\&1&&&&\\
&&F_{11}&&&F_{12}\\
&&&1&&\\&&&&1&\\
&&F_{21}&&&F_{22}\\
\end{smallmatrix}\right)
\pmat{\tilde d_{\mathrm Lj}\\\tilde d_{\mathrm Rj}}
\end{equation}
and these $F_{ij}$ are stored in \texttt{SBOTMIX} etc.


\subsubsection{Fermion composition}
Now we show the fermion-related Lagrangian terms verbosely:
\begin{equation}
  \begin{split}
&  \mathcal L\w{fermions} + \mathcal L\w{SFG}
\\&=
\end{split}
\end{equation}




\newpage
\begin{wraptable}{r}{200pt}\vspace{-3em}
 \begin{tabular}[t]{c@{\,}c@{\,}c@{\,}c@{\,}c}\toprule
 SLHA  && our notation && Martin/DHM\\\midrule
 $\CONV{(H_1, H_2)}$              &$=$& $(\Hd, \Hu)$ \\\midrule
 $\CONV{Y_{\mathrm{u,d,e}}}$      &$=$& $(y_{\mathrm{u,d,e}})^\TT$\\
 $\CONV{T_{\mathrm{u,d,e}}}$      &$=$& $(a_{\mathrm{u,d,e}})^\TT$\\
 $\CONV{A_{\mathrm{u,d,e}}}$      &$=$& $(A_{\mathrm{u,d,e}})^\TT$\\
 $\CONV{m^2_{U^\cc,D^\cc, E\cc}}$ &$=$& $(m^2_{U^\cc,D^\cc, E^\cc})^\dagger$\\
 $\CONV{M_{1,2,3}}$               &$=$& $-M_{1,2,3}$\\
 $\CONV{m_3^2}$                   &$=$& $b$\\
 $\CONV{m_A^2}$                   &$=$& $m^2_{A_0}$ (tree)\\\midrule
 &&$\kappa_i$ &$=$& $\CONV{-\mu_i'}$ \footnotesize{(rarely used)}\\
 $\CONV{D_i}$                     &$=$& $b_i$\\
 $\CONV{m^2_{\tilde L_iH_1}}$     &$=$& $M^2_{Li}$ \\\bottomrule
 \end{tabular}
\end{wraptable}

\subsection{SLHA convention}
The SLHA convention \cite{SLHA} is different from our notation; the reinterpretation rules for the MSSM parameters are given in the right table  (\textbf{\CONV{magenta color}} for objects in other conventions), while
\begin{equation*}
  \mu, b, m^2_{Q, L, \Hu, \Hd}, \text{RPV-trilinears ($\lambda$s and $T$s)}
\end{equation*}
 are in common.



\clearpage

In particular, the chargino/neutralino mass terms in RPC case are given by
\begin{align}
 \mathcal L\supset
&\left[
  \frac12{\CONV{M_1}}\tib\tib
+ \frac12{\CONV{M_2}}\tiw\tiw
- \mu\thu\thd
- \frac{g_Y}{2\sqrt2} \left(\hu^*\thu-\hd^*\thd\right)\tib
- \sqrt{2}{g_2} \left(\hu^*T^a\thu+\hd^*T_a\thd\right)\tiw
\right]+\text{H.c.}
\\
&\to
\frac{1}{2}
\begin{pmatrix}\tib \\ \tiw \\ \huz \\ \hdz\end{pmatrix}^\TT
\begin{pmatrix}
 -M_1 & 0 & -m_Z \co{\beta} \si{w} & m_Z \si{\beta} \si{w} \\
 0 & -M_ 2 & m_Z \co{\beta} \co{w} & -m_Z \si{\beta} \co{w} \\
 -m_Z \co{\beta} \si{w} & m_Z \co{\beta} \co{w} & 0 & -\mu  \\
 m_Z \si{\beta} \si{w} & -m_Z \si{\beta} \co{w} & -\mu  & 0
\end{pmatrix}
\begin{pmatrix}\tib \\ \tiw \\ \huz \\ \hdz\end{pmatrix}
\end{align}

\end{document}


