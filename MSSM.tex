\documentclass[CheatSheet]{subfiles}
\begin{document}



\summarystyle
\section{Minimal Supersymmetric Standard Model}

 Gauge symmetry: $\gSU(3)\w{color}\times\gSU(2)\w{weak}\times\gU(1)_{Y}$

\vspace{1em}


\noindent
 Particle content:
\begin{table}[h!]
\newcommand\h[1]{\multicolumn{1}{@{\,\,}c@{\,\,}}{#1}}
\begin{subtable}[t]{0.635\textwidth}\centering
\caption{Chiral superfields}
\catcode`#=\active \def#{\phantom{-}}
\begin{tabular}[t]{c|ccc|cc|l@{,\,}l@{~}l}\toprule
 \h{}      & \h{$\gSU(3)$} & \h{$\gSU(2)$} & \h{$\gU(1)$}
                                              & \h{$B$} & \h{$L$} & \multicolumn{3}{c}{~~~~scalar/spinor~~~~}\\\midrule
 $Q_i$     &$\THREE$   & $\TWO$    & $#1/6$   & $#1/3$  &         & $\tqL$     & $\qL$ & [$\to (\uL,\dL)$] \\
 $L_i$     &           & $\TWO$    & $-1/2$   &         & $#1$    & $\tlL$     & $\lL$ & [$\to (\nuL, \lL)$]\\
 $U_i^\cc$ &$\THREEbar$&           & $-2/3$   & $-1/3$  &         & $\tuR^\cc$ & $\uR^\cc$ \\
 $D_i^\cc$ &$\THREEbar$&           & $#1/3$   & $-1/3$  &         & $\tdR^\cc$ & $\dR^\cc$ \\
 $E_i^\cc$ &           &           & $ #1 $   &         & $-1$    & $\teR^\cc$ & $\eR^\cc$ \\
 $\Hu$     &           & $\TWO$    & $#1/2$   &         &         & $\hu$      & $\thu$ & [$\to (\hup, \huz)$] \\
 $\Hd$     &           & $\TWO$    & $-1/2$   &         &         & $\hd$      & $\thd$ & [$\to (\hdz, \hdm)$]\\\bottomrule
\end{tabular}
\end{subtable}
\begin{subtable}[t]{0.36\textwidth}
\caption{Vector superfields}\centering
\begin{tabular}[t]{c|ccc|c}\toprule
 \h{}      & \h{$\gSU(3)$} & \h{$\gSU(2)$} & \h{$\gU(1)$} & \h{ino/boson} \\\midrule
 $g$ & adj.      &           &          & $\tig, g_\mu$ \\
 $W$ &           &  adj.     &          & $\tiw, W_\mu$ \\
 $B$ &           &           &          & $\tib, B_\mu$ \\\bottomrule
\end{tabular}
\end{subtable}
\end{table}

 \noindent
Here, each of the column groups shows (from left to right) superfield name, charges for the gauge symmetries, other quantum numbers if relevant, and notation for corresponding fields (and $\gSU(2)$ decomposition).


\paragraph{``c''-notation}
For scalars, $\tilde\phi\w R^\cc:=\phi\w R^*=\opC \phi\w R\opC$ (because the intrinsic phase for $\mathsf C$ is $+1$ for quarks and leptons.)

For matter spinors, $\psi\w R^\cc:=\bar\psi\w R$ (and $\psi\w R=\bar\psi\w R^\cc$); Dirac spinors are thus
\begin{equation}
\displaystyle \psi\w L =\pmat{\psi\w L\\0},\quad
\displaystyle \overline{\psi\w L} = \pmat{0&\bar\psi\w L},\quad
 \psi\w R^\cc :=\pmat{\psi\w R^\cc \\ 0}=C\pmat{0\\\psi\w R} = C\psi\w R,
\quad
 \overline{\psi\w R^\cc} = \pmat{0 & \psi\w R} = \pmat{\bar\psi\w R & 0}C  = \overline{\psi\w R} C.\notag
\end{equation}

\paragraph{Superpotential and \cancel{SUSY}-terms}
\begin{align}
 W\w{RPC} &= \mu\Hu\Hd
           - \yu[ij] U^\cc_i\Hu Q_j
           + \yd[ij] D^\cc_i\Hd Q_j
           + \ye[ij] E^\cc_i\Hd L_j,\\
 W\w{RPV} &= -\kappa_i L_i\Hu 
           + \frac12 \lambda  _{ijk} L_i L_j E^\cc_k
           +         \lambda' _{ijk} L_i Q_j D^\cc_k
           + \frac12 \lambda''_{ijk} U^\cc_i D^\cc_j D^\cc_k,
\\
\mathcal L_{\text{\cancel{SUSY}}}&=
- \frac12\left(M_3\tig\tig+M_2\tiw\tiw+M_1\tib\tib+\text{H.c.}\right)
-V_{\text{\cancel{SUSY}}};\\
V^{\mathrm{RPC}}_{\text{\cancel{SUSY}}}&=
\left(\tqL^*m_Q^2\tqL+\tlL^*m_L^2\tlL+\tuR^* m_{U^\cc}^2\tuR + \tdR^* m_{D^\cc}^2\tdR + \teR^* m_{E^\cc}^2\teR
       +m^2_{\Hu}|\hu|^2 + m^2_{\Hd}|\hd|^2\right)
\notag\\&\quad
       +\left(-\tuR^*\hu {a\w u}\tqL + \tdR^* \hd {a\w d} \tqL+ \teR^* \hd {a\w e} \tlL + b\Hu\Hd+\text{H.c.}\right)
\notag\\&\quad
       +\left(-\tuR^*\hd^*{c\w u}\tqL+\tdR^* \hu^*{c\w d}\tqL+\teR^*\hu^*{c\w e}\tlL+\text{H.c.}\right),
\\
V^{\mathrm{RPV}}_{\text{\cancel{SUSY}}}&=
    \left(-b_i\tlL[i]\Hu +\frac12T_{ijk}\tlL[i]\tlL[j]\teR[k]^*+T'_{ijk}\tlL[i]\tqL[j]\tdR[k]^*
             +\frac12T''_{ijk}\tuR[i]^*\tdR[j]^*\tdR[k]^*
    +\tlL[i]^*M^2_{Li}\Hd+\text{H.c.}\right)
\notag\\&\qquad
   + \left(C^1_{ijk}\tlL[i]^*\tqL[j]\tuR[k]^* + C^2_i\hu^*\hd\teR[i]^* + C^3_{ijk}\tdR[i]\tuR[j]^*\teR[k]^*+\frac12C^4_{ijk}\tdR[i]\tqL[j]\tqL[k]+\text{H.c.}\right),
\end{align}
($\lambda_{ijk}=-\lambda_{jik}$, $\lambda''_{ijk}=-\lambda''_{ikj}$, and 
$C^4_{ijk}=C^4_{ikj}$.)



\clearpage

\detailstyle

We follow the notation of DHM~\cite[PhysRept]{0812.1594} and Martin~\cite[v7]{hep-ph9709356} (but note that Martin uses $(-,+,+,+)$-metric) for RPC part and SLHA2 convention for RPV part.

%\subsection{Lagrangian construction}
%As in Ref.~\cite{1712.05926},
%\begin{align}
% \mathcal L = \frac12\int\!\dd^4\theta\,
%\left[K(\ee^{2g V}\Phi,\Phi^\dagger) + K(\Phi,\Phi^\dagger\ee^{2g V})\right]
%+\left[\int\!\dd^2\theta\left( W(\Phi) + \sum\w{gauges}\frac{f_{ab}(\Phi)}{4} \mathcal W^{\alpha a}\mathcal W^b_\alpha
%\right)+\text{H.c.}\right]
%\end{align}


\subsection{Scalar potential}
The MSSM scalar potential has contributions from $F$-terms and $D$-terms:
\begin{alignat}{3}
 V\w{SUSY} &= F^{*}_i F_i+\frac12 D^aD^a;
\qquad&
 F_i &= -W_i^* = -\frac{\delta W^*}{\delta \phi_i^*},
\qquad&
 D^a &= -g(\phi^* T^a \phi),
\end{alignat}
where $T_a$ corresponds to the gauge-symmetry generator relevant for each $\phi$.
They are given by
\begin{align}
\input{calculator/mssm/Fterms.txt}
\\
D\w{SU(3)}^\alpha &= -g_3\sum_{i=1}^3\left(
 \sum_{a=1,2}
  \tqL[i]^{a*} \tau^\alpha \tqL[i]^{a}
- \tuR[i]^{*} \tau^\alpha \tuR[i]
- \tdR[i]^{*} \tau^\alpha \tdR[i]
\right),\\
%
D\w{SU(2)}^\alpha &= -g_2\left[
   \sum_{i=1}^3\left(
  \sum_{x=1}^3 \tqL[i]^{x*}T^\alpha\tqL[i]^{x}
+              \tlL[i]^{*}T^\alpha \tlL[i]
  \right)
+ \hu^*T^\alpha\hu
+ \hd^*T^\alpha\hd
\right],\\
%
 D\w{U(1)} &= -g_1\left(
  \frac16|\tqL|^2
- \frac12|\tlL|^2
- \frac23|\tuR|^2
+ \frac13|\tdR|^2
+        |\teR|^2
+ \frac12|\hu|^2
- \frac12|\hd|^2
\right).
\end{align}
Combining these, the full SUSY scalar potential is given by
\begin{align}
  V &= V\w{SUSY} + V_{\cancel{\mathrm{SUSY}}};\\
  \begin{split}
   V\w{SUSY} &= \input{calculator/mssm/SUSYPotential.txt}.
  \end{split}
\end{align}

\newpage
\begin{wraptable}{r}{200pt}\vspace{-3em}
 \begin{tabular}[t]{c@{\,}c@{\,}c@{\,}c@{\,}c}\toprule
 SLHA  && our notation && Martin/DHM\\\midrule
 $\col (H_1, H_2)$              &$=$& $(\Hd, \Hu)$ \\\midrule
 $\col Y_{\mathrm{u,d,e}}$      &$=$& $(y_{\mathrm{u,d,e}})^\TT$\\
 $\col T_{\mathrm{u,d,e}}$      &$=$& $(a_{\mathrm{u,d,e}})^\TT$\\
 $\col A_{\mathrm{u,d,e}}$      &$=$& $(A_{\mathrm{u,d,e}})^\TT$\\
 $\col m^2_{U^\cc,D^\cc, E\cc}$ &$=$& $(m^2_{U^\cc,D^\cc, E^\cc})^\dagger$\\
 $\col M_{1,2,3}$               &$=$& $-M_{1,2,3}$\\
 $\col m_3^2$                   &$=$& $b$\\
 $\col m_A^2$                   &$=$& $m^2_{A_0}$ (tree)\\\midrule
                           &&$\kappa_i$ &$=$& $\col -\mu_i'$ \footnotesize{(rarely used)}\\
 $\col D_i$                     &$=$& $b_i$\\
 $\col m^2_{\tilde L_iH_1}$     &$=$& $M^2_{Li}$ \\\bottomrule
 \end{tabular}
\end{wraptable}

\subsection{SLHA convention}
The SLHA convention \cite{SLHA} is different from our notation; the reinterpretation rules for the MSSM parameters are given in the right table  (\textbf{\col magenta color} for objects in other conventions), while
\begin{equation*}
  \mu, b, m^2_{Q, L, \Hu, \Hd}, \text{RPV-trilinears ($\lambda$s and $T$s)}
\end{equation*}
 are in common.



\clearpage

In particular, the chargino/neutralino mass terms in RPC case are given by
\begin{align}
 \mathcal L\supset
&\left[
  \frac12{\col M_1}\tib\tib
+ \frac12{\col M_2}\tiw\tiw
- \mu\thu\thd
- \frac{g_Y}{2\sqrt2} \left(\hu^*\thu-\hd^*\thd\right)\tib
- \sqrt{2}{g_2} \left(\hu^*T^a\thu+\hd^*T_a\thd\right)\tiw
\right]+\text{H.c.}
\\
&\to
\frac{1}{2}
\begin{pmatrix}\tib \\ \tiw \\ \huz \\ \hdz\end{pmatrix}^\TT
\begin{pmatrix}
 -M_1 & 0 & -m_Z \co{\beta} \si{w} & m_Z \si{\beta} \si{w} \\
 0 & -M_ 2 & m_Z \co{\beta} \co{w} & -m_Z \si{\beta} \co{w} \\
 -m_Z \co{\beta} \si{w} & m_Z \co{\beta} \co{w} & 0 & -\mu  \\
 m_Z \si{\beta} \si{w} & -m_Z \si{\beta} \co{w} & -\mu  & 0
\end{pmatrix}
\begin{pmatrix}\tib \\ \tiw \\ \huz \\ \hdz\end{pmatrix}
\end{align}

\end{document}


