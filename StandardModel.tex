%#!latexmk CheatSheet.tex
%%% Time-Stamp: <2010-11-19 20:45:27 misho>
%%% $BJ8;z%3!<%I$O(BShift-JIS$B$G$9$h$M(B

\section{$BML(B-Mills Theory}
\subsection{$\gU(1)$ Theory}

\subsubsection{General $\gSU(N)$}
\Paragraph{Gauge Transformation in General Gauge Group}
For any Lie group $G$, we can consider ``gauge transformation'' $\phi(x) \mapsto V(x)\phi(x)$, where $V:\Real^{1,3}\to G$. Also we can define a ``connection field'' $A_\mu(x)$ as:
\begin{align}
 \phi_\parallel (x+\dd x):= \phi(x)+\ii g A_\mu(x)\phi(x)\dd x^\mu
\qquad\text{s.t.}\quad \phi_\parallel(x+\dd x)\mapsto V(x+\dd x)\phi_\parallel(x+\dd x).
\end{align}
Then the covariant derivative $\Dm$ can be defined as $\Dm = \Pm - \ii g A_\mu$. Here we can see immediately
\begin{align}
 \phi&\mapsto V\phi,&
  A_\mu&\mapsto V\left(A_\mu + \frac\ii g \partial_\mu\right)V^{-1},&
 \Dm&\mapsto V\Dm V^{-1}.\label{eq:gaugetransf}
\end{align}
This discussion can be turned around: we can define the covariant derivative, a kind of derivative which satisfy \eqref{eq:gaugetransf}, as $\Dm=:\Pm-\ii g A_\mu$.

Next we introduce the curvature tensor, or ``field strength'' as
\begin{align}
& \Delta\phi(x):=\phi_{\parallel}^{x,y}(x+\dd x+\dd y)-\phi_{\parallel}^{y,x}(x+\dd x+\dd y)
= \left[\Dm,\Dn\right]\phi(x)\dd x^\mu\dd y^\nu
=: -\ii g F_{\mu\nu}\phi(x)\dd x^\mu\dd y^\nu;\\
&F_{\mu\nu}(x):=\frac{\ii}g\left[\Dm,\Dn\right]
 = \Pm A_\nu(x)-\Pn A_\mu(x)-\ii g\left[A_\mu(x),A_\nu(x)\right].
\end{align}
$\Delta\phi(x)$ is transformed in terms of $V(x+\dd x+\dd y)\simeq V(x)$, thus $F_{\mu\nu}(x)\mapsto V(x)F_{\mu\nu}(x)V^{-1}(x)$.

\Paragraph{Gauge Transformation in Compact Gauge Group}
If the gauge group $G$ is {\bf compact}, it has a finite-dimensional
unitary representation. The generators $T_a$ can be taken to be Hermitian, and $V(x) = \exp\left[\ii g\theta^a(x) T^a\right]$ for $\theta^a(x)\in\Real$;
\begin{align}
 [T^a,T^b] &= \ii f\T^a^b_c T^c\quad(f\in\Real)&
 0 &=f\T^D_a_b f\T^E_D_c + f\T^D_c_a f\T^E_D_b  + f\T^D_b_c f\T^E_D_a
\end{align}
For the sake of the compactness Killing form is positive-definite, where we can normalize the generators as $\Tr T^aT^b=\frac12\delta^{ab}$, and the structure constant $f^{abc}$ would be totally antisymmetric.
Then we can define ``adjoint representation'' as
\begin{align}
[T^{{\rm ad}\ a}]\T_i^j&:=-\ii f^{aij};&
[\Dm\suprm{ad}]\T_i^j&:=\delta_i^j\Pm+gf^{iaj}A^a_\mu.
\end{align}

In this normalized Hermitian generator basis, we can write down the relations as
\begin{align*}
\Dm          &:=\Pm-\ii g A^a_\mu T^a,&
 \phi'&=\ee^{\ii gT^a\theta^a}\phi;\\
F_{\mu\nu}^a&= \Pm A_\nu^a-\Pn A_\mu^a + gf^{abc}A_\mu^bA_\nu^c,&
A_\mu'&\simeq \left[A_\mu^a+\Pm\theta^a+gf^{abc}A_\mu^b\theta^c\right]T^a
         =A_\mu+\Dm\suprm{ad}\theta\\
&&F^a_{\mu\nu}{}'&\simeq F_{\mu\nu}^a+gf^{abc}F^b_{\mu\nu}\theta^c
\end{align*}
Note that $A_\mu$ is spanned by $T^a$ because it is induced by the gauge transformation, and so $F_{\mu\nu}$ is. This expansion is justified under any representation.

%\begin{align}
% F_{\mu\nu}&=F^a_{\mu\nu}T^a_{ij};\\
%F^a_{\mu\nu}T^a_{ij}&\mapsto F^a\ee^{\ii g \theta^bT^b}{}_{il}
%T^a_{lm}\ee^{-\ii g \theta^bT^b}{}_{mj}\\
%&=F^aT^a_{ij}+\ii g \theta^b[T^b,T^a]_{ij}+...\\
%&=F^aT^a_{ij}+\ii g \theta^bf^{bac}T^c_{ij}+...\\
%\end{align}
%
%It is useful to expand $A_\mu$ and $F_{\mu\nu}$ in adjoint representation.
\TODO{$B$3$3$^$G(B}
%
%$F^a[\tilde T^a]_{ij}\mapsto F^a \ee^{\ii g \theta}T^a\ee^{-\ii g\theta}=F^a\ee^{\ii g\theta^b[T^b,%}T^a$
%
%$[T^b,T^a]_{ij}=f^{abc}f^{cij}=$
%\newpage
Note that we have no charge-freedom.
 if we assign charge $q$, gauge transformtaion of $A_\mu^a$ would be $\frac1g\Pm\alpha^a + qf^{abc}A_\mu^b\alpha^c$.



\section{Standard Model}
We use the following notations for the gauge group.
\paragraph{Representation}\mbox{}\par\vskip-3.5zw
\begin{align*}
 \gSU(3)&:\quad G_\mu=G_\mu^a \tau^a\ ;\quad
\left[\tau^a,\tau^b\right]=\ii f^{abc}\tau^c,\quad
\Tr\left(\tau^a\tau^b\right)=\frac12\delta^{ab},\\
\gSU(2)&:\quad W_\mu=W_\mu^a T^a\ ;\quad
\left[T^a,T^b\right]=\ii \epsilon^{abc}T^c,\quad
\Tr\left(T^aT^b\right)=\frac12\delta^{ab},\\
%&\qquad T^{\pm} := T_1 \pm \ii T_2 = \left(\genfrac{}{0}{0}\right)\pmat{0&1\\0&0}, \pmat{0&0\\1&0},\\
\gU(1)&:\quad B_\mu
\end{align*}
\vskip-1zw
\paragraph{Field Strength}\mbox{}\par\vskip-3.5zw
\begin{align*}
F_{\mu\nu}
  &:=\Pm A_\nu-\Pn A_\mu-\ii g[A_\mu,A_\nu]\\
  & = \left(\Pm A^a_\nu- \Pn A^a_\mu + gf^{abc}A^b_\mu A^c_\nu\right)T^a
\end{align*}
\paragraph{Abridged Notation}
\begin{align*}
  (\partial A)_{\mu\nu}&:=\partial_\mu A_\nu-\partial_\nu A_\mu,&
  F_{\mu\nu}^a &:= \Pm A^a_\nu- \Pn A^a_\mu + gf^{abc}A^b_\mu A^c_\nu
\end{align*}
\subsection{Symmetries and Fields}
\paragraph{Gauge Group}
\begin{equation}
 \gSU(3)\stx{strong}\times\gSU(2)\stx{weak}\times\gU(1)_Y
\end{equation}
\paragraph{Field Content}
\begin{center}
 \begin{tabular}[b]{@{\Vrule\ }c@{\ }l|c|c|c@{\ \Vrule}}\Hrule
   & & $\gSU(3)\s{strong}$ & $\gSU(2)\s{weak} $ & $\gU(1)_Y$ \\\Hline
&\multicolumn{4}{l@{\Vrule}}{{\bf Matter Fields} (Fermionic / Lorentz Spinor)}\\\hline
 $\PL Q_i$ &: Left-handed quarks              & $\vc 3$ & $\vc 2$ & $1/6$\\\hline
 $\PL U_i$ &: Right-handed up-type quarks     & $\vc 3$ & $\vc 1$ & $2/3$\\\hline
 $\PR D_i$ &: Right-handed down-type quarks   & $\vc 3$ & $\vc 1$ & $-1/3$\\\hline
 $\PR L_i$ &: Left-handed leptons             & $\vc 1$ & $\vc 2$ & $-1/2$\\\hline
 $\PR E_i$ &: Right-handed leptons            & $\vc 1$ & $\vc 1$ & $-1$\\\Hline
&\multicolumn{4}{l@{\Vrule}}{{\bf Higgs Field} (Bosonic / Lorentz Scalar)}\\\hline
 $H$   &: Higgs                           & $\vc 1$ & $\vc 2$ & $1/2$\\\Hline
&\multicolumn{4}{l@{\Vrule}}{{\bf Gauge Fields} (Bosonic / Lorentz Vector)}\\\hline
 $G$   &: Gluons                          & $\vc 8$ & $\vc 1$ & $0$\\\hline
 $W$   &: Weak bosons                     & $\vc 1$ & $\vc 3$ & $0$\\\hline
 $B$   &: B boson                         & $\vc 1$ & $\vc 1$ & $0$\\\Hline
\end{tabular}
\end{center}
\paragraph{Full Lagrangian}
\begin{align}
 \Lag = \Lag\s{gauge}
       + \Lag\s{Higgs}&
       + \Lag\s{matter}
       + \Lag\s{$BEr@n(B},\\
\text{where\quad}
%%% Gauge Boson Kinetic
\Lag\s{gauge} =&\
 -\frac14B^{\mu\nu}B_{\mu\nu}
 -\frac14W^{a\mu\nu}W^a_{\mu\nu}
 -\frac14G^{a\mu\nu}G^a_{\mu\nu}\\
%%% Higgs
\Lag\s{Higgs} =&\
 \left|\left(\partial_\mu-\ii g_2W_\mu-\frac12\ii g_1B_\mu\right)H\right|^2
 - V(H),\\
%%% MATTERS
\begin{split}
\Lag\s{matter} =&\
% Left Quark
 \ol Q_i\ii\gamma^\mu\left(\partial_\mu-\ii g_3G_\mu-\ii
 g_2W_\mu-\frac16\ii g_1B_\mu\right)\PL Q_i\\
& % Up Quark
 + \ol U_i\ii\gamma^\mu\left(\partial_\mu-\ii g_3G_\mu-\frac23\ii g_1B_\mu\right)\PR U_i\\
& % Down Quark
 + \ol D_i\ii\gamma^\mu\left(\partial_\mu-\ii g_3G_\mu+\frac13\ii g_1B_\mu\right)\PR D_i\\
& % Left Lepton
 + \ol L_i\ii\gamma^\mu\left(\partial_\mu-\ii g_2W_\mu+\frac12\ii g_1B_\mu\right)\PL L_i\\
& % Right Electron
 + \ol E_i\ii\gamma^\mu\left(\partial_\mu+\ii g_1B_\mu\right)\PR E_i,
\end{split}\\
%%% YUKAWA
\Lag\s{$BEr@n(B} =&\
 - \ol U_i(y_u)_{ij}H\PL Q_j + \ol D_i (y_d)_{ij}H^\dagger\PL Q_j + \ol E_i(y_e)_{ij}H^\dagger\PL
 L_j + \Hc
\end{align}
We have no freedom to add other terms into this Lagrangian of the gauge theory. See Appendix~\ref{sec:no-other-term}.

\Paragraph{Gauge Kinetic Terms}
the gauge kinetic terms can be expanded as
\begin{align}
\Lag\s{gauge}
=& -\frac14(\partial B)(\partial B)\notag\\
& -\frac14(\partial W^a)(\partial W^a)
-g_2\epsilon^{abc}(\partial_\mu W_\nu^a)W^{\mu b}W^{\nu c}
-\frac{{g_2}^2}4
 \left(\epsilon^{eab}W_\mu^aW_\nu^b\right)\left(\epsilon^{ecd}W^{c\mu}W^{d\nu}\right)\\
&-\frac14(\partial G^a)(\partial G^a)
-g_3f^{abc}(\partial_\mu G_\nu^a)G^{\mu b}G^{\nu c}
-\frac{{g_3}^2}4
 \left(f^{eab}G_\mu^aG_\nu^b\right)\left(f^{ecd}G^{c\mu}G^{d\nu}\right).\notag
\end{align}

\subsection{Higgs Mechanism}
\Paragraph{Higgs Potential}
\label{sec:higgs-mechanism}
The (renormalizable) Higgs potential must be
\begin{equation}
 V(H) = -\mu^2(H^\dagger H) + \lambda\left(H^\dagger H\right)^2.
\end{equation}
for the $\gSU(2)$, and $\lambda>0$ in order not to run away the VEVs, while $\mu^2$ is positive for the EWSB.

To discuss this clearly, let us {\em redefine} the Higgs field as
\begin{equation}
 H = \frac1{\sqrt2}\pmat{\phi_1+\ii\phi_2\\v+(h+\ii\phi_3)}, \where v=\sqrt{\frac{\mu^2}{\lambda}}.
\end{equation}
Here $h$ is the ``Higgs boson,'' and $\phi_i$ are $BFnIt(B--Goldstone bosons.

The Higgs potential becomes
\begin{equation}
   V(h) = \frac{\mu^2}{4v^2}h^4 + \frac{\mu^2}v h^3 + \mu^2h^2,
\end{equation}
and now we know the Higgs boson has acquired mass $m_h=\sqrt2 \mu$. Also
\begin{align}
 \Lag\stx{Higgs}&=
   \left|\left(\partial_\mu-\ii g_2W_\mu-\frac12\ii g_1B_\mu\right)H\right|^2\\
&= \frac12(\partial_\mu h)^2+\frac{(v+h)^2}8\Bigl[{g_2}^2{W_1}^2+{g_2}^2{W_2}^2+(g_1B-g_2W_3)^2\Bigr].
\end{align}
Redefining the gauge fields (with concerning the norms) as
\begin{align}
 W^\pm_\mu&:=\frac1{\sqrt2}(W^1_\mu\mp\ii W^2_\mu),&
\pmat{Z_\mu \\ A_\mu}
&:= \pmat{%
\cos\theta\s w & -\sin\theta\s w\\
\sin\theta\s w & \cos\theta\s w
}
\pmat{W^3_\mu \\ B_\mu},
\end{align}
where
\begin{align}
 &\tan\theta\s w := \frac{g_1}{g_2},\qquad
 e              := -\frac{g_1g_2}{\sqrt{{g_1}^2+{g_2}^2}};\qquad
 g_Z            := \sqrt{{g_1}^2+{g_2}^2};\\
 &g_1=\frac{|e|}{\cos\theta\s w}=g_Z\sin\theta\s w,\qquad
  g_2=\frac{|e|}{\sin\theta\s w}=g_Z\cos\theta\s w.
\end{align}
We obtain the following terms in $\Lag\s{Higgs}$:
\begin{equation}
  \Lag\s{Higgs}
\supset \frac12(\partial_\mu h)^2+\frac{(v+h)^2}4\Bigl[{g_2}^2{W^+}^\mu W^-_\mu + \frac{{g_Z}^2}{2}Z^\mu Z_\mu\Bigr].
\end{equation}
Here we have omitted the $BFnIt(B--Goldstone bosons.


\begin{rightnote}
Here we present another form:
\begin{eqnarray}
  g_1B_\mu
&=& |e| A_\mu-\tan\theta\s w Z_\mu,\\
 g_2W_\mu
&=& \frac{g_2}{\sqrt2}\left(W^+_\mu T^+ + W^-_\mu T^-\right)
   +\left(\frac{|e|}{\tan\theta\s w}Z_\mu+|e|A_\mu\right)T^3,
\end{eqnarray}
\begin{equation}
Z^0_\mu:=\frac1{\sqrt{{g_1}^2+{g_2}^2}}(g_2 W^3_\mu-g_1B_\mu),\quad
A_\mu:=\frac1{\sqrt{{g_1}^2+{g_2}^2}}(g_1 W^3_\mu+g_2B_\mu)
\end{equation}
\end{rightnote}

You can see the gauge bosons have acquired the masses
\begin{equation}
  m_A = 0, \quad m_W :=\frac{g_2}2v,\quad m_Z :=\frac{g_Z}2v.
\end{equation}


\paragraph{Gauge Term}
The $\gSU(2)$ gauge term is converted into
\begin{align*}
   W^{a\mu\nu}W^a_{\mu\nu}
&= (\partial W^3)(\partial W^3) + 2(\partial W^+)(\partial W^-)
\\&\quad
- 4\ii g\left[
     (\partial W^3)^{\mu\nu}W^+_\mu W^-_\nu
   + (\partial W^+)^{\mu\nu}W^-_\mu W^3_\nu
   + (\partial W^-)^{\mu\nu}W^3_\mu W^+_\nu
\right]
\\&\quad
 -2g^2(\Hmn\Hrs-\Hmr\Hns)
\left(
  W^+_\mu W^+_\nu W^-_\rho W^-_\sigma  - 2 W^3_\mu W^3_\nu W^+_\rho W^-_\sigma
\right),
\end{align*}
and therefore the final expression is
\begin{equation}
\begin{split}
 \Lag\s{gauge}&:=
-\frac14\left[
G^{a\mu\nu}G^a_{\mu\nu} + (\partial Z)^{\mu\nu}(\partial Z)_{\mu\nu}+(\partial A)^{\mu\nu}(\partial A)_{\mu\nu}+2(\partial W^+)^{\mu\nu}(\partial W^-)_{\mu\nu}
\right]\\&\quad
+\frac{\ii |e|}{\tan\theta\s w}\Bigl[
(\partial W^+)^{\mu\nu}W^-_\mu Z_\nu + (\partial W^-)^{\mu\nu}Z_\mu W^+_\nu +  (\partial Z)^{\mu\nu}W^+_\mu W^-_\nu
\Bigr]\\&\quad
+ \ii |e|\Bigl[
(\partial W^+)^{\mu\nu}W^-_\mu A_\nu + (\partial W^-)^{\mu\nu}A_\mu W^+_\nu +  (\partial A)^{\mu\nu}W^+_\mu W^-_\nu
\Bigr]\\&\quad
+(\Hmn\Hrs-\Hmr\Hns)\left[
\frac{|e|^2}{2\sin^2\theta\s w}W^+_\mu W^+_\nu W^-_\rho W^-_\sigma
+\frac{|e|^2}{\tan^2\theta\s w}W^+_\mu Z_\nu W^-_\rho Z_\sigma
\right.\\&\left.\qquad\qquad\qquad
+\frac{|e|^2}{\tan\theta\s w}\left(W^+_\mu Z_\nu W^-_\rho A_\sigma + W^+_\mu A_\nu W^-_\rho Z_\sigma\right)
+|e|^2W^+_\mu A_\nu W^-_\rho A_\sigma\right].
\end{split}
\end{equation}

\paragraph{$BEr@n(B Term}
\begin{align}
 \Lag\s{$BEr@n(B}=
&-\ol U y_uH\PL Q + \ol D y_dH^\dagger\PL Q + \ol E y_eH^\dagger\PL L
 + \Hc\notag\\
=
&- \ol U y_u\epsilon^{\alpha\beta}H^{\alpha}\PL Q^{\beta}
 + \ol D y_d{H^\dagger}^{\alpha}\PL Q^\alpha
 + \ol E y_e{H^\dagger}^{\alpha}\PL L^\alpha + \Hc\notag\\
=
&\frac{v+h}{\sqrt2}\left(
   \ol U y_u\PL Q^1
 + \ol D y_d\PL Q^2
 + \ol E y_e\PL L^2
\right) + \Hc
\end{align}
\subsection{Full Lagrangian After Higgs Mechanism}
Now we have the following Lagrangian (with omitting $\PL$ etc.):
\begin{align}
 \Lag =
& \Lag\s{gauge}
{\ +\ }{m_W}^2W^+W^-
{\ +\ }\frac{{m_Z}^2}2Z^2\notag\\[.5zw]
%Higgs
\note{Higgs}&
{\ +\ }\frac12(\partial_\mu h)^2
{\ -}\frac12\mu^2h^2
{-}\sqrt{\frac\lambda2}m_h h^3
{-}\frac14\lambda h^4
\notag\\&
{\ +\ }\frac{v{g_2}^2}{4}W^+W^-h
{\ +\ }\frac{v({g_1}^2+{g_2}^2)}{8}Z^2h\notag\\&
{\ +\ }\frac{{g_2}^2}{4}W^+W^-h^2
{\ +\ }\frac{{g_1}^2+{g_2}^2}{8}Z^2h^2\notag\\
&
{\ +\ }\frac{1}{\sqrt2}h\bar U y_u Q^1
{\ +\ }\frac{1}{\sqrt2}h\bar D y_d Q^2
{\ +\ }\frac{1}{\sqrt2}h\bar E y_e L^2\  +\  \Hc\notag\\[1zw]
%SU(3) Interactions
\note{$\gSU(3)$}&
{\ +\ }\bar Q\left(\ii\slashed\partial+g_3\slashed G\right) Q
{\ +\ }\bar U\left(\ii\slashed\partial+g_3\slashed G\right) U
{\ +\ }\bar D\left(\ii\slashed\partial+g_3\slashed G\right) D
{\ +\ }\bar L\left(\ii\slashed\partial\right) L
{\ +\ }\bar E\left(\ii\slashed\partial\right) E\notag\\[1zw]
\note{$W$}&
{\ +\ }\bar Q\frac{g_2}{\sqrt2}\left(\slashed W^+ T^+ + \slashed W^- T^-\right)Q
{\ +\ }\bar L\frac{g_2}{\sqrt2}\left(\slashed W^+ T^+ + \slashed W^- T^-\right)L\notag\\
\note{$A$\&$Z^0$}&
{\ +\ }\bar Q\left[
    \left(T^3+\frac16\right)|e|\slashed A
   +\left(\frac{|e|c}{s}T^3-\frac{|e|s}{6c}\right)\slashed Z^0
  \right] Q\notag\\
&{\ +\ }\bar U\left(\frac23 |e|\slashed A-\frac{2|e|s}{3c}\slashed Z\right) U\notag\\
&{\ +\ }\bar D\left(-\frac13 |e|\slashed A+\frac{|e|s}{3c}\slashed Z\right) D\notag\\
&{\ +\ }\bar L\left[
    \left(T^3-\frac12\right)|e|\slashed A
   +\left(\frac{|e|c}{s}T^3+\frac{|e|s}{2c}\right)\slashed Z^0
  \right] L\notag\\
&
{\ +\ }\bar E\left(-|e|\slashed A+\frac{|e|s}{c}\slashed Z\right) E\notag\\[.5zw]
\note{$BEr@n9`(B}& % Yukawa
{\ +\ }\frac{1}{\sqrt2}v\bar U y_u Q^1
{\ +\ }\frac{1}{\sqrt2}v\bar D y_d Q^2
{\ +\ }\frac{1}{\sqrt2}v\bar E y_e L^2\quad+\quad \Hc
\end{align}
\subsection{Mass Eigenstates}
Here we will obtain the mass eigenstates of the fermions, by diagonalizing the $BEr@n(B matrices.

We use the singular value decomposition method to mass matrices $Y_\bullet:=vy_\bullet/\sqrt2$.
Generally, any matrices can be transformed with two unitary matrices $\Psi$ and $\Phi$ as
\begin{equation}
 Y=\Phi^\dagger\pmat{m_1&0&0\\0&m_2&0\\0&0&m_3}\Psi =:\Phi^\dagger M\Psi\qquad(m_i\ge0).
\end{equation}
Using this $\Psi$ and $\Phi$, we can rotate the basis as
\begin{align}
 &Q^1\mapsto \Psi_u^\dagger Q^1,\quad
 Q^2\mapsto \Psi_d^\dagger Q^2,\quad
 L\mapsto \Psi_e^\dagger L,\quad
 &U\mapsto \Phi_u^\dagger U,\quad
 D\mapsto \Phi_d^\dagger D,\quad
 E\mapsto \Phi_e^\dagger E,\label{eq:gaugeeig_to_masseig}
\end{align}
and now we have the $BEr@n(B terms in mass eigenstates as
\begin{equation}
  \Lag\s{$BEr@n(B}
= \left(1+\frac{1}{v}h\right)\left[
   {(m_u)_i} \ol U_i \PL Q_i^1
 + {(m_d)_i} \ol D_i \PL Q_i^2
 + {(m_e)_i} \ol E_i \PL L_i^2 + \Hc\right].
\end{equation}

In the transformation from the gauge eigenstates to the mass eigenstates, almost all the terms in the Lagrangian are not modified.
However, only the terms of quark--quark--$W$ interactions do change drastically, as
\begin{align}
 \Lag
&\supset
 \ol Q\ii\gamma^\mu\left(-\ii g_2W_\mu-\frac16\ii g_1B_\mu\right)\PL Q\\
&=\ol Q\frac{g_2}{\sqrt2}\left(\slashed W^+ T^+ + \slashed W^- T^-\right)\PL Q
\quad+\quad(\text{interaction terms with $Z$ and $A$})\\
&\mapsto%
\frac{g_2}{\sqrt2}
\pmat{\ol Q^1\Psi_u&\ol Q^2\Psi_d}
\pmat{0&\slashed W^+\\\slashed W^-&0}\PL\pmat{\Psi_u^\dagger Q^1\\\Psi_d^\dagger Q^2} + (\quad{\cdots}\quad)\\
&=\frac{g_2}{\sqrt2}\left[\ol Q^2\slashed W^-X \PL Q^1+ \ol Q^1\slashed W^+X^\dagger\PL Q^2\right]+(\quad{\cdots}\quad),
\end{align}
where $X:=\Psi_d\Psi_u^\dagger$ is a matrix, so-called the Cabbibo--$B>.NS(B--$B1W@n(B~(CKM) matrix, which is {\em not} diagonal, and {\em not} real, generally.
These terms violate the flavor symmetry of quarks, and even the $CP$-symmetry.
\begin{rightnote}
In our notation, $CP$-transformation of a spinor is described as
\begin{equation}
 \mathscr{CP}\left(\psi\right)= -\ii\eta^*\trans{(\ol\psi\G2)},\quad
 \mathscr{CP}\left(\ol\psi\right)= \ii\eta\trans{(\G2\psi)},
\end{equation}
where $\eta$ is a complex phase ($|\eta|=1$).
Under this transformation, those terms are transformed as, e.g.,
\begin{eqnarray}
\mathscr{CP}\left( \ol Q^2\slashed W^-X \PL Q^1\right)
&=&
\trans{(\G2Q^2)} \mathscr{P}(-{\slashed W^+})X\PL\trans{(\ol Q^1\G2)}\notag\\
&=&-{W_\mu^+}^P\trans{(\G2Q^2)}\trans{(\ol Q^1\trans X\G2\PL\trans{\Gm })}\\
&=&(\ol Q^1{\slashed W^+}\trans X\PL Q^2).\notag
\end{eqnarray}
Therefore, we can see that the $CP$-symmetry is maintained if and only if $\trans X=X^\dagger$, that is, if and only if $X$ is a real matrix.
\end{rightnote}

$B0J>e$h$j!$I8=`LO7?$N(BLagrangian$B$O(B
\begin{align}
 \Lag =
& \Lag\s{gauge}\notag\\
\note{$B<ANL9`(B}& % Yukawa
{\quad+\quad}{m_W}^2W^+W^-
{\quad+\quad}\frac{{m_Z}^2}2Z^2
\notag\\&
{\quad+\quad}\bar U M_u \PL Q^1
{\quad+\quad}\bar D M_d \PL Q^2
{\quad+\quad}\bar E M_e \PL L^2\quad+\quad \Hc
\notag\\[.5zw]
%Higgs
\note{Higgs Field}&
{\quad+\quad}\frac12(\partial_\mu h)^2
{\quad-}\frac12\mu^2h^2
{-}\sqrt{\frac\lambda2}m_h h^3
{-}\frac14\lambda h^4\notag\\
\note{Higgs$B$H$N7k9g(B}&
{\quad+\quad}\frac{v{g_2}^2}{4}W^+W^-h
{\quad+\quad}\frac{v({g_1}^2+{g_2}^2)}{8}Z^2h\notag\\&
{\quad+\quad}\frac{{g_2}^2}{4}W^+W^-h^2
{\quad+\quad}\frac{{g_1}^2+{g_2}^2}{8}Z^2h^2\notag\\
&
{\quad+\quad}\frac1v\bar U M_u \PL Q^1h
{\quad+\quad}\frac1v\bar D M_d \PL Q^2h
{\quad+\quad}\frac1v\bar E M_e \PL L^2h\quad +\quad \Hc\notag\\[1zw]
%SU(3) Interactions
\note{$\gSU(3)$$B$*$h$SHyJ,9`(B}&
{\quad+\quad}\bar Q\left(\ii\slashed\partial+g_3\slashed G\right) \PL Q
{\quad+\quad}\bar U\left(\ii\slashed\partial+g_3\slashed G\right) \PR U
{\quad+\quad}\bar D\left(\ii\slashed\partial+g_3\slashed G\right) \PR D\notag\\
&
{\quad+\quad}\bar L\left(\ii\slashed\partial\right) \PL L
{\quad+\quad}\bar E\left(\ii\slashed\partial\right) \PR E\notag\\[1zw]
\note{$W$ boson}&
{\quad+\quad}\frac{g_2}{\sqrt2}\left[
\bar Q^2\slashed W^-X \PL Q^1+ \bar Q^1\slashed W^+X^\dagger \PL Q^2\right]
\qquad\note{$B"+(B $CP$ and flavor violating!}
\notag\\
&
{\quad+\quad}\bar L\frac{g_2}{\sqrt2}\left(\slashed W^+ T^+ + \slashed W^- T^-\right)\PL L\notag\\
\note{$A$\&$Z^0$ boson}&
{\quad+\quad}\bar Q\left[
    \left(T^3+\frac16\right)|e|\slashed A
   +\left(\frac{|e|c}{s}T^3-\frac{|e|s}{6c}\right)\slashed Z^0
  \right]\PL  Q\notag\\
&{\quad+\quad}\bar U\left(\frac23 |e|\slashed A-\frac{2|e|s}{3c}\slashed Z\right) \PR U\notag\\
&{\quad+\quad}\bar D\left(-\frac13 |e|\slashed A+\frac{|e|s}{3c}\slashed Z\right) \PR D\notag\\
&{\quad+\quad}\bar L\left[
    \left(T^3-\frac12\right)|e|\slashed A
   +\left(\frac{|e|c}{s}T^3+\frac{|e|s}{2c}\right)\slashed Z^0
  \right] \PL L\notag\\
& 
{\quad+\quad}\bar E\left(-|e|\slashed A+\frac{|e|s}{c}\slashed Z\right) \PR E
\end{align}
$B$H$J$k!#(B

\subsection{Chiral Notation}
$B0J>e$N(BLagrangian$B$r(Bchiral$BI=<($GI=$9$H!$$^$::G=i$O(B
\begin{align}
 \mathcal L
=& \text{(Higgs terms)} + \text{(Gauge fields strength)}\notag\\
& + Q\s L^\dagger\ii\bar\sigma^\mu\left(\partial_\mu-\ii g_3G_\mu-\ii
 g_2W_\mu-\frac16\ii g_1B_\mu\right)Q\s L\notag\\
& + U\s R^\dagger\ii\sigma^\mu\left(\partial_\mu-\ii g_3G_\mu-\frac23\ii g_1B_\mu\right)U\s R\notag\\
& + D\s R^\dagger\ii\sigma^\mu\left(\partial_\mu-\ii g_3G_\mu+\frac13\ii g_1B_\mu\right)D\s R\notag\\
& + L\s L^\dagger\ii\bar\sigma^\mu\left(\partial_\mu-\ii g_2W_\mu+\frac12\ii g_1B_\mu\right)L\s L\notag\\
& + E\s R^\dagger\ii\sigma^\mu\left(\partial_\mu+\ii g_1B_\mu\right)E\s R\notag\\
& % Yukawa
 + U^\dagger\s R y_uHQ\s L + D^\dagger\s R y_dH^\dagger Q\s L +
 E^\dagger \s R y_eH^\dagger L\s L + \Hc\notag\\
=& \text{(Higgs terms)} + \text{(Gauge fields strength)}\notag\\
&
+ \ii Q\s L^\dagger\bar\sigma^\mu\partial_\mu Q\s L
+ \ii U\s R        \bar\sigma^\mu\partial_\mu U\s R^\dagger
+ \ii D\s R        \bar\sigma^\mu\partial_\mu D\s R^\dagger
+ \ii L\s L^\dagger\bar\sigma^\mu\partial_\mu L\s L
+ \ii E\s R        \bar\sigma^\mu\partial_\mu E\s R^\dagger \notag\\
&
+ g_3\left(
 Q\s L^\dagger\bar\sigma^\mu G_\mu Q\s L
+U\s R^\dagger\bar\sigma^\mu G_\mu U\s R
+D\s R^\dagger\bar\sigma^\mu G_\mu D\s R
\right)\notag\\
&
+ g_2\left(
 Q\s L^\dagger\bar\sigma^\mu W_\mu Q\s L
+L\s L^\dagger\bar\sigma^\mu W_\mu L\s L
\right)\notag\\
&+ g_1\left(
 \frac16 Q\s L^\dagger\bar\sigma^\mu B_\mu Q\s L
+\frac23 U\s R^\dagger\bar\sigma^\mu B_\mu U\s R
-\frac13 D\s R^\dagger\bar\sigma^\mu B_\mu D\s R
-\frac12 L\s L^\dagger\bar\sigma^\mu B_\mu L\s L
-        E\s R^\dagger\bar\sigma^\mu B_\mu E\s R
\right)\notag\\
& % Yukawa
 + U^\dagger\s R y_uHQ\s L + D^\dagger\s R y_dH^\dagger Q\s L +
 E^\dagger \s R y_eH^\dagger L\s L + \Hc
\end{align}
$B$G$"$j!$$=$7$F:G=*E*$K$O(B
\begin{align}
 \Lag =
& \text{(Gauge bosons and Higgs)}\notag\\
&
+ \ii Q\s L^\dagger\bar\sigma^\mu\partial_\mu Q\s L
+ \ii U\s R        \bar\sigma^\mu\partial_\mu U\s R^\dagger
+ \ii D\s R        \bar\sigma^\mu\partial_\mu D\s R^\dagger
+ \ii L\s L^\dagger\bar\sigma^\mu\partial_\mu L\s L
+ \ii E\s R        \bar\sigma^\mu\partial_\mu E\s R^\dagger \notag\\
&
+ g_3\left(
 Q\s L^\dagger\bar\sigma^\mu G_\mu Q\s L
+U\s R^\dagger\bar\sigma^\mu G_\mu U\s R
+D\s R^\dagger\bar\sigma^\mu G_\mu D\s R
\right)\notag\\
&
+m_u(u\s R^\dagger u\s L+u\s L^\dagger u\s R)+\text{(quarks)}
+m_e(e\s R^\dagger e\s L+e\s L^\dagger e\s R)+\text{(leptons)}
\notag\\&
+\frac{m_u}v(u\s R^\dagger u\s L+u\s L^\dagger u\s R)h + \text{(quarks)}
+\frac{m_e}v(e\s R^\dagger e\s L+e\s L^\dagger e\s R)h + \text{(leptons)}
\notag\\
&+
\frac{g_2}{\sqrt2}\left[
\pmat{d\s L^\dagger\ s\s L^\dagger\ b\s L^\dagger}
\bar\sigma^\mu W^-_\mu X \pmat{u\s L\\c\s L\\t\s L}
+
\pmat{u\s L^\dagger\ c\s L^\dagger\ t\s L^\dagger}
\bar\sigma^\mu W^+_\mu X^\dagger \pmat{d\s L\\s\s L\\b\s L}
\right]
\notag\\&
+\frac{g_2}{\sqrt2}\left[
\nu_e^\dagger\bar\sigma^\mu W^+_\mu e\s L + e\s L^\dagger\bar\sigma^\mu W^-_\mu \nu_e
\right]\notag\\
&
+|e|\left[
  \frac23 u\s L^\dagger \bar\sigma^\mu A_\mu u\s L
- \frac13 d\s L^\dagger \bar\sigma^\mu A_\mu d\s L
+ \frac23 u\s R^\dagger     \sigma^\mu A_\mu u\s R
- \frac13 d\s R^\dagger     \sigma^\mu A_\mu d\s R
+ \text{(quarks)}
\right.\notag\\&\left.\qquad\qquad
- e\s L^\dagger \bar\sigma^\mu A_\mu e\s L
- e\s R^\dagger     \sigma^\mu A_\mu e\s R
+ \text{(leptons)}\right]\notag\\
&%
+\frac{|e|s}{c}\left[
  \left(\frac{c^2}{2s^2}-\frac16\right)u\s L^\dagger \bar\sigma^\mu Z_\mu u\s L
- \left(\frac{c^2}{2s^2}+\frac16\right)d\s L^\dagger \bar\sigma^\mu Z_\mu d\s L
- \frac23 u\s R^\dagger     \sigma^\mu Z_\mu u\s R
+ \frac13 d\s R^\dagger     \sigma^\mu Z_\mu d\s R
\right.\notag\\&\left.\qquad\qquad
+ \left(\frac{c^2}{2s^2}+\frac12\right)\nu_e^\dagger \bar\sigma^\mu Z_\mu \nu_e
- \left(\frac{c^2}{2s^2}-\frac12\right)e\s L^\dagger \bar\sigma^\mu Z_\mu e\s L
+ e\s R^\dagger     \sigma^\mu Z_\mu e\s R
+ \text{(others)}\right]
\end{align}
$B$H$J$k!#(B

\subsection{Values of SM Parameters}
\subsubsection{Experimental Values}
\subparagraph{Low energy values}
\begin{align*}
 \alpha\s{EM} &= 1/137.035999679(94) &
 G\s{F}       &= \frac{{g_2}^2}{4\sqrt2{m_W}^2} = \frac1{\sqrt2v^2} = 1.166367(5)\EE-5\un{GeV^{-2}}&
\end{align*}

\subparagraph{Electroweak scale}\note{These values are all in $\overline{\rm{MS}}$ scheme.}
\begin{align*}
 \alpha^{-1}\s{EM}(m_Z)    &= 127.925(16) & m_W(m_W) &= 80.398(25)\un{GeV}\\
 \alpha^{-1}\s{EM}(m_\tau) &= 133.452(16) & m_Z(m_Z) &= 91.1876(21)\un{GeV}\\
 \alpha\s{s}(m_Z)          &= 0.1176(20)  & \sin^2\theta\s W(m_Z) &= 0.23119(14)\\
 \Gamma_{l^+l^-}           &= 83.984(86)\un{MeV}  & \sin^2\theta\s{eff}   &= 0.23149(13)
\end{align*}

\subparagraph{Fundamental masses}
\begin{align*}
 e   &: 0.510998910(13)\un{MeV}         &
 u   &: 1.5 \text{ to } 3.3\un{MeV}     & d&: 3.5\text{ to }6.0\un{MeV}\\
\mu  &: 105.658367(4)\un{MeV}           &
 c   &: 1.27\pmgosa{0.07}{0.11}\un{GeV} & s&: 104\pmgosa{26}{34}\un{MeV}\\
\tau &: 1.77784(17)\un{GeV}             &
 t   &: 171.2\gosa{2.1}\un{GeV}         & b&: 4.20\pmgosa{0.17}{0.07}\un{GeV}
\end{align*}
\begin{align*}
 \pi^\pm &: 139.57018(35)\un{MeV}&
 K^\pm   &: 493.677(16)  \un{MeV}&
 p       &: 938.27203(8) \un{MeV}\\
 \pi^0   &: 139.766(6)   \un{MeV}&
 K^0     &: 497.614(24)  \un{MeV}&
 n       &: 939.56536(8) \un{MeV}
\end{align*}
\subparagraph{Fundamental Lifetime} (also $c\tau$ for some particles)
\begin{align*}
 \mu     &: 2.197019(21)\un{\mu s} \quad(658\un{m}) &
 \pi^\pm &: 2.6033(5)\EE-8\un{s}&
 K^\pm   &: 1.2380(21)\EE-8\un{s}\\
 \tau    &: 2.906(10)\EE{-13}\un{s} \quad (87\un{\mu m})&
 \pi^0   &: 8.4(6)\EE-{17}\un{s}&
 K^0\s{S}&: 8.953(5)\EE{-11}\un{s}\\
&&&&
 K^0\s{L}&: 5.116(20)\EE-8\un{s}
\end{align*}
\subparagraph{CKM matrix}
\begin{align}
 V\s{CKM}&=
\pmat{
0.97419(22) & 0.2257(10) & 0.00359(16)\\
0.2256(10)  & 0.97334(23)& 0.0415(11)\\
0.00874(37) & 0.0407(10) & 0.999133(44)}
\sim
\pmat{
1-\epsilon^2 & \epsilon & \epsilon^4\\
\epsilon     & 1-\epsilon^2 & \epsilon^2\\
\epsilon^3   & \epsilon^2 & 1-\epsilon^4}\quad \text{for $\epsilon\sim0.23$}
\end{align}


\subsubsection{Estimation of SM Parameters}
For EW scale, we can estimate the values as
\begin{align}
 e&\sim 0.313,& g_1 &\sim 0.358, & g_2 &\sim 0.651; &
  v=\sqrt{\frac{\mu^2}\lambda} &\sim 246\un{GeV}
\end{align}
Therefore $BEr@n(B matrices are (after diagonalization), since $vy/\sqrt2=M$,
\begin{align}
 y_u&\sim \pmat{10^{-5}&0&0\\0&0.007&0\\0&0&0.98},&
 y_d&\sim \pmat{3\times10^{-5}&0&0\\0&0.0006&0\\0&0&0.02},&
 y_e&\sim \pmat{3\times10^{-6}&0&0\\0&0.0006&0\\0&0&0.01}.&
\end{align}

Also, for $m_h\sim 120\un{GeV}$, we can estimate the Higgs potential as $\mu\sim85\un{GeV}$ and $\lambda \sim 0.12$.
%\newpage


\newpage







\section{Spinor}
\Paragraph{$\eta^{\mu\nu}=(-,+,+,+)$ case}
\begin{tabular}[t]{l@{\ :\ }l}
 Grassmann Number
& $(ab)^\dagger=b^\dagger a^\dagger$ for $a,b\in\Grassmann$\\
& \then for $a,b\in\RealGrassmann$, $ab\in\ii\RealGrassmann$\\
 $\gamma$ matrix
& $\displaystyle\{\Gm,\Gn\}=2\minkow\mu\nu\cdot\vc1$\\
& $\displaystyle\G{\mu\nu} = \frac12\left(\Gm\Gn-\Gn\Gm\right)$
  \quad etc...\\
& $\displaystyle(\ii\G0)^\dagger:=\ii\G0,\quad\G i^\dagger:=\G i$\\
 Dirac Conjugate
& $\bar\psi=\ii\psi^\dagger\G0$
\end{tabular}





%%% Local Variables:
%%% TeX-master: "CheatSheet.tex"
%%% End:
