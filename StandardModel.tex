%#!platexmake CheatSheet
%%% Time-Stamp: <2013-01-09 22:23:14 misho>
%%% 一部で日本語が使用されています。

\section{Standard Model}
Any representations assumed to be {\it normalized Hermitian}. Note that the $SU(2)$ {\bf 2} representation is
\begin{equation}
 T^a=\frac12\sigma^a;\qquad  \left[T^a,T^b\right]=\ii \epsilon^{abc}T^c;\qquad T^\pm := T^1\pm\ii T^2.
\end{equation}
We use the following abridged notations:
\begin{align}
  (\partial A)_{\mu\nu}&:=\partial_\mu A_\nu-\partial_\nu A_\mu,&
  F_{\mu\nu}^a &:= \Pm A^a_\nu- \Pn A^a_\mu + gf^{abc}A^b_\mu A^c_\nu.
\end{align}
\subsection{Symmetries and Fields}
\begin{center}
 \begin{tabular}[b]{@{\Vrule\ }c@{\ }l|c|c|c@{\ \Vrule}}\Hrule
   & & $\gSU(3)\s{strong}$ & $\gSU(2)\s{weak} $ & $\gU(1)_Y$ \\\Hrule
&\multicolumn{4}{l@{\Vrule}}{{\bf Matter Fields} (Fermionic / Lorentz Spinor)}\\\hline
 $\PL Q_i$ &: Left-handed quarks              & $\vc 3$ & $\vc 2$ & $1/6$\\\hline
 $\PL U_i$ &: Right-handed up-type quarks     & $\vc 3$ & $\vc 1$ & $2/3$\\\hline
 $\PR D_i$ &: Right-handed down-type quarks   & $\vc 3$ & $\vc 1$ & $-1/3$\\\hline
 $\PR L_i$ &: Left-handed leptons             & $\vc 1$ & $\vc 2$ & $-1/2$\\\hline
 $\PR E_i$ &: Right-handed leptons            & $\vc 1$ & $\vc 1$ & $-1$\\\Hrule
&\multicolumn{4}{l@{\Vrule}}{{\bf Higgs Field} (Bosonic / Lorentz Scalar)}\\\hline
 $H$   &: Higgs                           & $\vc 1$ & $\vc 2$ & $1/2$\\\Hrule
&\multicolumn{4}{l@{\Vrule}}{{\bf Gauge Fields} (Bosonic / Lorentz Vector)}\\\hline
 $G$   &: Gluons                          & $\vc 8$ & $\vc 1$ & $0$\\\hline
 $W$   &: Weak bosons                     & $\vc 1$ & $\vc 3$ & $0$\\\hline
 $B$   &: B boson                         & $\vc 1$ & $\vc 1$ & $0$\\\Hrule
\end{tabular}
\end{center}
\paragraph{Full Lagrangian}
$ \Lag = \Lag\s{gauge}
       + \Lag\s{Higgs}
       + \Lag\s{matter}
       + \Lag\s{湯川}$
\begin{align}
\text{where\quad}
%%% Gauge Boson Kinetic
\Lag\s{gauge} =&\
 -\frac14B^{\mu\nu}B_{\mu\nu}
 -\frac14W^{a\mu\nu}W^a_{\mu\nu}
 -\frac14G^{a\mu\nu}G^a_{\mu\nu}\\
%%% Higgs
\Lag\s{Higgs} =&\
 \left|\left(\partial_\mu-\ii g_2W_\mu-\frac12\ii g_1B_\mu\right)H\right|^2
 - V(H),\\
%%% MATTERS
\begin{split}\label{eq:SMLagMatter}
\Lag\s{matter} =&\
% Left Quark
 \overline Q_i\ii\gamma^\mu\left(\partial_\mu-\ii g_3G_\mu-\ii
 g_2W_\mu-\frac16\ii g_1B_\mu\right)\PL Q_i\\
& % Up Quark
 + \overline U_i\ii\gamma^\mu\left(\partial_\mu-\ii g_3G_\mu-\frac23\ii g_1B_\mu\right)\PR U_i\\
& % Down Quark
 + \overline D_i\ii\gamma^\mu\left(\partial_\mu-\ii g_3G_\mu+\frac13\ii g_1B_\mu\right)\PR D_i\\
& % Left Lepton
 + \overline L_i\ii\gamma^\mu\left(\partial_\mu-\ii g_2W_\mu+\frac12\ii g_1B_\mu\right)\PL L_i\\
& % Right Electron
 + \overline E_i\ii\gamma^\mu\left(\partial_\mu+\ii g_1B_\mu\right)\PR E_i,
\end{split}\\
%%% YUKAWA
\Lag\s{湯川} =&\
 \overline U_i(y_u)_{ij}H\PL Q_j - \overline D_i (y_d)_{ij}H^\dagger\PL Q_j - \overline E_i(y_e)_{ij}H^\dagger\PL
 L_j + \Hc
\end{align}
We have no freedom to add other terms into this Lagrangian of the gauge theory. See Appendix~\ref{sec:no-other-term}.

\paragraph{Gauge Kinetic Terms}
the gauge kinetic terms can be expanded as
\begin{align}
\Lag\s{gauge}
=& -\frac14(\partial B)(\partial B)\notag\\
\label{eq:SMGaugeKineticBWG}
& -\frac14(\partial W^a)(\partial W^a)
-g_2\epsilon^{abc}(\partial_\mu W_\nu^a)W^{\mu b}W^{\nu c}
-\frac{{g_2}^2}4
 \left(\epsilon^{eab}W_\mu^aW_\nu^b\right)\left(\epsilon^{ecd}W^{c\mu}W^{d\nu}\right)\\
&-\frac14(\partial G^a)(\partial G^a)
-g_3f^{abc}(\partial_\mu G_\nu^a)G^{\mu b}G^{\nu c}
-\frac{{g_3}^2}4
 \left(f^{eab}G_\mu^aG_\nu^b\right)\left(f^{ecd}G^{c\mu}G^{d\nu}\right).\notag
\end{align}

\subsection{Higgs Mechanism}
\paragraph{Higgs Potential}
\label{sec:higgs-mechanism}
The (renormalizable) Higgs potential must be
\begin{equation}
 V(H) = -\mu^2(H^\dagger H) + \lambda\left(H^\dagger H\right)^2.
\end{equation}
for the $\gSU(2)$, and $\lambda>0$ in order not to run away the VEVs, while $\mu^2$ is positive for the EWSB.

To discuss this clearly, let us {\em redefine} the Higgs field as
\begin{equation}
 H = \frac1{\sqrt2}\pmat{\phi_1+\ii\phi_2\\v+(h+\ii\phi_3)}, \where v=\sqrt{\frac{\mu^2}{\lambda}}.
\end{equation}
Here $h$ is the ``Higgs boson,'' and $\phi_i$ are 南部--Goldstone bosons.

The Higgs potential becomes
\begin{equation}
   V(h) = \frac{\mu^2}{4v^2}h^4 + \frac{\mu^2}v h^3 + \mu^2h^2,
\end{equation}
and now we know the Higgs boson has acquired mass $m_h=\sqrt2 \mu$. Also
\begin{align}
 \Lag\stx{Higgs}&=
   \left|\left(\partial_\mu-\ii g_2W_\mu-\frac12\ii g_1B_\mu\right)H\right|^2\\
&= \frac12(\partial_\mu h)^2+\frac{(v+h)^2}8\Bigl[{g_2}^2{W_1}^2+{g_2}^2{W_2}^2+(g_1B-g_2W_3)^2\Bigr].
\end{align}
Redefining the gauge fields (with concerning the norms) as
\begin{align}
 W^\pm_\mu&:=\frac1{\sqrt2}(W^1_\mu\mp\ii W^2_\mu),&
\pmat{Z_\mu \\ A_\mu}
&:= \pmat{%
\cos\theta\s w & -\sin\theta\s w\\
\sin\theta\s w & \cos\theta\s w
}
\pmat{W^3_\mu \\ B_\mu},
\end{align}
where
\begin{align}
 &\tan\theta\s w := \frac{g_1}{g_2},\qquad
 e              := -\frac{g_1g_2}{\sqrt{{g_1}^2+{g_2}^2}};\qquad
 g_Z            := \sqrt{{g_1}^2+{g_2}^2};\\
 &g_1=\frac{|e|}{\cos\theta\s w}=g_Z\sin\theta\s w,\qquad
  g_2=\frac{|e|}{\sin\theta\s w}=g_Z\cos\theta\s w.
\end{align}
We obtain the following terms in $\Lag\s{Higgs}$:
\begin{equation}
  \Lag\s{Higgs}
\supset \frac12(\partial_\mu h)^2+\frac{(v+h)^2}4\Bigl[{g_2}^2{W^+}^\mu W^-_\mu + \frac{{g_Z}^2}{2}Z^\mu Z_\mu\Bigr].
\end{equation}
Here we have omitted the 南部--Goldstone bosons.


\begin{rightnote}
Here we present another form:
\begin{eqnarray}
  g_1B_\mu
&=& |e| A_\mu-\tan\theta\s w Z_\mu,\\
 g_2W_\mu
&=& \frac{g_2}{\sqrt2}\left(W^+_\mu T^+ + W^-_\mu T^-\right)
   +\left(\frac{|e|}{\tan\theta\s w}Z_\mu+|e|A_\mu\right)T^3,
\end{eqnarray}
\begin{equation}
Z^0_\mu:=\frac1{\sqrt{{g_1}^2+{g_2}^2}}(g_2 W^3_\mu-g_1B_\mu),\quad
A_\mu:=\frac1{\sqrt{{g_1}^2+{g_2}^2}}(g_1 W^3_\mu+g_2B_\mu)
\end{equation}
\end{rightnote}

You can see the gauge bosons have acquired the masses
\begin{equation}
  m_A = 0, \quad m_W :=\frac{g_2}2v,\quad m_Z :=\frac{g_Z}2v.
\end{equation}


\paragraph{Gauge Term}
The $\gSU(2)$ gauge term is converted into
\begin{align*}
   W^{a\mu\nu}W^a_{\mu\nu}
&= (\partial W^3)(\partial W^3) + 2(\partial W^+)(\partial W^-)
\\&\quad
- 4\ii g\left[
     (\partial W^3)^{\mu\nu}W^+_\mu W^-_\nu
   + (\partial W^+)^{\mu\nu}W^-_\mu W^3_\nu
   + (\partial W^-)^{\mu\nu}W^3_\mu W^+_\nu
\right]
\\&\quad
 -2g^2(\Hmn\Hrs-\Hmr\Hns)
\left(
  W^+_\mu W^+_\nu W^-_\rho W^-_\sigma  - 2 W^3_\mu W^3_\nu W^+_\rho W^-_\sigma
\right),
\end{align*}
and therefore the final expression is
\begin{equation}
\label{eq:SMGaugeKineticGZG}
\begin{split}
 \Lag\s{gauge}&:=
-\frac14\left[
G^{a\mu\nu}G^a_{\mu\nu} + (\partial Z)^{\mu\nu}(\partial Z)_{\mu\nu}+(\partial A)^{\mu\nu}(\partial A)_{\mu\nu}+2(\partial W^+)^{\mu\nu}(\partial W^-)_{\mu\nu}
\right]\\&\quad
+\frac{\ii |e|}{\tan\theta\s w}\Bigl[
(\partial W^+)^{\mu\nu}W^-_\mu Z_\nu + (\partial W^-)^{\mu\nu}Z_\mu W^+_\nu +  (\partial Z)^{\mu\nu}W^+_\mu W^-_\nu
\Bigr]\\&\quad
+ \ii |e|\Bigl[
(\partial W^+)^{\mu\nu}W^-_\mu A_\nu + (\partial W^-)^{\mu\nu}A_\mu W^+_\nu +  (\partial A)^{\mu\nu}W^+_\mu W^-_\nu
\Bigr]\\&\quad
+(\Hmn\Hrs-\Hmr\Hns)\left[
\frac{|e|^2}{2\sin^2\theta\s w}W^+_\mu W^+_\nu W^-_\rho W^-_\sigma
+\frac{|e|^2}{\tan^2\theta\s w}W^+_\mu Z_\nu W^-_\rho Z_\sigma
\right.\\&\left.\qquad\qquad\qquad
+\frac{|e|^2}{\tan\theta\s w}\left(W^+_\mu Z_\nu W^-_\rho A_\sigma + W^+_\mu A_\nu W^-_\rho Z_\sigma\right)
+|e|^2W^+_\mu A_\nu W^-_\rho A_\sigma\right].
\end{split}
\end{equation}

\paragraph{湯川 Term}
\begin{align}
 \Lag\s{湯川}=
&\overline U y_uH\PL Q - \overline D y_dH^\dagger\PL Q - \overline E y_eH^\dagger\PL L
 + \Hc\notag\\
=
& \overline U y_u\epsilon^{\alpha\beta}H^{\alpha}\PL Q^{\beta}
 -\overline D y_d{H^\dagger}^{\alpha}\PL Q^\alpha
 -\overline E y_e{H^\dagger}^{\alpha}\PL L^\alpha + \Hc\notag\\
=
&-\frac{v+h}{\sqrt2}\left(
   \overline U y_u\PL Q^1
 + \overline D y_d\PL Q^2
 + \overline E y_e\PL L^2
\right) + \Hc
\end{align}
\subsection{Full Lagrangian After Higgs Mechanism}
Now we have the following Lagrangian (with omitting $\PL$ etc.):
\begin{align}
 \Lag =
& \Lag\s{gauge}
{\ +\ }{m_W}^2W^+W^-
{\ +\ }\frac{{m_Z}^2}2Z^2\notag\\[.5zw]
%Higgs
\NOTE{Higgs}&
{\ +\ }\frac12(\partial_\mu h)^2
{\ -}\frac12{m_h}^2h^2
{-}\sqrt{\frac\lambda2}m_h h^3
{-}\frac14\lambda h^4
\notag\\&
{\ +\ }\frac{v{g_2}^2}{4}W^+W^-h
{\ +\ }\frac{v({g_1}^2+{g_2}^2)}{8}Z^2h\notag\\&
{\ +\ }\frac{{g_2}^2}{4}W^+W^-h^2
{\ +\ }\frac{{g_1}^2+{g_2}^2}{8}Z^2h^2\notag\\
&
{\ -\ }\left(\frac{1}{\sqrt2}h\bar U y_u Q^1
{\ +\ }\frac{1}{\sqrt2}h\bar D y_d Q^2
{\ +\ }\frac{1}{\sqrt2}h\bar E y_e L^2 \ +\ \Hc\right)\notag\\[1zw]
%SU(3) Interactions
\NOTE{$\gSU(3)$}&
{\ +\ }\bar Q\left(\ii\slashed\partial+g_3\slashed G\right) Q
{\ +\ }\bar U\left(\ii\slashed\partial+g_3\slashed G\right) U
{\ +\ }\bar D\left(\ii\slashed\partial+g_3\slashed G\right) D
{\ +\ }\bar L\left(\ii\slashed\partial\right) L
{\ +\ }\bar E\left(\ii\slashed\partial\right) E\notag\\[1zw]
\NOTE{$W$}&
{\ +\ }\bar Q\frac{g_2}{\sqrt2}\left(\slashed W^+ T^+ + \slashed W^- T^-\right)Q
{\ +\ }\bar L\frac{g_2}{\sqrt2}\left(\slashed W^+ T^+ + \slashed W^- T^-\right)L\notag\\
\NOTE{$A$\&$Z^0$}&
{\ +\ }\bar Q\left[
    \left(T^3+\frac16\right)|e|\slashed A
   +\left(\frac{|e|c}{s}T^3-\frac{|e|s}{6c}\right)\slashed Z^0
  \right] Q\notag\\
&{\ +\ }\bar U\left(\frac23 |e|\slashed A-\frac{2|e|s}{3c}\slashed Z\right) U\notag\\
&{\ +\ }\bar D\left(-\frac13 |e|\slashed A+\frac{|e|s}{3c}\slashed Z\right) D\notag\\
&{\ +\ }\bar L\left[
    \left(T^3-\frac12\right)|e|\slashed A
   +\left(\frac{|e|c}{s}T^3+\frac{|e|s}{2c}\right)\slashed Z^0
  \right] L\notag\\
&
{\ +\ }\bar E\left(-|e|\slashed A+\frac{|e|s}{c}\slashed Z\right) E\notag\\[.5zw]
\NOTE{湯川項}& % Yukawa
{\ -\ }\left(\frac{1}{\sqrt2}v\bar U y_u Q^1
{\ +\ }\frac{1}{\sqrt2}v\bar D y_d Q^2
{\ +\ }\frac{1}{\sqrt2}v\bar E y_e L^2\ +\ \Hc\right)
\end{align}
\subsection{Mass Eigenstates}
Here we will obtain the mass eigenstates of the fermions, by diagonalizing the 湯川 matrices.

We use the singular value decomposition method to mass matrices $Y_\bullet:=vy_\bullet/\sqrt2$.
Generally, any matrices can be transformed with two unitary matrices $\Psi$ and $\Phi$ as
\begin{equation}
 Y=\Phi^\dagger\pmat{m_1&0&0\\0&m_2&0\\0&0&m_3}\Psi =:\Phi^\dagger M\Psi\qquad(m_i\ge0).
\end{equation}
Using this $\Psi$ and $\Phi$, we can rotate the basis as
\begin{align}
 &Q^1\mapsto \Psi_u^\dagger Q^1,\quad
 Q^2\mapsto \Psi_d^\dagger Q^2,\quad
 L\mapsto \Psi_e^\dagger L,\quad
 &U\mapsto \Phi_u^\dagger U,\quad
 D\mapsto \Phi_d^\dagger D,\quad
 E\mapsto \Phi_e^\dagger E,\label{eq:gaugeeig_to_masseig}
\end{align}
and now we have the 湯川 terms in mass eigenstates as
\begin{equation}
  \Lag\s{湯川}
= -\left(1+\frac{1}{v}h\right)\left[
   {(m_u)_i} \overline U_i \PL Q_i^1
 + {(m_d)_i} \overline D_i \PL Q_i^2
 + {(m_e)_i} \overline E_i \PL L_i^2 + \Hc\right].
\end{equation}

In the transformation from the gauge eigenstates to the mass eigenstates, almost all the terms in the Lagrangian are not modified.
However, only the terms of quark--quark--$W$ interactions do change drastically, as
\begin{align}
 \Lag
&\supset
 \overline Q\ii\gamma^\mu\left(-\ii g_2W_\mu-\frac16\ii g_1B_\mu\right)\PL Q\\
&=\overline Q\frac{g_2}{\sqrt2}\left(\slashed W^+ T^+ + \slashed W^- T^-\right)\PL Q
\quad+\quad(\text{interaction terms with $Z$ and $A$})\\
&\mapsto%
\frac{g_2}{\sqrt2}
\pmat{\overline Q^1\Psi_u&\overline Q^2\Psi_d}
\pmat{0&\slashed W^+\\\slashed W^-&0}\PL\pmat{\Psi_u^\dagger Q^1\\\Psi_d^\dagger Q^2} + (\quad{\cdots}\quad)\\
&=\frac{g_2}{\sqrt2}\left[\overline Q^2\slashed W^-X \PL Q^1+ \overline Q^1\slashed W^+X^\dagger\PL Q^2\right]+(\quad{\cdots}\quad),
\end{align}
where $X:=\Psi_d\Psi_u^\dagger$ is a matrix, so-called the Cabbibo--小林--益川~(CKM) matrix, which is {\em not} diagonal, and {\em not} real, generally.
These terms violate the flavor symmetry of quarks, and even the $CP$-symmetry.
\begin{rightnote}
In our notation, $CP$-transformation of a spinor is described as
\begin{equation}
 \mathscr{CP}\left(\psi\right)= -\ii\eta^*\trans{(\overline\psi\G2)},\quad
 \mathscr{CP}\left(\overline\psi\right)= \ii\eta\trans{(\G2\psi)},
\end{equation}
where $\eta$ is a complex phase ($|\eta|=1$).
Under this transformation, those terms are transformed as, e.g.,
\begin{eqnarray}
\mathscr{CP}\left( \overline Q^2\slashed W^-X \PL Q^1\right)
&=&
\trans{(\G2Q^2)} \mathscr{P}(-{\slashed W^+})X\PL\trans{(\overline Q^1\G2)}\notag\\
&=&-{W_\mu^+}^P\trans{(\G2Q^2)}\trans{(\overline Q^1\trans X\G2\PL\trans{\Gm })}\\
&=&(\overline Q^1{\slashed W^+}\trans X\PL Q^2).\notag
\end{eqnarray}
Therefore, we can see that the $CP$-symmetry is maintained if and only if $\trans X=X^\dagger$, that is, if and only if $X$ is a real matrix.
\end{rightnote}

以上より,標準模型のLagrangianは
\begin{align}
 \Lag =
& \Lag\s{gauge}\notag\\
\NOTE{質量項}& % Yukawa
{\quad+\quad}{m_W}^2W^+W^-
{\quad+\quad}\frac{{m_Z}^2}2Z^2
\notag\\&
{\quad-\quad}\left(\bar U M_u \PL Q^1
{\ +\ }\bar D M_d \PL Q^2
{\ +\ }\bar E M_e \PL L^2\ +\ \Hc\right)
\notag\\[.5zw]
%Higgs
\NOTE{Higgs Field}&
{\quad+\quad}\frac12(\partial_\mu h)^2
{\quad-}\frac12{m_h}^2h^2
{-}\sqrt{\frac\lambda2}m_h h^3
{-}\frac14\lambda h^4\notag\\
\NOTE{Higgsとの結合}&
{\quad+\quad}\frac{v{g_2}^2}{4}W^+W^-h
{\quad+\quad}\frac{v({g_1}^2+{g_2}^2)}{8}Z^2h\notag\\&
{\quad+\quad}\frac{{g_2}^2}{4}W^+W^-h^2
{\quad+\quad}\frac{{g_1}^2+{g_2}^2}{8}Z^2h^2\notag\\
&
{\quad-\quad}\left(\frac1v\bar U M_u \PL Q^1h
{\ +\ }\frac1v\bar D M_d \PL Q^2h
{\ +\ }\frac1v\bar E M_e \PL L^2h\ +\ \Hc\right)\notag\\[1zw]
%SU(3) Interactions
\NOTE{$\gSU(3)$および微分項}&
{\quad+\quad}\bar Q\left(\ii\slashed\partial+g_3\slashed G\right) \PL Q
{\quad+\quad}\bar U\left(\ii\slashed\partial+g_3\slashed G\right) \PR U
{\quad+\quad}\bar D\left(\ii\slashed\partial+g_3\slashed G\right) \PR D\notag\\
&
{\quad+\quad}\bar L\left(\ii\slashed\partial\right) \PL L
{\quad+\quad}\bar E\left(\ii\slashed\partial\right) \PR E\notag\\[1zw]
\NOTE{$W$ boson}&
{\quad+\quad}\frac{g_2}{\sqrt2}\left[
\bar Q^2\slashed W^-X \PL Q^1+ \bar Q^1\slashed W^+X^\dagger \PL Q^2\right]
\qquad\NOTE{← $CP$ and flavor violating!}
\notag\\
&
{\quad+\quad}\bar L\frac{g_2}{\sqrt2}\left(\slashed W^+ T^+ + \slashed W^- T^-\right)\PL L\notag\\
\NOTE{$A$\&$Z^0$ boson}&
{\quad+\quad}\bar Q\left[
    \left(T^3+\frac16\right)|e|\slashed A
   +\left(\frac{|e|c}{s}T^3-\frac{|e|s}{6c}\right)\slashed Z^0
  \right]\PL  Q\notag\\
&{\quad+\quad}\bar U\left(\frac23 |e|\slashed A-\frac{2|e|s}{3c}\slashed Z\right) \PR U\notag\\
&{\quad+\quad}\bar D\left(-\frac13 |e|\slashed A+\frac{|e|s}{3c}\slashed Z\right) \PR D\notag\\
&{\quad+\quad}\bar L\left[
    \left(T^3-\frac12\right)|e|\slashed A
   +\left(\frac{|e|c}{s}T^3+\frac{|e|s}{2c}\right)\slashed Z^0
  \right] \PL L\notag\\
& 
{\quad+\quad}\bar E\left(-|e|\slashed A+\frac{|e|s}{c}\slashed Z\right) \PR E
\end{align}
となる。
\newpage
\subsection{Chiral Notation}\mbox{}\par\vspace{-23pt}
In the chiral expression, the Lagrangian is written as\vspace{-4pt}
\begin{align}
 \mathcal L
=& \text{(Higgs terms)} + \text{(Gauge fields strength)}\notag\\
& + Q\s L^\dagger\ii\bar\sigma^\mu\left(\partial_\mu-\ii g_3G_\mu-\ii
 g_2W_\mu-\frac16\ii g_1B_\mu\right)Q\s L\notag\\
& + U\s R^\dagger\ii\sigma^\mu\left(\partial_\mu-\ii g_3G_\mu-\frac23\ii g_1B_\mu\right)U\s R\notag\\
& + D\s R^\dagger\ii\sigma^\mu\left(\partial_\mu-\ii g_3G_\mu+\frac13\ii g_1B_\mu\right)D\s R\notag\\
& + L\s L^\dagger\ii\bar\sigma^\mu\left(\partial_\mu-\ii g_2W_\mu+\frac12\ii g_1B_\mu\right)L\s L\notag\\
& + E\s R^\dagger\ii\sigma^\mu\left(\partial_\mu+\ii g_1B_\mu\right)E\s R\notag\\
& % Yukawa
 -\left(U^\dagger\s R y_uHQ\s L + D^\dagger\s R y_dH^\dagger Q\s L +
 E^\dagger \s R y_eH^\dagger L\s L + \Hc\right)\notag\\
=& \text{(Higgs terms)} + \text{(Gauge fields strength)}\notag\\
&
+ \ii Q\s L^\dagger\bar\sigma^\mu\partial_\mu Q\s L
+ \ii U\s R        \bar\sigma^\mu\partial_\mu U\s R^\dagger
+ \ii D\s R        \bar\sigma^\mu\partial_\mu D\s R^\dagger
+ \ii L\s L^\dagger\bar\sigma^\mu\partial_\mu L\s L
+ \ii E\s R        \bar\sigma^\mu\partial_\mu E\s R^\dagger \notag\\
&
+ g_3\left(
 Q\s L^\dagger\bar\sigma^\mu G_\mu Q\s L
+U\s R^\dagger\bar\sigma^\mu G_\mu U\s R
+D\s R^\dagger\bar\sigma^\mu G_\mu D\s R
\right)\notag\\
&
+ g_2\left(
 Q\s L^\dagger\bar\sigma^\mu W_\mu Q\s L
+L\s L^\dagger\bar\sigma^\mu W_\mu L\s L
\right)\notag\\
&+ g_1\left(
 \frac16 Q\s L^\dagger\bar\sigma^\mu B_\mu Q\s L
+\frac23 U\s R^\dagger\bar\sigma^\mu B_\mu U\s R
-\frac13 D\s R^\dagger\bar\sigma^\mu B_\mu D\s R
-\frac12 L\s L^\dagger\bar\sigma^\mu B_\mu L\s L
-        E\s R^\dagger\bar\sigma^\mu B_\mu E\s R
\right)\notag\\
& % Yukawa
 - \left(U^\dagger\s R y_uHQ\s L + D^\dagger\s R y_dH^\dagger Q\s L +
 E^\dagger \s R y_eH^\dagger L\s L + \Hc\right),
\end{align}
and finally we obtain
\begin{align}
 \Lag =
& \text{(Gauge bosons and Higgs)}\notag\\
&
+ \ii Q\s L^\dagger\bar\sigma^\mu\partial_\mu Q\s L
+ \ii U\s R        \bar\sigma^\mu\partial_\mu U\s R^\dagger
+ \ii D\s R        \bar\sigma^\mu\partial_\mu D\s R^\dagger
+ \ii L\s L^\dagger\bar\sigma^\mu\partial_\mu L\s L
+ \ii E\s R        \bar\sigma^\mu\partial_\mu E\s R^\dagger \notag\\
&
+ g_3\left(
 Q\s L^\dagger\bar\sigma^\mu G_\mu Q\s L
+U\s R^\dagger\bar\sigma^\mu G_\mu U\s R
+D\s R^\dagger\bar\sigma^\mu G_\mu D\s R
\right)\notag\\
&
-m_u(u\s R^\dagger u\s L+u\s L^\dagger u\s R)-\text{(quarks)}
-m_e(e\s R^\dagger e\s L+e\s L^\dagger e\s R)-\text{(leptons)}
\notag\\&
-\frac{m_u}v(u\s R^\dagger u\s L+u\s L^\dagger u\s R)h - \text{(quarks)}
-\frac{m_e}v(e\s R^\dagger e\s L+e\s L^\dagger e\s R)h - \text{(leptons)}
\notag\\
&+
\frac{g_2}{\sqrt2}\left[
\pmat{d\s L^\dagger\ s\s L^\dagger\ b\s L^\dagger}
\bar\sigma^\mu W^-_\mu X \pmat{u\s L\\c\s L\\t\s L}
+
\pmat{u\s L^\dagger\ c\s L^\dagger\ t\s L^\dagger}
\bar\sigma^\mu W^+_\mu X^\dagger \pmat{d\s L\\s\s L\\b\s L}
\right]
\notag\\&
+\frac{g_2}{\sqrt2}\left[
\nu_e^\dagger\bar\sigma^\mu W^+_\mu e\s L + e\s L^\dagger\bar\sigma^\mu W^-_\mu \nu_e
\right]\notag\\
&
+|e|\left[
  \frac23 u\s L^\dagger \bar\sigma^\mu A_\mu u\s L
- \frac13 d\s L^\dagger \bar\sigma^\mu A_\mu d\s L
+ \frac23 u\s R^\dagger     \sigma^\mu A_\mu u\s R
- \frac13 d\s R^\dagger     \sigma^\mu A_\mu d\s R
+ \text{(quarks)}
\right.\notag\\&\left.\qquad\qquad
- e\s L^\dagger \bar\sigma^\mu A_\mu e\s L
- e\s R^\dagger     \sigma^\mu A_\mu e\s R
+ \text{(leptons)}\right]\notag\\
&%
+\frac{|e|s}{c}\left[
  \left(\frac{c^2}{2s^2}-\frac16\right)u\s L^\dagger \bar\sigma^\mu Z_\mu u\s L
- \left(\frac{c^2}{2s^2}+\frac16\right)d\s L^\dagger \bar\sigma^\mu Z_\mu d\s L
- \frac23 u\s R^\dagger     \sigma^\mu Z_\mu u\s R
+ \frac13 d\s R^\dagger     \sigma^\mu Z_\mu d\s R
\right.\notag\\&\left.\qquad\qquad
+ \left(\frac{c^2}{2s^2}+\frac12\right)\nu_e^\dagger \bar\sigma^\mu Z_\mu \nu_e
- \left(\frac{c^2}{2s^2}-\frac12\right)e\s L^\dagger \bar\sigma^\mu Z_\mu e\s L
+ e\s R^\dagger     \sigma^\mu Z_\mu e\s R
+ \text{(others)}\right].
\end{align}


\newpage

\subsection{Values of SM Parameters}
\newcommand{\upd}[1]{{\RED{#1}}}
\vskip-24pt\hskip200pt{\small(Extracted from PDG 2010 / \upd{2012})}
\subsubsection{Experimental Values}
\subparagraph{Theoretical Parameters}
\NOTE{These values are all in $\overline{\rm{MS}}$ scheme.}
\begin{align*}
 \alpha^{-1}\s{EM}(0) &= 137.035999\upd{074(44)} &
 G\s{F}       &= \frac{{g_2}^2}{4\sqrt2{m_W}^2} = \frac1{\sqrt2v^2} = 1.16637\upd{87(6)}\EE-5\un{GeV^{-2}}&
\end{align*}
\begin{align*}
 \alpha^{-1}\s{EM}(m_Z)    &= 127.9\upd{44(14)} &
 m_W(m_W)                  &= 80.3\upd{85(15)}\un{GeV} &
 \Gamma_W                  &\approx 2.085(42)\un{GeV}
\\
 \alpha^{-1}\s{EM}(m_\tau) &= 133.4\upd{71(14)} &
 m_Z(m_Z)                  &= 91.1876(21)\un{GeV} &
 \Gamma_Z                  &\approx 2.4952(23)\un{GeV}
\\
 \alpha\s{s}(m_Z)          &= 0.118\upd{4(7)}  &
 \sin^2\theta\s W(m_Z)     &= 0.23116(1\upd{2}) &
 \sin^2\theta\s{eff}       &= 0.23146(12)
\end{align*}

\subparagraph{Masses and Lifetimes}
\NOTE{$t$ pole mass is the ``MC mass''. Quark $\overline{\rm{MS}}$ mass at 1\,GeV can be obtained by $\times 1.35$.}
\begin{align*}
 e   &: 0.5109989\upd{28(11)}\un{MeV}         &
\mu  &: 105.6583\upd{715(35)}\un{MeV}         &
\tau &: 1.77682(16)\un{GeV}                   &
\end{align*}
\begin{align*}
\text{\small [$\overline{\rm MS}$ (2\,GeV)]}\ \
 u   &: \upd{2.3\pmunc{0.7}{0.5}}\un{MeV}    &
\text{\small [$\overline{\rm MS}$($m$)]}\ \
 c   &: 1.2{\upd{75(25)}}\un{GeV}             &
\text{\small [pole]}\ \
 c   &: 1.67(7)\un{GeV}
\\
 d   &: \upd{4.8\pmunc{0.7}{0.3}}\un{MeV}    &
 b   &: 4.1\upd{8(3)}\un{GeV}                 &
 b   &: 4.78(6)\un{GeV}
\\
 s   &: \upd{95\pm5}\un{MeV}                  &
 t   &: 160\pmunc{5}{4}\un{GeV}              &
 t   &: 17\upd{3.5(6)(8)}\un{GeV}
\end{align*}
\begin{align*}
 \pi^\pm &: 139.57018(35)\un{MeV}&
 K^\pm   &: 493.677(16)  \un{MeV}&
 p       &: 938.2720\upd{46(21)} \un{MeV}\\
 \pi^0   &: 134.9766(6)   \un{MeV}&
 K^0     &: 497.614(24)  \un{MeV}&
 n       &: 939.5653\upd{79(21)}\un{MeV}
\end{align*}
\begin{align*}
 \mu     &: 2.1969\upd{811(22)}   \un{\mu s} \ (659\un{m}) &
 \pi^\pm &: 2.6033(5)\EE-8        \un{s}&
 K^\pm   &: 1.2380(21)\EE-8       \un{s}     \ (3.7\un{m})\\
 \tau    &: 2.906(10)\EE{-13}     \un{s}     \ (87\un{\mu m})&
 \pi^0   &: 8.\upd{52(18)}\EE-{17}\un{s}&
 K^0\s{S}&: 8.95\upd{64}(33)\EE{-11}\un{s}   \ (2.68\un{cm})&\\
&&&&
 K^0\s{L}&: 5.116(2\upd{1})\EE-8\un{s}       \ (15.3\un{m})
\end{align*}
\subparagraph{Other Important Values}
\begin{align*}
     a_e                  &= 11596521.8076(27)\EE{-10} &
 d^{\rm EDM}_e            &< 10.5\EE{-28}e\un{cm}&
& \Br(\tau\to e)          = 17.83(4)\%
\\
     a_\mu                &= 11659209(6)\EE{-10} &
 d^{\rm EDM}_\mu          &= -1(9)\EE{-20}e\un{cm}&
& \Br(\tau\to \mu)         = 17.41(4)\%
\\
&&
 \sin^22\theta_{12} &= 0.857(24)&
& \Br(\tau\to \text{had})  \sim 64.8\%
\\
&&
 \sin^22\theta_{23} &> 0.95&
& \Delta m_{\nu21}^2 = 7.50(20)\EE-5\eV^2
\\
&&
 \sin^22\theta_{13} &= 0.098(13)&
& \left|\Delta m_{\nu32}^2\right| = 0.00232(^{12}_{08})\eV^2\\
\end{align*}
\subparagraph{CKM matrix}
\begin{equation*}
  V\s{CKM}=
\pmat{
0.97425(22) & 0.2252(9)   & 0.0084(6)\\
0.230(11)   & 1.006(23)   & 0.0429(26)\\
0.00415(49) & 0.0409(11)  & 0.89(7)
}
\approx
\pmat{
0.9742\upd{7(15)}         & 0.225\upd{34(65)}        & 0.003\upd{51(^{15}_{14})}\\
0.225\upd{20(65)}         & 0.9734\upd{4}(16)        & 0.04\upd{12(^{11}_{05})}\\
0.008\upd{67(^{29}_{31})} & 0.04\upd{04(^{11}_{05})} & 0.9991\upd{46(^{21}_{46})}
}
\end{equation*}
\begin{align*}
 \lambda &= 0.22535(65),&
 A&=0.811\pmunc{0.022}{0.012},&
 \bar\rho&=0.131\pmunc{0.026}{0.013},&
 \bar\eta&=0.345\pmunc{0.013}{0.014};&
 J&=(2.96\pmunc{20}{16})\EE-5
\end{align*}


\subsubsection{Estimation of SM Parameters}
For EW scale, we can estimate the values as
\begin{align}
 |e|&\sim 0.313,& g_1 &\sim 0.357, & g_2 &\sim 0.652, & g_Z &\sim 0.743; &
  v=\sqrt{\frac{\mu^2}\lambda} &\sim 246\un{GeV}
\end{align}
Therefore 湯川 matrices are (after diagonalization), since $vy/\sqrt2=M$,
\begin{align}
 y_u&\sim \pmat{10^{-5}&0&0\\0&0.007&0\\0&0&0.997}&
 y_d&\sim \pmat{3\times10^{-5}&0&0\\0&0.0005&0\\0&0&0.02}&
 y_e&\sim \pmat{3\times10^{-6}&0&0\\0&0.0006&0\\0&0&0.01}&
\end{align}

Also, for $m_h\sim 125\un{GeV}$, we can estimate the Higgs potential as $\mu\sim88\un{GeV}$ and $\lambda \sim 0.13$.

%%% Local Variables:
%%% TeX-master: "CheatSheet.tex"
%%% End:
