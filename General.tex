%#!latexmk CheatSheet.tex
%%% Time-Stamp: <2009-09-22 00:06:56 misho>
% �����R�[�h�ɒ��ӂ��܂��傤
\section{General Definitions and Tools}
\subsection{Notations and Conventions}
\subsubsection{Metric etc.}

\begin{tabular}{l@{ :\ \ \ }l}
Minkowski Metric   & $\Hmn = \diag(+,-,-,-)$\\
Coordinates        & $\displaystyle x^\mu = (t,x,y,z)$; \quad Therefore $\Pm = \left(\pbib{}t,\vgrad\right)$\\
Gamma Matrices     & $\{\Gm,\Gn\}=2\Hmn;\quad \G5=\ii\G0\G1\G2\G3$\\[1zw]
Gamma Combinations & $1, \{\Gm\}, \{\sigma^{\mu\nu}\}, \{\Gm\G5\}, \G5;\quad
                      \sigma^{\mu\nu}=\frac\ii2[\Gm,\Gn]=0\Big/\ii\Gm\Gn$\\
\end{tabular}

\subsubsection{Fields}
\begin{tabular}{l@{ :\ \ \ }l}
Klein-Gordon Equation & $|\Pm \phi|^2-m^2|\phi|^2=0$\\
Klein-Gordon Field    &
    $\displaystyle\phi(x)=\intP3p \frac{1}{\sqrt{2E_{\vc p}}}
            \Bigl[a_\vc p\ee^{-\ii px} + b^\dagger_\vc p \ee^{\ii px}\Bigr]$\\[1.5zw]
Dirac Equation   & $(i\slashed\partial-m)\psi(x)=0$\\
Dirac Field      &
    $\displaystyle\psi(x)=\intP3p \frac{1}{\sqrt{2E_{\vc p}}}
     \sum_{s=1,2}\Bigl[a^s_\vc p u^s(p)\ee^{-\ii px}
                     + b^{s\dagger}_\vc p v^s(p)\ee^{\ii px}\Bigr]$\\[1.5zw]
Gauge Boson & $(\Psq+m^2)A^\mu(x)=0$\quad \text{\footnotesize (Real Klein-Gordon Equation)}\\
\quad{\footnotesize(Before Gauge Fixing)}
  & $\displaystyle A^\mu(x)=\intP3p \frac{1}{\sqrt{2E_{\vc p}}}
     \sum_{r=0,1,2,3}\Bigl[a^r_\vc p \epsilon^r(p)\ee^{-\ii px}
                     + a^{r\dagger}_\vc p \epsilon^{r*}(p)\ee^{\ii px}\Bigr]$\\[1.5zw]
�암-Goldstone Boson & \TODO.
\end{tabular}
\begin{rightnote}
\Subparagraph{Chiral Notation}
\begin{tabular}{l@{ :\ \ \ }l}
Gamma Matrices & $\Gm=\pmat{0&\Sm\\\bSm&0};\quad \G5=\pmat{-1&0\\0&1}$\\
Dirac Field    & $\psi    =\pmat{\psi\s L\\\psi\s R}; \quad
                  \bar\psi=\psi^\dagger\G0=\pmat{\psi\s R^\dagger&\psi\s L^\dagger}$\\
 & $u^s(p)=\pmat{\sqrt{p\cdot\sigma}\xi^s\\\sqrt{p\cdot\bar\sigma}\xi^s};\
    v^s(p)=\pmat{\sqrt{p\cdot\sigma}\eta^s\\-\sqrt{p\cdot\bar\sigma}\eta^s}$\\
 & $\Bigl[\eta^s=\xi^{-s}:=-\ii\sigma^2(\xi^s)^*=(\xi^2,-\xi^1)\Bigr]$\\

Weyl Equations & $\ii\bar\sigma\cdot\partial\psi_L=m\psi_R;\quad
                  \ii\sigma\cdot\partial\psi_L=m\psi_L$\\
CPT transf.
 & $P\psi(t,\vc x) P = \eta\G0\psi(t,-\vc x)\quad(|\eta|^2=1)$\\
 & $T\psi(t,\vc x) T = \G1\G3\psi(-t,\vc x)\quad\text{(ignoring intrinsic phase)}$\\
 & $C\psi(t,\vc x) C = -\ii\G2\psi^*(t,\vc x) = -\ii\trans{(\bar\psi\G0\G2)}\quad
                       \text{(�V)}$\\
 & $\bar\psi\longrightarrow
     P: \eta^*\bar\psi\G0\quad
     T: -\bar\psi\G1\G3\quad
     C: \ii\bar\psi^*\G2=-\ii\trans{(\G0\G2\psi)}$\\
\end{tabular}
\end{rightnote}

\begin{tabular}{l@{ :\ \ \ }l}
Electromagnetic Fields & $A^\mu=(\phi,\vc A)$
 \note{We can invert the signs, but cannot lower the index.}\\
& $F_{\mu\nu} =\pmat{0 & & \vc E & \\ & 0 & -B_3 & B_2 \\ -\vc E&B_3&0&-B_1\\&-B_2&B_1&0}$\\
& $F_{\mu\nu}F^{\mu\nu} = -2\left(\vnorm E^2-\vnorm B^2\right)$
\end{tabular}

\subsubsection{CPT Table}
\begin{tabular}[t]{c|c|c|cccccc}
 & $\phi$ & $A^\mu$
 & $\bar\psi\psi$ & $\bar\psi\Gm\psi$ & $\bar\psi\Smn\psi$
 & $\bar\psi\Gm\G5\psi$ & $\ii\bar\psi\G5\psi$ & $\Pm$\\\hline
$P$
  & $\eta\phi$ & $\eta$$-$$+$$+$$+$$A^\mu$
  &$+$&$+$$-$$-$$-$&($+$$-$$-$$-$)($+$$-$$-$$-$)&$-$$+$$+$$+$&$-$&$+$$-$$-$$-$\\
$T$
  & $\zeta\phi$ & $\zeta$$+$$-$$-$$-$$A^\mu$
  &$+$&$+$$-$$-$$-$&$-$($+$$-$$-$$-$)($+$$-$$-$$-$)&$+$$-$$-$$-$&$-$&$-$$+$$+$$+$\\
$C$
  & $\xi\phi^*$ & $\xi$$+$$A^{\mu*}$
  &$+$&$-$&$-$&$+$&$+$&$+$\\
\end{tabular}\vspace{.5zw}

($\eta\zeta\xi=1$;
 especially, photon $A^\mu$ is $(\eta,\zeta,\xi)=(-,+,-)$. )

\newpage
\subsection{Dirac's Gamma Algebras}
\subsubsection{Traces}\vspace{-2.5zw}
\begin{align}
 \Tr(\text{any odd \# of $\gamma$'s})&=0\\
 \Tr(\Gm\Gn)&= 4\Hmn\\
 \Tr(\Gm\Gn\Gr\Gs) &= 4(\Hmn\Hrs-\Hmr\Hns+\Hms\Hnr)\\
 \Tr(\text{$\G5$ and any odd \# of $\gamma$'s})&=0\\
 \Tr(\Gm\Gn\G5)&= 0\\
 \Tr(\Gm\Gn\Gr\Gs\G5) &= -4\ii\epsilon^{\mu\nu\rho\sigma}\\
\intertext{Generally, for some $\gamma$-matrices $A,B,C,\dots$,}\quad
\Tr(ABCDEF\cdots) &=
\Minkow{AB}\Tr(CDEF\cdots) - \Minkow{AC}\Tr(BDEF\cdots)\notag\\
& + \Minkow{AD}\Tr(BCEF\cdots) - \Minkow{AE}\Tr(BCDF\cdots) + \cdots,\\[1.5zw]
\Tr(ABCDEF\cdots\G5) &= \text{\emp{Not Established.}}
\end{align}
To prove the second equation, we use following technique:
\begin{equation}
 \Tr(\Gm\Gn\Gr\Gs\cdots)=\Tr(\cdots\Gs\Gr\Gn\Gm);\qquad
 \Tr(\Gm\Gn\Gr\Gs\cdots\G5)=\Tr(\G5\cdots\Gs\Gr\Gn\Gm).
\end{equation}

\subsubsection{Contractions}\vspace{-2.5zw}
\begin{align}
 \Gm\Gmd&=4\\
 \Gm\Gn\Gmd&=-2\Gn\\
 \Gm\Gn\Gr\Gmd&=4\Hnr\\
 \Gm\Gn\Gr\Gs\Gmd&=-2\Gs\Gr\Gn
\end{align}
Generally, for some $\gamma$-matrices $A,B,C,\dots$,
\begin{align}
\text{ODD \# :\quad}&\Gm ABC\cdots \Gmd = -2(\cdots CBA),\\
\text{EVEN \# :\quad}&\Gm ABC\cdots \Gmd = \Tr(ABC\cdots)-\Tr(ABC\cdots\G5)\cdot\G5.
\end{align}

\subparagraph{Contractions in $d$-dimension}
\begin{align}
 \Gm\Gmd&=d\\
 \Gm\Gn\Gmd&=-(d-2)\Gn\\
 \Gm\Gn\Gr\Gmd&=4\Hnr-(4-d)\Gn\Gr\\
 \Gm\Gn\Gr\Gs\Gmd&=-2\Gs\Gr\Gn + (4-d)\Gn\Gr\Gs
\end{align}


\subparagraph{Contractions of $\epsilon$'s}
\begin{align}
& \epsilon^{\alpha\beta\gamma\delta}\epsilon_{\alpha\beta\gamma\delta}=-24;\quad
 \epsilon^{\alpha\beta\gamma\mu}\epsilon_{\alpha\beta\gamma\nu}=-6\delta^\mu_\nu;\quad
 \epsilon^{\alpha\beta\mu\nu}\epsilon_{\alpha\beta\rho\sigma}=-2(
\delta^\mu_\rho\delta^\nu_\sigma-\delta^\mu_\sigma\delta^\nu_\rho
)\\
&\epsilon^{\mu\alpha\beta\gamma}\epsilon_{\mu\alpha'\beta'\gamma'}=-
\left
(\delta^\alpha_{\alpha'}\delta^\beta_{\beta'}\delta^\gamma_{\gamma'}
+\delta^\alpha_{\beta'}\delta^\beta_{\gamma'}\delta^\gamma_{\alpha'}
+\delta^\alpha_{\gamma'}\delta^\beta_{\alpha'}\delta^\gamma_{\beta'}
-\delta^\alpha_{\alpha'}\delta^\beta_{\gamma'}\delta^\gamma_{\beta'}
-\delta^\alpha_{\beta'}\delta^\beta_{\alpha'}\delta^\gamma_{\gamma'}
-\delta^\alpha_{\gamma'}\delta^\beta_{\beta'}\delta^\gamma_{\alpha'}
\right)
\end{align}

\subsection{Miscellaneous Techniques}
$(p\cdot\sigma)(p\cdot\bar\sigma)=p^2$


\subsubsection{Dirac Field Techniques}
\begin{tabular}{l@{ :\ \ \ }l}
Dirac Equations & $(\slashed p-m)u^s(p)=0;\quad(\slashed p+m)v^s(p)=0$\\
&$\bar u^s(p)(\slashed p-m)=0;\quad \bar v^s(p)(\slashed p+m)=0$\\
Dirac Components &
   $u^{r\dagger}(p) u^s(p)=2E_\vc p\delta^{rs};\quad
    v^{r\dagger}(p) v^s(p)=2E_\vc p\delta^{rs}$\\
 & $\bar u^r(p)u^s(p)=2m\delta^{rs};\quad
    \bar v^r(p)v^s(p)=-2m\delta^{rs};\quad
    \bar u^r(p)v^s(p)=\bar v^r(p)u^s(p)=0$\\
Spin Sums & $\displaystyle
             \sum\s{spin}u^s(p)\bar u^s(p)=\slashed p+m;\quad
             \sum\s{spin}v^s(p)\bar v^s(p)=\slashed p-m$\\
\end{tabular}

\subsubsection{Polarization Sum}

\subparagraph{Single photon case}$M=\epsilon_\mu^*(k)M^{\mu}$\par
When Ward identity $k_\mu M^\mu=0$ is valid,
\begin{equation}
   \sum\s{pol.}\left|M\right|^2
= \sum\s{pol.}\epsilon_\mu^*(k)\epsilon_\nu(k)M^{\mu}M^{\nu*}
=             \Hmnd M^{\mu}M^{\nu*}.
\end{equation}

\subparagraph{Double photons case}$M=\epsilon_\mu^*(k)\epsilon_\nu'^*(k')M^{\mu\nu}$\par
When $k_\mu M^{\mu\nu}=k'_\nu M^{\mu\nu}=0$ is valid,
\begin{equation}
  \sum\s{pol.}\left|M\right|^2
= \sum\s{pol.}
\epsilon_\mu^*(k)\epsilon_\rho(k)\epsilon_\nu'^*(k')\epsilon_\sigma'(k')
M^{\mu\nu}M^{\rho\sigma*}
= \Hmrd\Hnsd M^{\mu\nu}M^{\rho\sigma*}.
\end{equation}
\note{See Sec.~\ref{Sec:Verbose:PolarizationSum} for verbose information.}


\subsubsection{Fierz identities}
For Dirac spinors $a,b,c,d$ and their left-handed projections $a\s L:=
P\s La$ etc.,
\begin{equation}
  (\bar a\s L\Gm b\s L)(\bar c\s L\Gmd d\s L)=
 -(\bar a\s L\Gm d\s L)(\bar c\s L\Gmd b\s L)
\end{equation}
Here we can create another equations using
\begin{equation}
 (\Sm)_{\alpha\beta}(\Smd)_{\gamma\delta} =  2\epsilon_{\alpha\gamma}\epsilon_{\beta\delta};
\qquad
 (\bSm)_{\alpha\beta}(\bSmd)_{\gamma\delta} =  2\epsilon_{\alpha\gamma}\epsilon_{\beta\delta}.
\end{equation}

\subsubsection{Gordon identity}
For  $P:=p'+p$ and $q:=p'-p$,
\begin{align}
 \bar u(p')\Gm u(p) &=  \bar u(p') \left[\frac{P^\mu + \ii\Smn q_\nu}{2m}\right] u(p)&
 \bar u(p')\Gm v(p) &=  \bar u(p') \left[\frac{q^\mu + \ii\Smn P_\nu}{2m}\right] v(p)\\
 \bar v(p')\Gm v(p) &= -\bar v(p') \left[\frac{P^\mu + \ii\Smn q_\nu}{2m}\right] v(p)&
 \bar v(p')\Gm u(p) &= -\bar v(p') \left[\frac{q^\mu + \ii\Smn P_\nu}{2m}\right] u(p)
\end{align}

\subsubsection{Gauge group algebra}
For a gauge group $G$ s.t.
\begin{equation}
 [t^a,t^b]=\ii f^{abc}t^c,
\end{equation}
we have two constants which {\bf depend on representation $r$}.
\begin{align}
  \Tr(t^at^b) =: C(r)\delta^{ab};\qquad  \Tr(t^at^a) =: C_2(r)\cdot\vc 1\quad
 \text{\footnotesize(quadratic Casimir operator)}
\end{align}
They satisfy
\begin{align}
 C(r) &= \dfrac{d(r)}{d(\text{Adj.})}C_2(r),&
 t^at^bt^a&=\left[C_2(r)-\frac12 C_2(\text{Adj.})\right]t^b.\\
 f^{acd}f^{bcd} &= C_2(\text{Adj.})\delta^{ab},&
 f^{abc}t^bt^c &= \frac12\ii C_2(\text{Adj.})t^a.
\end{align}

\subparagraph{For $\gSU(N)$}
For $\gSU(N)$ groups and its fundamental representation $N$, we have
\begin{align*}
  C(N)&=\frac12, & C_2(N)&= \frac{N^2-1}{2N}, & C(\text{Adj.}) =
 C_2(\text{Adj.}) &= N;&
  (t^a)_{ij}(t^a)_{kj} &= \frac12\left(\delta_{il}\delta_{kj}-\frac{\delta_{ij}\delta_{kl}}{N}\right).
\end{align*}

\newpage
\subsection{Loop Integrals and Dimensional Regularization}
\subsubsection{Feynman Parameters}\vspace{-2.5zw}
\begin{align}
 \frac{1}{A_1A_2\cdots A_n}&=
\int_0^1\dd x_1\cdots x_n\ \delta\left(\sum x_i-1\right)
\frac{(n-1)!}{[x_1A_1+x_2A_2+\cdots+x_nA_n]^n}\\
 \frac{1}{A_1A_2}&=
\int_0^1\dd x\frac{1}{[xA_1+(1-x)A_2]^2}
\end{align}
\subsubsection{$d$-dimensional integrals in Minkowski space}\vspace{-2.5zw}
\begin{align}
 \intP dl\frac1{(l^2-\Delta)^n}&=
\frac{(-1)^n\ii}{(4\pi)^{d/2}}\frac{\Gamma(n-\frac d2)}{\Gamma(n)}
\left(\frac1\Delta\right)^{n-\frac d2}
\\
 \intP dl\frac{l^2}{(l^2-\Delta)^n}&=
\frac{(-1)^{n-1}\ii}{(4\pi)^{d/2}}\frac d2\frac{\Gamma(n-\frac d2-1)}{\Gamma(n)}
\left(\frac1\Delta\right)^{n-\frac d2-1}
\\
 \intP dl\frac{l^\mu l^\nu}{(l^2-\Delta)^n}&=
\frac{(-1)^{n-1}\ii}{(4\pi)^{d/2}}\frac {\Hmn}2\frac{\Gamma(n-\frac d2-1)}{\Gamma(n)}
\left(\frac1\Delta\right)^{n-\frac d2-1}
\\
 \intP dl\frac{(l^2)^2}{(l^2-\Delta)^n}&=
\frac{(-1)^n\ii}{(4\pi)^{d/2}}\frac{d(d+2)}4\frac{\Gamma(n-\frac d2-2)}{\Gamma(n)}
\left(\frac1\Delta\right)^{n-\frac d2-2}
\\
 \intP dl\frac{l^\mu l^\nu l^\rho l^\sigma}{(l^2-\Delta)^n}&=
\frac{(-1)^n\ii}{(4\pi)^{d/2}}\frac{\Gamma(n-\frac d2-2)}{\Gamma(n)}
\left(\frac1\Delta\right)^{n-\frac d2-2}
\frac{\Hmn\Hrs+\Hmr\Hns+\Hms\Hnr}4
\end{align}

Here we can use following expansions: \qquad $(\gamma\simeq 0.5772)$
\begin{align}
 \left(\frac{1}{\Delta}\right)^{2-\frac d2} &=
 1 - (d-4)\frac{\log\Delta}{2} + \Order\left((d-4)^2\right)\quad\text{around $d=4$},
\\
 \Gamma(x)&=
 \frac{1}{x} - \gamma + \Order(x)\quad\text{around $x=0$},
\\
 \Gamma(x)&=
 \frac{(-1)^n}{n!}\left[ \frac{1}{x+n} - \gamma +
 \sum_{k=1}^{n}\frac{1}{k} + \Order(x+n)\right]
\quad\text{around $x=-n$}.
\end{align}
and we get following expansion:
\begin{equation}
 \frac{\Gamma(2-\frac  d2)}{(4\pi)^{d/2}}\left(\frac1\Delta\right)^{2-\frac d2}
= \frac1{(4\pi)^2}\left[\left(\frac{2}{4-d} - \gamma + \log4\pi \right)
                   - \log \Delta + \Order(4-d) \right].
\end{equation}
Usually this $\Delta$ is positive, but when $\Delta$ contains some
timelike momenta, it becomes negative. Then these integrals acquire
imaginary parts, which give the discontinuities of $S$-matrix elements.
To compute the $S$-matrix in a physical region choose the correct branch
\begin{equation}
 \left(\frac1\Delta\right)^{n-\frac d2}\to
 \left(\frac1{\Delta-\ii\epsilon}\right)^{n-\frac d2}.
\end{equation}


\newpage

\subsection{Cross Sections and Decay Rates}
\paragraph{General expression}
\begin{align}
 \dd\sigma &=
\frac{1}{2E_A2E_B|v_A-v_B|}\Biggl[\prod_f\frac{\dd^3p_f}{(2\pi)^3}\frac{1}{2E_f}\Biggr]
\Bigl|\mathcal M(p_A,p_B\to\{p_f\})\Bigr|^2(2\pi)^4\delta^{(4)}\left(p_A+p_B-\{p_f\}\right)\\
 \dd\Gamma &=
\frac{1}{2m_A}\Biggl[\prod_f\frac{\dd^3p_f}{(2\pi)^3}\frac{1}{2E_f}\Biggr]
\Bigl|\mathcal M(m_A\to\{p_f\})\Bigr|^2(2\pi)^4\delta^{(4)}\left(m_A-\{p_f\}\right)
\quad \text{(in $A$-rest frame.)}
\end{align}
\paragraph{2-body phase space in center-of-mass frame}
\begin{align}
 \int\Pi_2&:=
\intP3{p_1}\intP3{p_2}\frac{1}{2E_1}\frac{1}{2E_2}
(2\pi)^4\delta^{(4)}\left(E\s{cm}-(p_1+p_2)\right)\qquad\text{(in
 center-of-mass frame)}\\
&= \int\frac{\dd\Omega}{4\pi}\frac1{8\pi}\frac{2\vnorm{p_1}}{E\s{cm}}\\
&= \frac{1}{8\pi}
   \sqrt{1-\frac{2({m_1}^2+{m_2}^2)}{E\s{cm}^2}+\frac{({m_1}^2-{m_2}^2)^2}{E\s{cm}^4}}
\quad \too[m_2=0]\ \frac{1}{8\pi}\left(1-\frac{{m_1}^2}{{E\s{cm}}^2}\right)
\end{align}

\begin{itemize}
 \item Fierz Transf.
 \item Noether current
 \item Majorana Ferminos
 \item Feynman Rules(A.1)
\end{itemize}

%%% Local Variables:
%%% TeX-master: CheatSheet.tex
%%% End:
