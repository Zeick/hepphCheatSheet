\documentclass[CheatSheet]{subfiles}

\begin{document}
\summarystyle



\section[Values]{Values\footnotemark}

\footnotetext{Data source: \BLUE{\textbf{PDG2020}}. \bounddescription}


{%%%%%%%%%CATCODE MODIFICATION GUARD
\catcode`#=\active \def#{\thinspace}
\subparagraph{Mass and width}
\begin{alignat*}{5}
 e     &: \BLUE{0.510#998#9461(31)}\MeV &&\hspace{14em}
 &&m_{\nu;\text{tot}}<\BLUE{0.2\text{--}0.3}\eV \quad&
\\
 \mu   &: \BLUE{105.658#3745(24)}\MeV, && \BLUE{2.196#9811(22)}\unit{\mu s}=\BLUE{659}\unit{m}
 &&h: \BLUE{125.10(14)}\GeV
\\
 \tau  &: \BLUE{1.776#86(12)}\GeV,     && \BLUE{290.3(5)\EE{-15}}\unit{s}=\BLUE{87.0}\unit{\mu m}
 &&W : \BLUE{80.379(12)}\GeV,  &&\BLUE{2.085(42)}\GeV
\\
 t     &: \BLUE{172.76(30)}\GeV\footnotemark, && \BLUE{1.42\unc{+0.19}{-0.15}}\GeV
 &&Z     : \BLUE{91.1876(21)}\GeV,  &&\BLUE{2.4952(23)}\GeV
\end{alignat*}
\footnotetext{Cross section measurement gives $\MSbar$ top mass $\BLUE{162.5\unc{+2.1}{-1.5}}\GeV$, equivalent to $\BLUE{172.4(7)}\GeV$.}

\vspace{-3em}

\begin{align*}
 (u,d,s)^{\MSbar}_{2\GeV} &: (\BLUE{2.16\unc{+0.49}{-0.26}}, \BLUE{4.67\unc{+0.48}{-0.17}}, \BLUE{93\unc{+11}{-\ 5}})\MeV\footnotemark
&
c&:\BLUE{1.27(2)}\GeV^{\MSbar}_{m_c}\quad(\BLUE{1.67(7)}\GeV^{\text{pole}})
\\
 (\tfrac{u+d}{2},\tfrac{u}{d},\tfrac{2s}{u+d})^{\MSbar}_{2\GeV} &: (\BLUE{3.45\unc{+0.55}{-0.15}}\MeV, \BLUE{0.47\unc{+0.06}{-0.07}},\BLUE{27.3\unc{+0.7}{-1.3}})
&
b&:\BLUE{4.18\unc{+0.03}{-0.02}}\GeV^{\MSbar}_{m_b}\quad(\BLUE{4.78(6)}\GeV^{\text{pole}})
\end{align*}
\footnotetext{$m^{\MSbar}_{1\GeV}=m^{\MSbar}_{2\GeV}\times1.35$.}

\vspace{-3em}

\begin{align*}
 \pi^\pm  &: \BLUE{139.570#39(18)}\MeV&
 \rho_{770}^\pm &: \BLUE{775.11(34)}\MeV&
 \eta_c(1S)   &: \BLUE{2983.9(5)}\MeV
\\
 \pi^0    &: \BLUE{134.9768(5)}\MeV&
 \rho_{770}^0   &: \BLUE{775.26(25)}\MeV&
 J/\psi(1S)   &: \BLUE{3096.900(6)}\MeV
\\
 \eta     &: \BLUE{547.862(17)}\MeV&
 \phi_{1020}     &: \BLUE{1019.461(16)}\MeV&
 \eta_b(1S)   &: \BLUE{9398.7(20)}\MeV
\\
 \eta'    &: \BLUE{957.78(6)}\MeV&
 \omega_{782}   &: \BLUE{782.65(12)}\MeV&
 \Upsilon(1S)   &: \BLUE{9460.30(26)}\MeV
\\
 K^\pm    &: \BLUE{493.677(16)}\MeV&
 K^{*\pm}_{892} &: \BLUE{891.66(26)}\MeV&
 \Upsilon(2S)   &: \BLUE{10023.26(31)}\MeV
\\
 K^0      &: \BLUE{497.611(13)}\MeV&
 K^{*0}_{892} &: \BLUE{895.55(20)}\MeV&
 \Upsilon(3S)   &: \BLUE{10355.2(5)}\MeV
\\
 D^0      &: \BLUE{1864.83(5)}\MeV&
 B^\pm    &: \BLUE{5279.34(12)}\MeV&
 \Upsilon(4S)   &: \BLUE{10579.4(12)}\MeV
\\
 D^\pm    &: \BLUE{1869.65(5)}\MeV&
 B^0      &: \BLUE{5279.65(12)}\MeV&
p      &:\BLUE{938.272#0813(58)}\MeV&
\\
 D_s^\pm  &: \BLUE{1968.34(7)}\MeV&
 B_s      &: \BLUE{5366.88(14)}\MeV&
 n      &:\BLUE{939.565#413(6)}\MeV
\\
&& B_c^\pm  &: \BLUE{6274.9(8)}\MeV
\end{align*}

\vspace{-2em}

\begin{align*}
 \pi^\pm  &: \BLUE{2.6033(5)\EE-8}\unit{s} = \BLUE{7.80}\unit{m} &
 K^\pm    &: \BLUE{1.2380(20)\EE-8}\unit{s} = \BLUE{3.71}\unit{m} &
 p      & \bound{90}>\BLUE{3.6\EE{29}}\unit{yr}\\
 \pi^0    &: \BLUE{8.52(18)\EE{-17}}\unit{s} = \BLUE{0.0255}\unit{\mu m}&
 K^0\w S  &: \BLUE{0.8954(4)\EE{-10}}\unit{s} = \BLUE{26.8}\unit{mm}&
 n       &: \BLUE{879.4(6)}\unit{s}
\\
&&
 K^0\w L  &: \BLUE{5.116(21)\EE{-8}}\unit{s} = \BLUE{15.3}\unit{m}
\end{align*}

\vspace{-1em}


\subparagraph{Electric and magnetic moment, important branching ratios, and neutrino property}
\begin{align*}
   &a_e = \BLUE{11#596#521.8091(26)\EE{-10}}&
   &\Br(\tau\to e,\mu)\simeq\BLUE{35.2}\%&
   &\Delta m_{21}^2/\text{eV}^2 = \BLUE{7.53(18)\EE-5}
\\
   &a_\mu = \BLUE{11#659#208.9(54)(33)\EE{-10}}&
   &\Br(\tau\to \text{had})\simeq\BLUE{64.8}\%&
   &\Delta m_{32}^2/\text{eV}^2 =
\\[-.3em]
   &a_\tau\bound{95}{\in}\BLUE{[-0.052,0.013]}&
   &\Br(\tau;\text{1-prong})=\BLUE{85.24(6)}\%&
   &\quad\BLUE{2.453(34)\EE-3} \text{(NH)}
\\
   &\mu_p = \BLUE{2.792#847#3446(8)\mu_N}&
   &\Br(\tau;\text{3-prong})=\BLUE{14.55(6)}\%&
   &\quad\BLUE{-2.546\unc{+0.034}{-0.040}\EE-3}\text{(IH)}
\\
   &\mu_n = \BLUE{-1.913#0427(5)\mu_N}&
   &\Br(Z\to \text{had})=\BLUE{69.911(56)}\%&
   &\sin^2\theta_{12} = \BLUE{0.307(13)}
\\[-.3em]
   &d_e \bound{90}< \BLUE{0.11\EE{-28}e\unit{cm}}&
   &\quad\Br(Z\to b\bar b)=\BLUE{15.12(5)}\%&
   &\sin^2\theta_{13}=\BLUE{0.0218(7)}
\\[-.3em]
   &d_\mu \bound{95}< \BLUE{1.8\EE{-19}e\unit{cm}}&
   &\Br(Z\to e,\mu,\tau)  \simeq  \BLUE{10.10}\%&
   &\sin^2\theta_{23}\text{(NH)} = \BLUE{0.545(21)}
\\[-.3em]
   &d_p \stackrel{?}<\BLUE{0.021\EE{-23}e\unit{cm}}&
   &\Br(Z\to \text{inv})=\BLUE{20.000(55)}\%&
   &\sin^2\theta_{23}\text{(IH)} = \BLUE{0.547(21)}
\\[-.3em]
   &d_n \bound{90}<\BLUE{0.18\EE{-25}e\unit{cm}}&
   &\Br(W\to \text{had})=\BLUE{67.41(27)}\%&
   &\delta=\BLUE{1.36(17)\pi}
\end{align*}

\subparagraph{CKM matrix}
\begin{align*}
&  V\w{CKM}=\pmat{V_{ud}&V_{us}&V_{ub}\\V_{cd}&V_{cs}&V_{cb}\\V_{td}&V_{ts}&V_{tb}} =
\BLUE{\pmat{
  0.97401(11) & 0.22650(48) & 0.00361\unc{(+11)}{(-\phantom09)} \\
  0.22636(48) & 0.97320(11) & 0.04053\unc{(+83)}{(-61)}\\
  0.00854\unc{(+23)}{(-16)} & 0.03978\unc{(+82)}{(-60)} & 0.0999172\unc{(+24)}{(-35)}
}};~~
J=\BLUE{3.00\unc{(+15)}{(-\phantom09)}\EE-5}\\
 &(\lambda,A,\bar\rho,\bar\eta)=\BLUE{(0.22650(48), 0.790\unc{(+17)}{(-12)}, 0.141\unc{(+16)}{(-17)}, 0.357(11))}\\
 &(\sin\theta_{12},\sin\theta_{13},\sin\theta_{23},\delta)=\BLUE{(0.22650(48), 0.00361\unc{(+11)}{(-\phantom09)}, 0.04053\unc{(+83)}{(-61)}, 1.196\unc{(+45)}{(-43)})}
\end{align*}

\newpage

\subparagraph{Astrophysical}
\begin{align*}
   &T_0=\BLUE{2.7255(6)}\unit{K}&
   &H_0=100h\unit{km/s/Mpc},~h=\BLUE{0.674(5)}&
   &M_{\Sun}=\BLUE{1.98841(4)\EE{30}}\unit{kg}\\
   &n_\gamma = \BLUE{410.7(3)}\hat T_0^3\unit{cm^{-3}}&
   &\rho\w{crit}=\BLUE{1.053#672(24)\EE-5}h^2\unit{GeV/cm^3}&
   &M_{\Earth}=\BLUE{5.97217(13)\EE{24}}\unit{kg}\\
   &\rho_\gamma=\BLUE{0.2606(2)}\hat T_0^4\unit{eV/cm^3}&
   &G\w{N}=\BLUE{6.708#83(15)\EE{-39}}\unit{GeV^{-2}}&
   &R_0=\BLUE{8.178(13)(22)}\unit{kpc}\\
   &s=\BLUE{2891.2}\hat T_0^3\unit{cm^{-3}}&
   &M\w{Pl}=\BLUE{1.220#890(14)\EE{19}}\GeV&
   &v_0=\BLUE{240(8)}\unit{km/s}\\
   &\Omega_{\gamma}h^2=\BLUE{2.473\EE-5}\hat T_0^4&
   &M_0    =\BLUE{2.435#323(28)\EE{18}}\GeV&
   &\rho\w{disk}=\BLUE{3.7(5)}\unit{GeV/cm^3}\\
   &\quad\text{\small{$[\hat T_0=T_0/\BLUE{2.7255}\unit{K}]$}}&
   &\eta=n\w{b}/n_\gamma\bound{95}{\in}\BLUE{[5.8,6.5]\EE{-10}}&
\end{align*}
Planck 2018 6-parameter fit to flat $\Lambda$CDM cosmology:
\begin{align*}
 &\{\Omega\w{b} h^2, \Omega\w{CDM} h^2\} = \{\BLUE{0.02237(15)},\BLUE{0.1200(12)}\}&
 &(z,t)\w{M=R}=\BLUE{3402(26),5.11(8)\EE4}\unit{yr}\\
 &\Omega\w{\{b,CDM,\Lambda\}}   = \{\BLUE{0.0493(6)}, \BLUE{0.265(7)}, \BLUE{0.685(7)}\}&
 &(z,t)\w{*}=\BLUE{1089.92(25),3.729(10)\EE5}\unit{yr}\\
 &\Lambda = \BLUE{1.088(30)\EE{-56}}\unit{cm^{-2}}&
 &(z,t)\w{i}=\BLUE{7.7(7),6.90(90)\EE{8}}\unit{yr}\\
 &\Omega_K=\BLUE{0.0007(19)}&
 &(z,t)\w{q}=\BLUE{0.636(18),7.70(10)\EE9}\unit{yr}\\
 &N\w{eff}=\BLUE{2.99(17)}&
 &t_0=\BLUE{1.3797(23)\EE{10}}\unit{yr}\\
\end{align*}
\subparagraph{Standard Model parameter fit}
\begin{align*}
   &\alpha^{-1}\w{EM}(0) = \BLUE{137.035#999#084(21)}&
   &\sin^2\theta^{\MSbar}(M_Z)=\BLUE{0.23121(4)}&
   & \alpha\w{s}(m_Z) = \BLUE{0.1179(10)}\\
   &\hat\alpha^{(4)}(m_\tau)^{-1}=\BLUE{133.472(7)}&
   &\sin^2\theta^{\MSbar}(0)=\BLUE{0.23857(5)}&
   & G\w{F} = \BLUE{1.166#378#7(6)\EE{-5}}\unit{GeV^{-2}}\\
   &\hat\alpha^{(5)}(m_Z)^{-1}=\BLUE{127.952(9)}&
   &\sin^2\theta^{\text{on-shell}}=\BLUE{0.22337(10)}&
   &\phantom{G\w{F}}\stackrel{\text{tree}}=g_2^2/(4\sqrt2 m_W^2)=1/(\sqrt2 v^2)
\\
   &\Delta\alpha^{(5)}\w{had}(m_Z) = \BLUE{0.02766(7)}&
   &\quad\stackrel{\text{tree}}=(g'/g_Z)^2=1-(m_W/m_Z)^2
\end{align*}

$\MSbar$parameters at $Q_0=173.1\GeV$ based on Ref.~\cite{Huang:2020hdv} (cf.~Ref.~\cite{Martin:2019lqd}):
\begin{align*}
 g_s&=1.161#8(4#5)&
 v  &=246.605(12)\GeV\qquad\qquad\qquad  \lambda=0.126#07(30)\\
 g  &=0.647#653(281)&
 -m^2&=8612.0(22.8)\GeV^2=(92.80(12)\GeV)^2\\
 g' &=0.358#542(70)&
 y_{t,c,u}&=\{0.931(4),0.0341(10),6.8(1.1)\EE{-6}\}\\
 |e|&=0.313#68(18)&
 y_{b,s,d}&=\{0.015#53(14),0.000#293(25),1.47(10)\EE{-5}\}\\
 g_Z&=0.740#27(25)&
 y_{\tau,\mu,e}&=\{0.009#994#4(8),0.000#588#38(11),2.793#0(2#6)\EE-6\}\\
\end{align*}

} %%%%%%%%%%%%%%% CATCODE MODIFICATION GUARD END

\newpage

\detailstyle
\begin{tabular}[t]{c@{}c|ccccccccccc}\toprule
$n\,^{2s+1}l_J$ & $J^{PC}$ & $I=1$ & $I=1/2$ & \multicolumn{2}{c}{$I=0$} &   $c\bar c$ & $b\bar b$ & \multicolumn{2}{c}{charm} & \multicolumn{3}{c}{bottom}\\\midrule
$1\,^{1}S_0$    & $0^{-+}$ & $\pi$ & $K$ & $\eta$ & $\eta'_{958}$ & $\eta_c(1S)$ & $\eta_b(1S)$ & $D$ & $D_s$ & $B$ & $B_s$ & $B_c$\\
$1\,^{1}S_1$    & $1^{--}$ & $\rho_{770}$ & $K^*_{892}$ & $\phi_{1020}$ & $\omega_{782}$ & $J/\psi(1S)$ & $\Upsilon(1S)$ & $D^*$ & $D^*_s$ & $B^*$ & $B^*_s$ & \\
\bottomrule
\end{tabular}

\TODO{define PMNS matrix and explain neutrino hierarchy.}







%%% Local Variables:
%%% TeX-master: "CheatSheet.tex"
%%% End:
\end{document}
