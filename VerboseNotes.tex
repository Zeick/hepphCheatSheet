%#!latexmk CheatSheet.tex
%%% Time-Stamp: <2010-03-25 01:33:22 misho>
\section{Verbose Notes}
\subsection{Spinor Fields}
\subsubsection{Lorentz group and Lorentz algebra}
\begin{tabular}{l@{ :\ \ \ }l}
Metric &  $\eta^\% = \diag(+1, -1, -1, -1),\qquad \eta^\# = \diag(-1,+1,+1,+1).$\\
Lorentz transf. in $\mathbb R^{1,3}$ & Linear transf. $x^\mu \mapsto \Lambda\T^\mu_\nu x^\nu$ which conserve $x^2$.\\
 & \then $\Hrsd = \Hmnd \Lambda\T^\mu_\rho \Lambda\T^\nu_\sigma.$
 and form a group $L$.\\[.8zw]

Disconnected parts of $L$ &
  $L_0 := \left\{ \det\Lambda = +1 \land \Lambda\T_0^0 >0\right\} \quad
   L_P := \left\{ \det\Lambda = -1 \land \Lambda\T_0^0 >0\right\}$\\
& $L_T := \left\{ \det\Lambda = +1 \land \Lambda\T_0^0 <0\right\} \quad
   L_{PT} := \left\{ \det\Lambda = -1 \land \Lambda\T_0^0 <0\right\}$\\
& \qquad ($L_0$ is identical with $\%\gSO(1,3) / \#\gSO(3,1)$. )\\[1zw]

Infinitesimal one in $L_0$& $\Lambda\T^\mu_\nu = \delta^\mu_\nu + \epsilon\T^\mu_\nu$ where $\epsilon_{\mu\nu}=-\epsilon_{\nu\mu}$ (for $\eta=\eta\Lambda\Lambda$)\\[1zw]

Generators &
 $x'^\mu =  x^\mu + \epsilon\T^\mu_\nu x^\nu
   =: \left(\delta^\mu_\nu\mp\frac\ii2\epsilon_{\rho\sigma}(J^{\rho\sigma})\T^\mu_\nu\right)x^\nu$
   \quad with $J^{\rho\sigma} = -J^{\sigma\rho}$\\
&\then 
$ (J_{\rho\sigma})\T^\mu_\nu = \mp\ii\left(
                                     \delta^\mu_\rho\Hsnd-\delta^\mu_\sigma\Hrnd
                                    \right),$\\
Lorentz algebra & therefore
 $[J_{\mu\nu},J_{\rho\sigma}] = \pm\ii(
    \Hmrd J_{\nu\sigma} + \Hnsd J_{\mu\rho} - \Hmsd J_{\nu\rho} - \Hnrd J_{\mu\sigma})$\\
\end{tabular}

\vspace{1zw}
Here we physicists derive the algebra from the property of the Lorentz transformation, but we can start from the albegra like mathematicians.

\TODO{J�̕����m�F}

\subsubsection{Lorentz group and $\gSL(2,\Complex)$}
Here we will explain that $L_0$ is isomorphic to $\gSL(2,\Complex)/\gZ_2$, which is defined as
\begin{align}
 \aSL(2,\Complex) :=& \{a\in \aGL(2,\Complex)\ |\ \Tr(a)=0\},&
 \gSL(2,\Complex) :=& \{g\in \gGL(2,\Complex) |\ \det(g)=1\}.
\end{align}

We can build a one-to-one correspondent between $x^\mu$ and an Hermitian matrix
\begin{equation}
 A := t\cdot1+\vipro x \sigma = \pmat{t+z&x-\ii y\\x+\ii y&t-z}.
\end{equation}
Then, we can define Lorentz transformations $f^g$ from $g\in\gSL(2,\Complex)$ as
\begin{equation}
 f^g: A\mapsto g^\dagger Ag,
\end{equation}
and thus we can induce Lorentz transformations $\Lambda(g)$ from $g$.

Actually, $\Lambda(g)$ covers only $L_0$, not whole $L$. I do not provide the proof, but since $\gSL(2,\Complex)$ is a connected group,
it is clear that it can cover only the connected part of Lorentz group.
\begin{rightnote}
I omit the proof here, but in principle, it can be done by direct calculation. For instance, $\Lambda\T^0_0>0$ can be proved as follows:

Since $t = \frac12\Tr(t\cdot1+\vipro x\sigma)$,
$\Lambda\T^0_0$ is nothing but
$\frac1{2t}\Tr\left[g^\dagger\left({t\atop0}{0\atop t}\right)g\right]$.
Therefore
\begin{equation}
 \Lambda\T^0_0
 = \frac12\Tr\left(g^\dagger g\right)
 = \frac12\left(|g_{11}|+|g_{12}|+|g_{21}|+|g_{22}|\right)>0.
\end{equation}
This shows $\Lambda(g)$ are orthochronous.
\end{rightnote}


As an element of $\aSU(2)$ can be expressed by the 3 bases known as Pauli matrices, 

The Lorentz group has 6 generators $J_{\rho\sigma}$, and these corresponds to
the 6 elements of $\aSL(2,\Complex)$.
These can be broken down into three ``rotation generators'' $\vc J$, and three ``boost generators'' $\vc K$ as follows:
\begin{align}
 J_i &:= (J_{23}, J_{31}, J_{12}) &  K_i &:= (J_{10},J_{20},J_{30}) &\for i=1,2,3.\footnotemark
\end{align}
\footnotetext{For your information, $J_{23} = J^{23}$ and $K_{01} = -K^{01}$, regardless of the definition of the metic.}
With this definition, the algebra is, regardless of signature of the metric, as follows:
\begin{align}
 [J_i, J_j]&= \ii\epsilon_{ijk}J_k,&
 [J_i, K_j]&= \ii\epsilon_{ijk}K_k,&
 [K_i, K_j]&= -\ii\epsilon_{ijk}J_k.
\end{align}

Then, we define $\vc A$ and $\vc B$ as
\begin{align}
 \vc A&:=\frac12(\vc J+\ii\vc K), & \vc B &:= \frac12(\vc J-\ii\vc K).
\end{align}
Now the commutation relations are
\begin{align}
 [A_i, A_j] &= \ii\epsilon_{ijk}A_k,&
 [B_i, B_j] &= \ii\epsilon_{ijk}B_k,&
 [A_i, B_j] &= 0,
\end{align}
which means we have reduced the Lorentz group into $\gSU(2)\times\gSU(2)$.
\starline

In some neighborhood of the identity, the elements $\Lambda$ of Lorentz group can be described as
\begin{equation}
 \Lambda = \exp(\mp\frac\ii2\epsilon_{\rho\sigma}J^{\rho\sigma}),
\end{equation}
using the (antisymmetric) generators $J^{\rho\sigma}$.
Here, defining $\vc \theta$ and $\vc \beta$ as
\begin{align}
 \pm\theta_i&:= (\epsilon_{23},\epsilon_{31},\epsilon_{12})&
 \pm\beta_i &:= (\epsilon_{10},\epsilon_{20},\epsilon_{30})&
 \for i=1,2,3,
\end{align}
the element is denoted as:
\begin{equation}
 \Lambda = \exp\left[\ii\left(\vipro\theta J + \vipro \omega K\right)\right].
\end{equation}

\subsection{Weyl Spinor}
Now we introduce Weyl spinors. First, I de
%%% Local Variables:
%%% TeX-master: t
%%% End:


\subsection{Polarization Sum}\label{Sec:Verbose:PolarizationSum}
Firstly we focus on the single photon case $M=\epsilon_\mu^*(k)\epsilon_\nu'^*(k')M^{\mu\nu}$.
Here we set $k=(E,0,0,E)$, and $\epsilon=(0,1,0,0)\oplus(0,0,1,0)$. Then
\begin{equation}
  \sum\s{pol.}\left|M\right|^2
= \sum\s{pol.}\epsilon_\mu^*(k)\epsilon_\nu(k)M^{\mu}M^{\nu*}
= |M^1|^2+|M^2|^2,
\end{equation}
while
\begin{equation}
 \Hmnd M^{\mu}M^{\nu*} = |M^1|^2+|M^2|^2
\end{equation}
for Ward identity $k_\mu M^\mu=0$. Therefore the replacement
\begin{equation}
 \sum\s{pol.}\epsilon_\mu\epsilon'_\nu \to \Hmnd
\end{equation}
is valid.

Secondly we think about the double photons case
\footnote{This part is derived from �_���K��'s notebook.}
$M=\epsilon_\mu^*(k)\epsilon_\nu'^*(k')M^{\mu\nu}$.
Here we set
\begin{align}
 k &=(E,0,0, E) & \epsilon &=(0,1,0,0)\oplus(0,0,1,0)\\
 k'&=(E,0,0,-E) & \epsilon'&=(0,\cos\theta,\sin\theta,0)\oplus(0,-\sin\theta,\cos\theta,0).
\end{align}
Then doing some simple calculations, we can get
\begin{equation}
  \sum\s{pol.}\left|M\right|^2 = |M^{11}|^2+|M^{12}|^2+|M^{21}|^2+|M^{22}|^2.
\end{equation}
Nevertheless, na\"ive replacement does not work, because our Ward identities
\begin{equation}
 k_\mu\epsilon_\nu'^*(k')M^{\mu\nu} =
 \epsilon_\mu^*(k)k'_\nu M^{\mu\nu} = 0 \label{eq:WardIdentities}
\end{equation}
obviously does not help us. If we can omit $\epsilon$s from these
identities, that is if
\begin{equation}
 k_\mu M^{\mu\nu} = k'_\nu M^{\mu\nu} = 0,\label{eq:TrueIdentities}
\end{equation}
we can recover validity of the replacement:
\begin{align}
 \Hmrd\Hnsd M^{\mu\nu}M^{\rho\sigma*}
&= -\Hnsd\left(
M^{1\nu}M^{1\sigma*}+M^{2\nu}M^{2\sigma*}
\right)\\
&= |M^{11}|^2+|M^{12}|^2+|M^{21}|^2+|M^{22}|^2.
\end{align}

Then what's happening? Why this replacement is not valid?
Actually our new conditions \eqref{eq:TrueIdentities} seem to guarantee
that we are summing not only ``physical'' but also ``unphysical'' polarizations.
Meanwhile if we use some physical condition such as $\epsilon\cdot k=0$,
\eqref{eq:TrueIdentities} break down while Ward identities
\eqref{eq:WardIdentities} are still valid.

Now let's check what is happening from another viewpoint. First we suppose
$M$ satisfies our new conditions \eqref{eq:TrueIdentities}, and define
$\tilde M^{\mu\nu}$ and $\tilde M$ as
\begin{align}
 \tilde M^{\mu\nu}& := M^{\mu\nu} + k^\mu p^\nu + p'^\mu k'^\nu,\\
 \tilde M         & := \epsilon_\mu^*(k)\epsilon_\nu'^*(k')\tilde M^{\mu\nu}.
\end{align}
This alternative amplitude satisfies Ward identities (since photon is
massless and $\epsilon\cdot k=0$), and furthermore $\tilde M = M$.
Therefore $\tilde M$ is physically identical to $M$.
However technically these are very different, just because we cannot
perform our ``na\"ive replacement'' for this $\tilde M$:
\begin{align}
 \Hmrd\Hnsd \tilde M^{\mu\nu}\tilde M^{\rho\sigma*}
 &=  \Hmrd\Hnsd
\left(M^{\mu\nu} + k^\mu p^\nu + p'^\mu k'^\nu \right)
\left(M^{\rho\sigma*} + k^{\rho} p^{\sigma*} + p'^{\rho*}k'^{\sigma} \right)\\
 &= \sum\s{pol.}|M|^2 + \left[(k\cdot p'^*)(k'\cdot p) + \Hc\right].
\end{align}
After all, we have obtained following expression:
\begin{align}
 \sum\s{pol.}|\tilde M|^2 &=
 \sum\s{pol.}|M|^2\qquad \text{(Furthermore $\tilde M=M$)}\notag\\
&= \sum\s{pol.}|\epsilon_\mu^*(k)\epsilon_\nu'^*(k')M^{\mu\nu}|^2
 = \sum\s{pol.}|\epsilon_\mu^*(k)\epsilon_\nu'^*(k')\tilde M^{\mu\nu}|^2\\
&= \Hmrd\Hnsd M^{\mu\nu}M^{\rho\sigma*}\notag\\
&\ne \Hmrd\Hnsd \tilde M^{\mu\nu}\tilde M^{\rho\sigma*}
 = \sum\s{pol.}|\tilde M|^2 + \left[(k\cdot p'^*)(k'\cdot p) + \Hc\right].\notag
\end{align}

\subsection{Phantom Terms in the Gauge Theory}
\label{sec:no-other-term}
You may think we forget to introduce
$\ol\psi\G5\psi$�C
$\ol\psi\G5\slashed D\psi$�C
$\epsilon^{\mu\nu\rho\sigma}F^a_{\mu\nu}F^a_{\rho\sigma}$�C
$\epsilon^{\mu\nu\rho\sigma}D_\mu D_\nu F^a_{\rho\sigma}$
terms, but being a bit careful,
\begin{itemize}
 \item the first two terms are nonsense, for now we use $\PL$ and $\PR$,
 \item the last term is equivalent to the third term as\\[.5zw]\qquad
$\epsilon^{\mu\nu\rho\sigma}D_\mu D_\nu F^a_{\rho\sigma}
=\epsilon^{\mu\nu\rho\sigma}\dfrac12[D_\mu,D_\nu]F^a_{\rho\sigma}
=\dfrac12\epsilon^{\mu\nu\rho\sigma}F^a_{\mu\nu}F^a_{\rho\sigma}$.
\end{itemize}\vspace{.5zw}
Therefore, we have to discuss only the $\epsilon FF$ terms.
If the gauge group is simple, we can take the structure constant as totally antisymmetric, which leads these terms to fall into surface terms as:
\begin{align}
 \epsilon^{\mu\nu\rho\sigma}f^{abc}f^{ade}A^b_\mu A^c_\nu A^d_\rho A^e_\sigma
 &= \epsilon^{\mu\nu\rho\sigma}\left(-f^{acd}f^{abe}-f^{adb}f^{ace}\right)
     A^b_\mu A^c_\nu A^d_\rho A^e_\sigma\notag\\
 &= -2\epsilon^{\mu\nu\rho\sigma}f^{abc}f^{ade}A^b_\mu A^c_\nu A^d_\rho A^e_\sigma\\
 &=  0,\notag
\end{align}
\begin{align}
 \therefore\ \epsilon^{\nu\nu\rho\sigma}F_{\mu\nu}F_{\rho\sigma}
&= 4\epsilon^{\mu\nu\rho\sigma}\Pm A^a_\nu\Pr A^a_\sigma
 + 4g\epsilon^{\mu\nu\rho\sigma}f^{abc}A^a_\mu A^b_\nu\Pr A^c_\sigma\notag\\
&= 2\Pm G^\mu,&
\end{align}
where $G^\mu$ is the Chern--Simons term which is defined as
\begin{equation}
  G^\mu
:= 2\epsilon^{\mu\nu\rho\sigma}
\left( A^a_\nu\Pr A^a_\sigma+\frac13gf^{abc}A^a_\nu A^b_\rho A^c_\sigma \right)
= \epsilon^{\mu\nu\rho\sigma}
\left( A^a_\nu F^a_{\rho\sigma}-\frac13gf^{abc}A^a_\nu A^b_\rho A^c_\sigma \right).
\end{equation}
See Appendix~\ref{sec:cs-instanton} for the instanton effect.

\subsection{Instanton}
\label{sec:cs-instanton}
\TODO{Postpone...}
%%% Local Variables:
%%% TeX-master: "CheatSheet.tex"
%%% End:
