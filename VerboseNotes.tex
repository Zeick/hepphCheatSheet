%#!./compile CheatSheet
%%% Time-Stamp: <2010-11-20 21:54:56 misho>
%%% �����R�[�h��Shift-JIS�ł����

\section{Verbose Notes}
\subsection{Spinor Fields}
\def\redblue#1#2{({\RED{#1}},{\BLUE{#2}})}

\subsubsection{Lorentz group and Lorentz algebra}
\label{sec:verbose-lorentz-group}
\begin{tabular}{l@{ :\ \ \ }l}
Metric &  ${\RED\eta} = \diag(+1, -1, -1, -1),\qquad {\BLUE \eta} = \diag(-1,+1,+1,+1).$\\
Lorentz transf. in $\mathbb R^{1,3}$ & Linear transf. $x^\mu \mapsto \Lambda\T^\mu_\nu x^\nu$ which conserve $x^2$.\\
 & \then $\Hrsd = \Hmnd \Lambda\T^\mu_\rho \Lambda\T^\nu_\sigma.$
 and form a group $L$.\\
 & \qquad
   ($\then\quad
     (\Lambda^{-1})\T^\mu_\nu=\eta_{\nu\alpha}\eta^{\mu\beta}\Lambda\T^\alpha_\beta=:\Lambda\T_\nu^\mu\quad\then\quad\Lambda\T^\mu_\nu\Lambda\T_\mu^\rho=\delta^\rho_\nu$)\\[.8zw]

Disconnected parts of $L$ &
  $L_0 := \left\{ \det\Lambda = +1 \land \Lambda\T_0^0 >0\right\} \quad
   L_P := \left\{ \det\Lambda = -1 \land \Lambda\T_0^0 >0\right\}$\\
& $L_T := \left\{ \det\Lambda = +1 \land \Lambda\T_0^0 <0\right\} \quad
   L_{PT} := \left\{ \det\Lambda = -1 \land \Lambda\T_0^0 <0\right\}$\\
& \qquad ($L_0$ is identical with $\redblue{\gSO(1,3)}{\gSO(3,1)}$. )\\[1zw]
Infinitesimal one in $L_0$& $\Lambda\T^\mu_\nu = \delta^\mu_\nu + \epsilon\T^\mu_\nu$ where $\epsilon_{\mu\nu}=-\epsilon_{\nu\mu}$ (for $\eta=\eta\Lambda\Lambda$)\\[1zw]
\end{tabular}

\vspace{1zw}

$BHy>.JQ49$O(B$\epsilon\T^\mu_\nu=\spmat{
0&\beta_x&\beta_y&\beta_z\\
\beta_x & 0 & -\theta_z & \theta_y\\
\beta_y & \theta_z & 0 & -\theta_x\\
\beta_y & -\theta_y & \theta_x & 0}$$B$N7A$H$J$C$F$$$k$N$G!$2sE>@8@.;R(B$\vc J$$B$H2CB.@8@.;R(B$\vc K$$B$O(B
\begin{equation}
 \Lambda = \exp\left[\kappa\ii(\vipro\theta J +\vipro\beta K)\right]
\quad\then\quad
J_x = -\kappa\ii\spmat{0&0&0&0\\0&0&0&0\\0&0&0&-1\\0&0&1&0},\quad
K_x = -\kappa\ii\spmat{0&1&0&0\\1&0&0&0\\0&0&0&0\\0&0&0&0}\label{eq:explicit_JK}
\end{equation}
$B$N7A$G$"$k!#$3$3$G(B$\kappa=\pm1$$B$O(Bnotation$B$G$"$k!#(B

$B0lJ}!$Hy>.JQ49$+$i@8@.;R$r(B
 $\epsilon\T^\mu_\nu  =: \mp\frac\ii2\epsilon^{\rho\sigma}(J_{\rho\sigma})\T^\mu_\nu$
$B$HDj5A$9$k$H!$7WNL$K$h$i$:$K(B$\epsilon_{\mu\nu}$$B$OH?BP>N$H$J$j!$(B
\begin{align*}
  \vc\theta &= \redblue+-(\epsilon^{23},\epsilon^{31},\epsilon^{12}),&
  \vc\beta  &=
\redblue+-(\epsilon^{10},\epsilon^{20},\epsilon^{30}).\\
&=\redblue+-(\epsilon_{23},\epsilon_{31},\epsilon_{12})&
&=\redblue-+(\epsilon_{10},\epsilon_{20},\epsilon_{30})
\end{align*}
$B$G!$=>$C$F(B$J^{\rho\sigma}$$B$bH?BP>N!#(B
$ (J_{\rho\sigma})\T^\mu_\nu =
\pm\ii\left(\delta^\mu_\rho\Hsnd-\delta^\mu_\sigma\Hrnd\right)$
$B$H$J$j!$8r494X78$,F@$i$l!$$3$l$,JD$8$F$$$k$N$G(BLie$BBe?t$G$"$k$3$H$b$o$+$k!'(B
\begin{equation}
 [J_{\mu\nu},J_{\rho\sigma}] = \mp\ii(
    \Hmrd J_{\nu\sigma} + \Hnsd J_{\mu\rho} - \Hmsd J_{\nu\rho} - \Hnrd J_{\mu\sigma}).
\end{equation}

$B@8@.;R$N6qBN7A$O7WNL$K0MB8$7!$(B
$
 \displaystyle J\T_1_0^\mu_\nu=\redblue{\pm\ii}{\mp\ii}
 \spmat{0&1&0&0\\1&0&0&0\\0&0&0&0\\0&0&0&0}_{\mu\nu},
 \displaystyle J\T_2_3^\mu_\nu=\redblue{\pm\ii}{\mp\ii}
 \spmat{0&0&0&0\\0&0&0&0\\0&0&0&-1\\0&0&1&0}_{\mu\nu}
$$B$H$J$k!#(B
$B$h$C$F!$$3$3$G$NJ#9f$N<h$jJ}$H(B$\kappa$$B$*$h$S7WNL$NDj5A$K$h$C$F!$(B$\vc J$$B!&(B$\vc K$$B$H(B$J_{\rho\sigma}$$B$NBP1~$,Dj$^$k$3$H$K$J$k!#(B
\starline

$\kappa$$B$HJ#9f$K$D$$$F(B$({\RED -,$B>e(B})$,$({\RED +,$B2<(B})$,$({\BLUE -,$B2<(B})$,$({\BLUE +,$B>e(B})$$B$r<h$l$P(B
\begin{equation}
  \vc J     = (J_{23}, J_{31}, J_{12}),\quad  \vc K = (J_{10},J_{20},J_{30});\qquad
\end{equation}
$B$H$J$k!'(B
\begin{equation}
 \Lambda = \exp\epsilon = \exp\left[\kappa\ii(\vipro\theta J +\vipro\beta K)\right]
         = \exp\left[\mp\ii(\epsilon^{\rho\sigma}J_{\rho\sigma})/2\right].
\end{equation}

\newpage
\subsubsection{Lorentz group and $\gSL(2,\Complex)$}
$B<!$K!$O"7k(BLie$B72(B$L_0$$B$,!$O"7k(BLie$B72(B$\gSL(2,\Complex)/\gZ_2$$B$HF17?$G$"$k$3$H$r8+$k!'(B
\begin{align}
 \aSL(2,\Complex) :=& \{a\in \aGL(2,\Complex)\ |\ \Tr(a)=0\},&
 \gSL(2,\Complex) :=& \{g\in \gGL(2,\Complex) |\ \det(g)=1\}.
\end{align}
$B$^$:(B,$\Sm$$B$r!J6K$a$F0lHLE*$K!K(B $\Sm:=(\alpha1,\beta\vc\sigma)$$B$HDj5A$9$k(B($\alpha=\beta=\pm1$)$B!#(B
$x^2 = ({\RED+}{\BLUE-})\det(x_\mu\Sm)$$B$J$N$G(B
\begin{equation}
 f^g: (x_\mu\Sm)\mapsto g (x_\mu\Sm) g^\dagger;\quad g\in\gSL(2,\Complex)
\end{equation}
$B$O(B$x^2$$B$rJ]B8$9$k!#$h$C$F(BLorentz$BJQ49$G$"$j!$@8@.;R$rHf$Y$k$3$H$G6I=jF17?$@$H$o$+$k!'(B
\begin{rightnote}
$\gSL(2,\Complex)\ni g = \exp(-\ii a)$$B$H$7$F(B
$x_\mu (g \Sm g^\dagger) = \Lambda\T_\mu^\nu x_\nu \Sm$$B$rHy>.E83+$9$k$H(B
\begin{equation}
\Lambda\T^\mu_\nu\Sn=g^{-1}\Sm (g^{-1})^\dagger\quad\then\quad
\epsilon\T^\mu_\nu\Sn=\ii(a\Sm-\Sm a^\dagger)
\end{equation}
$B$G$"$j!$$3$3$+$i(B$g$$B$,$o$+$k!'(B
\begin{equation}
 g=\exp\left(-\frac{\ii}2\vipro\theta\sigma-\frac{\alpha\beta}2\vipro\beta\sigma\right).
\end{equation}
\end{rightnote}

$B$3$N$3$H$rJL$N4QE@$+$i8+$k!#(BLorentz$B72$N@8@.;R$N8r494X78$r8+$k$H!$!J@5$7$/J#9f$r<h$C$?>l9g!K(B
\begin{equation}
  [J_i, J_j]= \ii\epsilon_{ijk}J_k,\qquad
 [J_i, K_j]= \ii\epsilon_{ijk}K_k,\qquad
 [K_i, K_j]= -\ii\epsilon_{ijk}J_k
\end{equation}
$B$H$J$k$N$G!$(B
\begin{equation}
  \vc A:=\frac{1}2(\vc J+\ii\vc K), \qquad \vc B := \frac{1}2(\vc J-\ii\vc K).
\end{equation}
$B$HDj5A$9$k$H(B
\begin{equation}
  [A_i, A_j] = \ii\epsilon_{ijk}A_k,\qquad
 [B_i, B_j] = \ii\epsilon_{ijk}B_k,\qquad
 [A_i, B_j] = 0,
\end{equation}
$B$H$J$j!$(BLorentz$B72$,(B$\gSU(2)\times\gSU(2)$$B$KJ,2r$G$-$k!#(B

\subsection{Weyl Spinor}
$\gSU(2)_A\times\gSU(2)_B$$B$KBP$7$F(B$(1/2,0)$$BI=8=$r0Y$9$b$N$r:84,$-(Bspinor $\xi$$B!$(B$(0,1/2)$$BI=8=$r0Y$9$b$N$r1&4,$-(Bspinor $\bar\xi$$B$HDj5A$9$k!#(B
\begin{align}
 \xi&\mapsto\left(1-\frac\ii2\vipro\theta\sigma-\frac12\vipro\beta\sigma\right)\xi&
 \bar\xi&\mapsto\left(1-\frac\ii2\vipro\theta\sigma+\frac12\vipro\beta\sigma\right)\bar\xi.
\end{align}

$\alpha\beta=1$$B$H$9$k$H(B$g$$B$O:84,$-(Bspinor$B$NJQ49;R$H$J$k!#5-9f$r(B
$\xi_\alpha\mapsto g\T_\alpha^\beta\xi_\beta$,
$\bar\xi^\dalpha\mapsto (g^\dagger)^{-1}\T^\dalpha_\dbeta\bar\xi^\dbeta$
$B$HDj5A$9$k!#(B

$B<!$K!$(B$\xi^\alpha\chi_\alpha$$B$*$h$S(B$\bar\xi_\dalpha\bar\chi^\dalpha$$B$,(Bscalar$B$H$J$k$h$&$K$7$?$$!#(B
$E=\spmat{0&1\\-1&0}$$B$H$7$F$*$/$H!$(B$-E\trans{g}E=g^{-1}$$B$h$j(B
\begin{equation}
 (\xi')^\alpha=\xi^\beta(g^{-1})\T_\beta^\alpha
              =-\xi^\beta(E_{\beta A}g\T_B^AE_{B\alpha})\qquad\therefore (-E_{\gamma\alpha})(\xi')^\alpha = -g\T_\gamma^\beta(E_{\beta\delta}\xi^\delta).
\end{equation}
$B$h$C$F!$(B$\epsilon^{12}=\epsilon_{21}=1$$B$H$7$F(B
$\xi^\alpha:=\epsilon^{\alpha\beta}\xi_\beta$$B!$(B$\xi_\alpha=\epsilon_{\alpha\beta}\xi^\beta$
$B$H$9$l$PNI$$!#(B


$BF1MM$K!$(B$\bar\xi'_\dalpha=-E(g^\dagger)^{-1}E\bar\xi_\dbeta$$B$+$i(B
$(E^{\dalpha\dbeta}\bar\xi_\dbeta)'=(g^\dagger)^{-1}{}\T^\dalpha_\dbeta (E^{\dbeta\dgamma}\bar\xi_\dgamma)$$B$H$J$k!#(B$\epsilon_{12}=\epsilon_{\dot1\dot2}$$B$H$7$F(B$\bar\xi_\dalpha:=\epsilon_{\dalpha\dbeta}\bar\xi^\dalpha$$B$H$9$k$N$,0lHLE*$G$"$k!#(B
$B$^$?!$$3$N$3$H$+$i(B$(\xi_a)^*=\bar\xi_\dalpha$$B!J0?$$$O(B$\xi^\dagger=\bar\xi$$B!K$,J,$+$k!#(B

$x_\mu\Sm\mapsto x_\mu(g\Sm g^\dagger)$$B$G$"$k$N$G!$(B
$x_\mu(\xi^\alpha \Sm \bar\chi^\dalpha)$$B$O(Bscalar$B$G$"$k!#$h$C$F!$(B$\Sm_{\alpha\dalpha}$$B$N$h$&$K=q$1$k!#(B
$B99$K$3$3$G(B$x_\mu(\bar\xi_\dalpha\bSm{}^{\dalpha\alpha}\chi_\alpha)$$B$b(Bscalar$B$H$J$k$h$&$K(B$\bar\sigma$$B$rDj$a$h$&!#(B
\begin{equation}
 \bar\xi_\dalpha\bSm{}^{\dalpha\alpha}\chi_\alpha=-\epsilon_{\alpha\beta}\epsilon_{\dalpha\dbeta}\chi^\beta\bSm{}^{\dalpha\alpha}\bar\xi^\dbeta
\qquad\therefore \Sm_{\beta\dbeta}\propto \epsilon_{\alpha\beta}\epsilon_{\dalpha\dbeta}\bSm{}^{\dalpha\alpha}
\end{equation}
$B$G$"$j!$$"$H$O(Bconvention$B$G$"$k!#(B

\begin{conclusion}{}
\begin{align*}
  \epsilon^{12}=\epsilon_{21}=\epsilon^{\dot1\dot2}=\epsilon_{\dot2\dot1}&=1,&
\xi^\alpha&:=\epsilon^{\alpha\beta}\xi_\beta,& \xi_\alpha&=\epsilon_{\alpha\beta}\xi^\beta,&
\bar\xi^\dalpha&:=\epsilon^{\dalpha\dbeta}\bar\xi_\dbeta,& \bar\xi_\dalpha&=\epsilon_{\dalpha\dbeta}\bar\xi^\dbeta
\end{align*}\vspace{-2.8zw}
\begin{align*}
  \xi_\alpha      &\mapsto g\T_\alpha^\beta\xi_\beta,
& \xi^\alpha      &\mapsto \xi^\beta (g^{-1})\T_\beta^\alpha,
& \bar\xi_\dalpha &\mapsto \bar\xi_\dbeta(g^\dagger)\T^\dbeta_\dalpha
& \bar\xi^\dalpha &\mapsto (g^\dagger)^{-1}\T^\dalpha_\dbeta\bar\xi^\dbeta
\end{align*}\vspace{-2.8zw}
\begin{align*}
  \bSm{}^{\dalpha\alpha} &:= \epsilon^{\alpha\beta}\epsilon^{\dalpha\dbeta}\Sm_{\beta\dbeta}&
 \Sm_{\alpha\dalpha}&= \epsilon_{\alpha\beta}\epsilon_{\dalpha\dbeta}\bSm{}^{\dbeta\beta}&
 \therefore \Sm:=(\alpha1,\beta\vc\sigma),\quad\bSm:=(\alpha1,-\beta\vc\sigma)
\end{align*}
\end{conclusion}


\subsection{Polarization Sum}\label{Sec:Verbose:PolarizationSum}
Firstly we focus on the single photon case $M=\epsilon_\mu^*(k)\epsilon_\nu'^*(k')M^{\mu\nu}$.
Here we set $k=(E,0,0,E)$, and $\epsilon=(0,1,0,0)\oplus(0,0,1,0)$. Then
\begin{equation}
  \sum\s{pol.}\left|M\right|^2
= \sum\s{pol.}\epsilon_\mu^*(k)\epsilon_\nu(k)M^{\mu}M^{\nu*}
= |M^1|^2+|M^2|^2,
\end{equation}
while
\begin{equation}
 \Hmnd M^{\mu}M^{\nu*} = |M^1|^2+|M^2|^2
\end{equation}
for Ward identity $k_\mu M^\mu=0$. Therefore the replacement
\begin{equation}
 \sum\s{pol.}\epsilon_\mu\epsilon'_\nu \to \Hmnd
\end{equation}
is valid.

Secondly we think about the double photons case
\footnote{This part is derived from $B_@8}9,0l(B's notebook.}
$M=\epsilon_\mu^*(k)\epsilon_\nu'^*(k')M^{\mu\nu}$.
Here we set
\begin{align}
 k &=(E,0,0, E) & \epsilon &=(0,1,0,0)\oplus(0,0,1,0)\\
 k'&=(E,0,0,-E) & \epsilon'&=(0,\cos\theta,\sin\theta,0)\oplus(0,-\sin\theta,\cos\theta,0).
\end{align}
Then doing some simple calculations, we can get
\begin{equation}
  \sum\s{pol.}\left|M\right|^2 = |M^{11}|^2+|M^{12}|^2+|M^{21}|^2+|M^{22}|^2.
\end{equation}
Nevertheless, na\"ive replacement does not work, because our Ward identities
\begin{equation}
 k_\mu\epsilon_\nu'^*(k')M^{\mu\nu} =
 \epsilon_\mu^*(k)k'_\nu M^{\mu\nu} = 0 \label{eq:WardIdentities}
\end{equation}
obviously does not help us. If we can omit $\epsilon$s from these
identities, that is if
\begin{equation}
 k_\mu M^{\mu\nu} = k'_\nu M^{\mu\nu} = 0,\label{eq:TrueIdentities}
\end{equation}
we can recover validity of the replacement:
\begin{align}
 \Hmrd\Hnsd M^{\mu\nu}M^{\rho\sigma*}
&= -\Hnsd\left(
M^{1\nu}M^{1\sigma*}+M^{2\nu}M^{2\sigma*}
\right)\\
&= |M^{11}|^2+|M^{12}|^2+|M^{21}|^2+|M^{22}|^2.
\end{align}

Then what's happening? Why this replacement is not valid?
Actually our new conditions \eqref{eq:TrueIdentities} seem to guarantee
that we are summing not only ``physical'' but also ``unphysical'' polarizations.
Meanwhile if we use some physical condition such as $\epsilon\cdot k=0$,
\eqref{eq:TrueIdentities} break down while Ward identities
\eqref{eq:WardIdentities} are still valid.

Now let's check what is happening from another viewpoint. First we suppose
$M$ satisfies our new conditions \eqref{eq:TrueIdentities}, and define
$\tilde M^{\mu\nu}$ and $\tilde M$ as
\begin{align}
 \tilde M^{\mu\nu}& := M^{\mu\nu} + k^\mu p^\nu + p'^\mu k'^\nu,\\
 \tilde M         & := \epsilon_\mu^*(k)\epsilon_\nu'^*(k')\tilde M^{\mu\nu}.
\end{align}
This alternative amplitude satisfies Ward identities (since photon is
massless and $\epsilon\cdot k=0$), and furthermore $\tilde M = M$.
Therefore $\tilde M$ is physically identical to $M$.
However technically these are very different, just because we cannot
perform our ``na\"ive replacement'' for this $\tilde M$:
\begin{align}
 \Hmrd\Hnsd \tilde M^{\mu\nu}\tilde M^{\rho\sigma*}
 &=  \Hmrd\Hnsd
\left(M^{\mu\nu} + k^\mu p^\nu + p'^\mu k'^\nu \right)
\left(M^{\rho\sigma*} + k^{\rho} p^{\sigma*} + p'^{\rho*}k'^{\sigma} \right)\\
 &= \sum\s{pol.}|M|^2 + \left[(k\cdot p'^*)(k'\cdot p) + \Hc\right].
\end{align}
After all, we have obtained following expression:
\begin{align}
 \sum\s{pol.}|\tilde M|^2 &=
 \sum\s{pol.}|M|^2\qquad \text{(Furthermore $\tilde M=M$)}\notag\\
&= \sum\s{pol.}|\epsilon_\mu^*(k)\epsilon_\nu'^*(k')M^{\mu\nu}|^2
 = \sum\s{pol.}|\epsilon_\mu^*(k)\epsilon_\nu'^*(k')\tilde M^{\mu\nu}|^2\\
&= \Hmrd\Hnsd M^{\mu\nu}M^{\rho\sigma*}\notag\\
&\ne \Hmrd\Hnsd \tilde M^{\mu\nu}\tilde M^{\rho\sigma*}
 = \sum\s{pol.}|\tilde M|^2 + \left[(k\cdot p'^*)(k'\cdot p) + \Hc\right].\notag
\end{align}

\subsection{Phantom Terms in the Gauge Theory}
\label{sec:no-other-term}
You may think we forget to introduce
$\ol\psi\G5\psi$$B!$(B
$\ol\psi\G5\slashed D\psi$$B!$(B
$\epsilon^{\mu\nu\rho\sigma}F^a_{\mu\nu}F^a_{\rho\sigma}$$B!$(B
$\epsilon^{\mu\nu\rho\sigma}D_\mu D_\nu F^a_{\rho\sigma}$
terms, but being a bit careful,
\begin{itemize}
 \item the first two terms are nonsense, for now we use $\PL$ and $\PR$,
 \item the last term is equivalent to the third term as\\[.5zw]\qquad
$\epsilon^{\mu\nu\rho\sigma}D_\mu D_\nu F^a_{\rho\sigma}
=\epsilon^{\mu\nu\rho\sigma}\dfrac12[D_\mu,D_\nu]F^a_{\rho\sigma}
=\dfrac12\epsilon^{\mu\nu\rho\sigma}F^a_{\mu\nu}F^a_{\rho\sigma}$.
\end{itemize}\vspace{.5zw}
Therefore, we have to discuss only the $\epsilon FF$ terms.
If the gauge group is simple, we can take the structure constant as totally antisymmetric, which leads these terms to fall into surface terms as:
\begin{align}
 \epsilon^{\mu\nu\rho\sigma}f^{abc}f^{ade}A^b_\mu A^c_\nu A^d_\rho A^e_\sigma
 &= \epsilon^{\mu\nu\rho\sigma}\left(-f^{acd}f^{abe}-f^{adb}f^{ace}\right)
     A^b_\mu A^c_\nu A^d_\rho A^e_\sigma\notag\\
 &= -2\epsilon^{\mu\nu\rho\sigma}f^{abc}f^{ade}A^b_\mu A^c_\nu A^d_\rho A^e_\sigma\\
 &=  0,\notag
\end{align}
\begin{align}
 \therefore\ \epsilon^{\nu\nu\rho\sigma}F_{\mu\nu}F_{\rho\sigma}
&= 4\epsilon^{\mu\nu\rho\sigma}\Pm A^a_\nu\Pr A^a_\sigma
 + 4g\epsilon^{\mu\nu\rho\sigma}f^{abc}A^a_\mu A^b_\nu\Pr A^c_\sigma\notag\\
&= 2\Pm G^\mu,&
\end{align}
where $G^\mu$ is the Chern--Simons term which is defined as
\begin{equation}
  G^\mu
:= 2\epsilon^{\mu\nu\rho\sigma}
\left( A^a_\nu\Pr A^a_\sigma+\frac13gf^{abc}A^a_\nu A^b_\rho A^c_\sigma \right)
= \epsilon^{\mu\nu\rho\sigma}
\left( A^a_\nu F^a_{\rho\sigma}-\frac13gf^{abc}A^a_\nu A^b_\rho A^c_\sigma \right).
\end{equation}
See Appendix~\ref{sec:cs-instanton} for the instanton effect.

\subsection{$BML(B-Mills Theory}\label{sec:yang-mills-theory}

\subsubsection{General Gauge Theory}\vskip-4pt
For any Lie group $G$, we can consider ``gauge transformation'' $\phi(x) \mapsto V(x)\phi(x)$, where $V:\Real^{1,3}\to G$. Also we can define a ``connection field'' $A_\mu(x)$ as:
\begin{align}
 \phi_\parallel (x+\dd x):= \phi(x)+\ii g A_\mu(x)\phi(x)\dd x^\mu
\qquad\text{s.t.}\quad \phi_\parallel(x+\dd x)\mapsto V(x+\dd x)\phi_\parallel(x+\dd x).
\end{align}
Then the covariant derivative $\Dm$ can be defined as
\begin{equation}
 \Dm\phi(x)\dd x^\mu := \Delta_{\dd x}\phi(x):= \phi(x+\dd x)-\phi_\parallel(x+\dd x)\qquad
\therefore  \Dm := \Pm - \ii g A_\mu.
\end{equation}
Note that $\Delta_{\dd x}\phi(x)\mapsto V(x+\dd x)\Dm\phi(x)\dd x^\mu$, which means
$\Dm\phi(x)\mapsto V(x)\Dm\phi(x)$.
Now we can see
\begin{align}
 \phi&\mapsto V\phi,&
  A_\mu&\mapsto V\left(A_\mu + \frac\ii g \partial_\mu\right)V^{-1},&
 \Dm&\mapsto V\Dm V^{-1}.\label{eq:gaugetransf}
\end{align}
We can do another discussion: we can define $\Dm$ as a kind of derivative which satisfies \eqref{eq:gaugetransf}{}.

Next we introduce the curvature tensor, or ``field strength'' as
\begin{align}
& \Delta\phi(x):=\phi_{\parallel}^{xy}(x+\dd x+\dd y)-\phi_{\parallel}^{yx}(x+\dd x+\dd y)
= \left[\Dm,\Dn\right]\phi(x)\dd x^\mu\dd y^\nu
=: -\ii g F_{\mu\nu}\phi(x)\dd x^\mu\dd y^\nu;\\
&F_{\mu\nu}(x):=\frac{\ii}g\left[\Dm,\Dn\right]
 = \Pm A_\nu(x)-\Pn A_\mu(x)-\ii g\left[A_\mu(x),A_\nu(x)\right].
\end{align}
$\Delta\phi(x)$ is transformed in terms of $V(x+\dd x+\dd y)\simeq V(x)$, thus $F_{\mu\nu}(x)\mapsto V(x)F_{\mu\nu}(x)V^{-1}(x)$.



\subsubsection{Compact Gauge Theory}
\subparagraph{Generators}
If the gauge group $G$ is {\bf compact}, it has a finite-dimensional
unitary representation. The generators $T_a$ can be taken to be Hermitian, and $V(x) = \exp\left[\ii g\theta^a(x) T^a\right]$ for $\theta^a(x)\in\Real$;
\begin{align}
 [T^a,T^b] &= \ii f\T^a^b_c T^c\quad(f\in\Real)&
 0 &=f\T^D_a_b f\T^E_D_c + f\T^D_c_a f\T^E_D_b  + f\T^D_b_c f\T^E_D_a
\end{align}

For the sake of the compactness Killing form is positive-definite, where we can normalize the generators as $\Tr T^aT^b=\frac12\delta^{ab}$, and the structure constant $f^{abc}$ would be totally antisymmetric.

\subparagraph{Adjoint Representations}
\begin{equation}
 [\tilde T^a]\T_i^j:=-\ii f^{aij};\qquad
[\tilde\Dm]\T_i^j:=\delta_i^j\Pm+gf^{iaj}A^a_\mu.
\end{equation}

\subparagraph{Field Expansion}
In this {\em normalized Hermitian} basis, the relations would be\footnote{We can expand $A_\mu$ in $T^a$-basis, because it is induced by the gauge transformation.}
\begin{align*}
\phi'&=\ee^{\ii gT^a\theta^a}\phi;&
F_{\mu\nu}^a  &= \Pm A_\nu^a-\Pn A_\mu^a + gf^{abc}A_\mu^bA_\nu^c\\
A_\mu^a{}'&\simeq A_\mu^a+\Pm\theta^a+gf^{abc}A_\mu^b\theta^c&
F^a_{\mu\nu}{}'&=[\ee^{\ii g\theta^c\tilde T^c}]^{ab}F^b_{\mu\nu}\\
         &=A_\mu^a+(\tilde\Dm\theta)^a,&
&\simeq F_{\mu\nu}^a+gf^{abc}F^b_{\mu\nu}\theta^c.
\end{align*}

\subparagraph{Covariant Derivative}
For a field $\lambda^a$ under the adjoint representation,
\begin{equation}
 (\Dm\lambda)^a   = \Pm\lambda^a    + g f^{abc} A_\mu^b \lambda^c\qquad\text{or}\quad
  \Dm\lambda^aT^a = \Pm\lambda^aT^a - \ii g [A_\mu^bT^b,\lambda^aT^a]\ \footnotemark
\end{equation}
\footnotetext{Note that we can use any representation $T^a$ but must the same ones for $A^a_\mu T^a$ and $\lambda^a T^a$.}

\subparagraph{Bianchi Equation}
\begin{equation}
 \epsilon^{\mu\nu\rho\sigma}\left[\Dn,\left[\Dr,\Ds\right]\right] = \epsilon^{\mu\nu\rho\sigma}\Dn F_{\rho\sigma} = 0.
\end{equation}

\newpage







\subsection{Spinor}
\Paragraph{$\eta^{\mu\nu}=(-,+,+,+)$ case}
\begin{tabular}[t]{l@{\ :\ }l}
 Grassmann Number
& $(ab)^\dagger=b^\dagger a^\dagger$ for $a,b\in\Grassmann$\\
& \then for $a,b\in\RealGrassmann$, $ab\in\ii\RealGrassmann$\\
 $\gamma$ matrix
& $\displaystyle\{\Gm,\Gn\}=2\minkow\mu\nu\cdot\vc1$\\
& $\displaystyle\G{\mu\nu} = \frac12\left(\Gm\Gn-\Gn\Gm\right)$
  \quad etc...\\
& $\displaystyle(\ii\G0)^\dagger:=\ii\G0,\quad\G i^\dagger:=\G i$\\
 Dirac Conjugate
& $\bar\psi=\ii\psi^\dagger\G0$
\end{tabular}

\TODO{SPINOR}
\subsection{Instanton}
\label{sec:cs-instanton}
\TODO{INSTANTON}





%%% Local Variables:
%%% TeX-master: "CheatSheet.tex"
%%% End: